
\chapter{Fundamentação Teórica}\label{cap_exemplos}

Este capítulo estabelece o contexto teórico e empírico para o estudo, fundamentando a análise no conhecimento acadêmico existente.

\section{O ENEM no Cenário Educacional Brasileiro}\label{sec-divisoes}

O Exame Nacional do Ensino Médio (ENEM) teve sua primeira edição em 1998, contando com a participação de aproximadamente 115 mil participantes. Na época, suas notas só eram utilizadas por 2 instituições de ensino superior, saltando para 93 instituições no ano seguinte. A importância do ENEM cresce com o passar dos anos, alcançando a marca de mais de 1 milhão de participantes na sua quarta edição e tornando-se uma das principais formas de acesso ao ensino superior, com a criação do Programa Universidade Para Todos (ProUni) em 2005. \cite{INEP1}

Em 2009, com a criação do Sistema de Seleção Unificada (SISU), o ENEM foi reformulado e assume o formato que tem hoje: 180 questões objetivas divididas em 4 áreas do conhecimento e uma redação. No ano seguinte, os resultados do ENEM passaram a ser adotados pelo Fundo de Financiamento Estudantil (Fies) e em 2013, quase todas as instituições federais adotam o ENEM como critério de seleção. 2 universidade portuguesas, a Universidade de Coimbra e Universidade de Algrave, adotam o ENEM como critério de seleção em 2014, número que chega a 35 instituições portuguesas em 2018. \cite{INEP1}

É evidente que o ENEM deixa de ser apenas uma ferramenta de avaliação e transforma em um instrumento multifacetado que desempenha um papel central na trajetória educacional dos jovens brasileiros. Além de aferir o desempenho dos estudantes ao final do ensino médio, o ENEM serve como a principal porta de acesso ao ensino superior, sendo a base para o SISU, o ProUni e o Fies. \cite{INEP2} Essa centralidade significa que qualquer fator que influencie o desempenho no exame tem um impacto direto e significativo nas oportunidades de acesso ao ensino superior e, consequentemente, na mobilidade social dos indivíduos.

Os microdados do ENEM, disponibilizados anualmente pelo Instituto Nacional de Estudos e Pesquisas Educacionais Anísio Teixeira (INEP), representam uma fonte de informação rica e valiosa para pesquisas educacionais. \cite{ref_05} Esses dados detalhados permitem uma compreensão aprofundada dos padrões de desempenho, das características socioeconômicas dos participantes e dos contextos escolares, possibilitando análises complexas sobre as disparidades educacionais no país.

\section{Teorias sobre Desigualdades Educacionais: O Capital Cultural de Bourdieu}\label{sec-divisoes}

Para compreender a reprodução das desigualdades sociais no sistema educacional, a teoria do capital cultural de Pierre Bourdieu oferece um arcabouço teórico fundamental. \cite{ref_01} Bourdieu argumenta que o sucesso escolar não depende apenas do mérito individual ou da capacidade cognitiva, mas também da posse de diferentes formas de capital: o econômico (posses que o indivíduo tem), social ()




O capital cultural, em particular, manifesta-se em três estados: incorporado (disposições duradouras do corpo e da mente, como hábitos e habilidades adquiridas na família), objetivado (bens culturais como livros, obras de arte) e institucionalizado (títulos acadêmicos, diplomas). \cite{ref_10}

A posse de capital cultural, juntamente com o capital social (redes de relacionamentos e recursos que delas advêm), pode influenciar significativamente o desempenho escolar e o acesso a bens educacionais. Se os dados do ENEM confirmarem a forte influência de variáveis socioeconômicas e de escolaridade parental, isso reforçará a tese da reprodução escolar das desigualdades, sugerindo que o sistema educacional, em vez de ser um equalizador, pode perpetuar as hierarquias sociais. Isso se manifesta, por exemplo, na forma como a escolaridade da mãe e a renda familiar são fatores relevantes para o desempenho e a dispersão das notas do ENEM. \cite{ref_01}













































% exemplo






\section{Resultados de comandos}\label{sec-divisoes}

% ---
\subsection{Tabelas e quadros}

O \textbf{Tutorial do Pacote USPSC para modelos de trabalhos de acad\^emicos em LaTeX - vers\~ao 3.2} apresenta orientações completas e diversas formatações de tabelas, dentre elas a \autoref{tab-ibge}, que é um exemplo de tabela alinhada que pode ser longa ou curta, conforme padrão do Instituto Brasileiro de Geografia e Estatística (IBGE).

%\begin{table}[H]
\begin{table}[htb]
	\IBGEtab{%
		\caption{Frequência anual por categoria de usuários}%
		\label{tab-ibge}
	}{%
		\begin{tabular}{ccc}
			\toprule
			Categoria de Usuários & Frequência de Usuários \\
			\midrule \midrule
			Graduação & 72\% \\
			\midrule 
			Pós-Graduação & 15\% \\
			\midrule 
			Docente & 10\% \\
			\midrule 
			Outras & 3\% \\
			\bottomrule
		\end{tabular}%
	}{%
		\fonte{Elaborada pelos autores.}%
		\nota{Exemplo de uma nota.}%
		\nota[Anotações]{Uma anotação adicional, que pode ser seguida de várias
			outras.}%
		
	}
\end{table}


A formatação do quadro é similar à tabela, mas deve ter suas laterais fechadas e conter as linhas horizontais.
\newpage

% o comando \newpage foi utilizado para forçar a quebra de página

\begin{quadro}[htb]
	\caption{\label{quadro_modelo}Níveis de investigação}
	\begin{tabular}{|p{2.6cm}|p{6.0cm}|p{2.25cm}|p{3.40cm}|}
		\hline
		\textbf{Nível de Investigação} & \textbf{Insumos}  & \textbf{Sistemas de Investigação}  & \textbf{Produtos}  \\
		\hline
		Meta-nível & Filosofia\index{filosofia} da Ciência  & Epistemologia &
		Paradigma  \\
		\hline
		Nível do objeto & Paradigmas do metanível e evidências do nível inferior &
		Ciência  & Teorias e modelos \\
		\hline
		Nível inferior & Modelos e métodos do nível do objeto e problemas do nível inferior & Prática & Solução de problemas \\ 
		\hline
	\end{tabular}
	\begin{flushleft}
		%\fonte{\citeonline{van1986}}
		Fonte: \citeonline{van1986}
	\end{flushleft}
\end{quadro} 


No \textbf{Tutorial do Pacote USPSC para modelos de trabalhos de acad\^emicos em LaTeX - vers\~ao 3.2} são apresentados mais exemplos de quadros.

% ---
\subsection{Figuras}\label{sec_figuras}
% ---
\index{figuras}Figuras podem ser criadas diretamente em \LaTeX,
como o exemplo da \autoref{fig_circulo}.

\begin{figure}[htb]
	\caption{\label{fig_circulo}A delimitação do espaço}
	\begin{center}
		\setlength{\unitlength}{9cm}
		\begin{picture}(1,1)
		\put(0,0){\line(0,1){1}}
		\put(0,0){\line(1,0){1}}
		\put(0,0){\line(1,1){1}}
		\put(0,0){\line(1,2){.5}}
		\put(0,0){\line(1,3){.3333}}
		\put(0,0){\line(1,4){.25}}
		\put(0,0){\line(1,5){.2}}
		\put(0,0){\line(1,6){.1667}}
		\put(0,0){\line(2,1){1}}
		\put(0,0){\line(2,3){.6667}}
		\put(0,0){\line(2,5){.4}}
		\put(0,0){\line(3,1){1}}
		\put(0,0){\line(3,2){1}}
		\put(0,0){\line(3,4){.75}}
		\put(0,0){\line(3,5){.6}}
		\put(0,0){\line(4,1){1}}
		\put(0,0){\line(4,3){1}}
		\put(0,0){\line(4,5){.8}}
		\put(0,0){\line(5,1){1}}
		\put(0,0){\line(5,2){1}}
		\put(0,0){\line(5,3){1}}
		\put(0,0){\line(5,4){1}}
		\put(0,0){\line(5,6){.8333}}
		\put(0,0){\line(6,1){1}}
		\put(0,0){\line(6,5){1}}
		\end{picture}
	\end{center}
	\legend{Fonte: \citeonline{equipeabntex2}}
\end{figure}

Consulte o \textbf{Tutorial do Pacote USPSC para modelos de trabalhos de acad\^emicos em LaTeX - vers\~ao 3.2} para conhecer mais recursos referentes à figuras. 

% ---
\section{Divisões do documento}\label{sec-divisoes-b}
Esta seção exemplifica o uso de divisões de documentos em conformidade com a ABNT NBR 6024  \cite{nbr6024}.
% ---
% ---
\subsection{Divisões do documento: subseção}\label{sec-divisoes-subsection}
% ---

Um exemplo de seção é a \autoref{sec-divisoes-b}. Esta é a \autoref{sec-divisoes-subsection}.

\subsubsection{Divisões do documento: subsubseção}\label{sec-divisoes-subsubsection}

Isto é uma \texttt{subsubsection} do \LaTeX, mas é denominada de ``subseção'' porque no português não temos a palavra ``subsubseção''.

\subsubsection{Divisões do documento: subsubseção}

Isto é outra subsubseção.

\subsection{Divisões do documento: subseção}\label{sec-exemplo-subsec}

Isto é uma subseção.

\subsubsection{Divisões do documento: subsubseção}

Isto é mais uma subsubseção da \autoref{sec-exemplo-subsec}.


\subsubsubsection{Esta é uma subseção de quinto
nível}\label{sec-exemplo-subsubsubsection}

Esta é uma seção de quinto nível. Ela é produzida com o seguinte comando:

\begin{verbatim}
\subsubsubsection{Esta é uma subseção de quinto
nível}\label{sec-exemplo-subsubsubsection}
\end{verbatim}

\subsubsubsection{Esta é outra subseção de quinto nível}\label{sec-exemplo-subsubsubsection-outro}

Esta é outra seção de quinto nível.


\paragraph{Este é um parágrafo numerado}\label{sec-exemplo-paragrafo}

Este é um exemplo de parágrafo numerado. Ele é produzido com o comando de
parágrafo:

\begin{verbatim}
\paragraph{Este é um parágrafo numerado}\label{sec-exemplo-paragrafo}
\end{verbatim}

A numeração entre parágrafos numerados e subsubsubseções são contínuas.

\paragraph{Esta é outro parágrafo numerado}\label{sec-exemplo-paragrafo-outro}

Este é outro parágrafo numerado.

% ---
\subsection{Este é um exemplo de nome de subseção longa que se aplica a seções e demais divisões do documento. Ele deve estar alinhado à esquerda e a segunda e demais linhas devem iniciar logo abaixo da primeira palavra da primeira linha} 

Observe que o alinhamento do título obedece esta regra também no sumário.
	






