
\chapter[Introdução]{Introdução}
\label{introducao}


O Exame Nacional do Ensino Médio (ENEM) consolidou-se, na última década, como a principal avaliação educacional do Ensino Médio no Brasil, transcendendo seu papel inicial de termômetro da qualidade da educação básica para se tornar a porta de entrada para o ensino superior em instituições públicas e privadas, através de programas como o Sistema de Seleção Unificada (SISU), o Programa Universidade Para Todos (ProUni) e o Fundo de Financiamento Estudantil (Fies). Sua relevância reside na capacidade de fornecer um panorama detalhado do desempenho dos estudantes, bem como de aspectos socioeconômicos e contextuais que permeiam o ambiente escolar e familiar dos participantes.

Apesar dos esforços contínuos para aprimorar a qualidade da educação no Brasil, persistem desafios significativos, evidenciados pelas variações no desempenho dos estudantes em avaliações de larga escala como o ENEM. A literatura acadêmica aponta para a influência de múltiplos fatores nesse desempenho, que vão desde as condições socioeconômicas das famílias até as características estruturais e pedagógicas das escolas, além das peculiaridades regionais \cite{ref_01}.  A análise estatística de microdados do ENEM entre 2021 e 2023, por exemplo, revela desigualdades estruturais marcantes entre estudantes de escolas públicas e privadas \cite{ref_05}.  A persistência dessas disparidades indica que as desigualdades educacionais no Brasil não são meramente aleatórias, mas profundamente associadas às desigualdades sociais \cite{ref_04}.

A análise aprofundada dos microdados do ENEM, portanto, constitui uma oportunidade ímpar para desvendar a complexa interação entre os fatores socioeconômicos, as características do ambiente escolar e as peculiaridades regionais que moldam o desempenho dos estudantes. Isso permite ir além da simples constatação das disparidades, oferecendo um panorama mais claro de como um instrumento concebido para democratizar o acesso ao ensino superior pode, na prática, atuar como um espelho das desigualdades sociais estruturais e, em certos contextos, até mesmo contribuir para a sua perpetuação, um fenômeno consistentemente observado em análises de dados históricos \cite{ref_05}.  A compreensão desses mecanismos é vital para a formulação de políticas públicas que não apenas mitiguem as lacunas, mas que atuem nas causas-raiz das iniquidades educacionais.

Nesse contexto, este Trabalho de Conclusão de Curso propõe investigar e quantificar a influência dos principais fatores socioeconômicos, características da escola e particularidades regionais no desempenho dos estudantes no ENEM. A pergunta central que guia esta pesquisa é: ``Quais são os principais fatores socioeconômicos, características da escola e particularidades regionais que influenciam o desempenho dos estudantes no ENEM e qual a magnitude da influência de cada um desses conjuntos de fatores nas notas dos participantes?''. O objetivo geral é utilizar os microdados do exame para fornecer \textit{insights} robustos sobre a qualidade da educação básica no Brasil, contribuindo para a identificação de áreas que necessitam de maior atenção e investimento. A quantificação da influência dos fatores, por meio de modelos preditivos e análise de importância de variáveis \cite{ref_05}, é um diferencial crucial. Não se trata apenas de identificar a existência de correlações, mas de medir o grau de impacto, o que é fundamental para a formulação de políticas públicas eficazes e direcionadas.

Para tanto, buscam-se os seguintes objetivos específicos: i) Coletar, pré-processar e realizar uma análise exploratória dos microdados do ENEM \cite{ref_08} e do Censo Escolar \cite{ref_07}, selecionando as variáveis relevantes; ii) Identificar padrões, tendências e correlações entre as variáveis selecionadas e o desempenho dos estudantes; iii) Aplicar técnicas de Ciência de Dados para construir modelos preditivos e determinar a importância relativa de cada grupo de fatores; e iv) Discutir os resultados obtidos, correlacionando-os com a literatura existente e extraindo dados práticos.

A relevância desta pesquisa reside na sua capacidade de oferecer uma análise quantitativa detalhada das correlações entre múltiplos fatores e o desempenho educacional, utilizando uma vasta base de dados. Os dados gerados podem servir como subsídio para educadores, formuladores de políticas públicas e pesquisadores, auxiliando na compreensão das raízes das desigualdades educacionais e na elaboração de estratégias direcionadas para a melhoria do ensino médio no país. A pesquisa não se limita a um exercício acadêmico; ela tem um potencial transformador social ao fornecer dados concretos para subsidiar políticas públicas mais justas e fortalecer a rede pública de ensino \cite{ref_05}. 