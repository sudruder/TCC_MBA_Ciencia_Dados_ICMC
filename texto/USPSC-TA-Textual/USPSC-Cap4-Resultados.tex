
\chapter{Resultados}\label{cap_resultados}

Este capítulo apresenta os resultados obtidos a partir da aplicação da metodologia descrita no Capítulo \ref{cap_metodologia}. Os resultado serão apresentados na mesma ordem das etapas descritas na metodologia.

\section{Entendimento dos Dados}

\subsection{Tratamento dos Valores nulos}\label{sec_resultados_tratamento_nulos}

Conforme descrito na Seção \ref{sec_metodologia_tratamento_nulos}, os dados nulos foram tratados de acordo com a natureza de cada variável. A Tabela \ref{tab_obs_variaveis_enem} apresenta o percentual de valores nulos por variável na conjunto de dados integrados incial, antes dos tratamentos.

\begin{table}[h]

    \centering

    \caption{Percentual inicial de valores nulos por variável}

    \label{tab_obs_variaveis_enem}

    \begin{tabular}{|l|c|}
        \hline
        \textbf{Variável} & \textbf{Percentual de nulos} \\ \hline
        sigla\_uf\_escola & 78,1\% \\ \hline
        cod\_municipio\_escola & 78,1\% \\ \hline
        tp\_adm\_escola & 78,1\% \\ \hline
        nome\_municipio\_escola & 78,1\% \\ \hline
        cod\_uf\_escola & 78,1\% \\ \hline
        tp\_local\_escola & 78,1\% \\ \hline
        funcionamento\_escola & 78,1\% \\ \hline
        tp\_ensino & 69,8\% \\ \hline
        tp\_escola & 68,2\% \\ \hline
        nota\_ciencias\_natureza & 40,4\% \\ \hline
        nota\_matematica & 40,4\% \\ \hline
        nota\_ciencias\_humanas & 37,0\% \\ \hline
        nota\_redacao & 37,0\% \\ \hline
        nota\_linguagem\_codigos & 37,0\% \\ \hline
        03\_ocupacao\_pai & 12,5\% \\ \hline
        01\_escolaridade\_pai & 9,8\% \\ \hline
        04\_ocupacao\_mae & 9,0\% \\ \hline
        estado\_civil & 4,2\% \\ \hline
        02\_escolaridade\_mae & 3,5\% \\ \hline
        cor\_raca & 1,8\% \\ \hline
        10\_qtde\_carro & 0,6\% \\ \hline
        05\_qtde\_moradores & 0,6\% \\ \hline
        06\_renda\_familiar & 0,6\% \\ \hline
        07\_qtde\_trabalhador\_domestico & 0,6\% \\ \hline
        08\_qtde\_banheiro & 0,6\% \\ \hline
        09\_qtde\_quarto & 0,6\% \\ \hline
        18\_flag\_aspirador\_po & 0,6\% \\ \hline
        11\_qtde\_motocicleta & 0,6\% \\ \hline
        12\_qtde\_geladeira & 0,6\% \\ \hline
        13\_qtde\_freezer & 0,6\% \\ \hline
        14\_qtde\_maq\_lavar\_roupa & 0,6\% \\ \hline
        15\_qtde\_maq\_secar\_roupa & 0,6\% \\ \hline
        16\_qtde\_micro\_ondas & 0,6\% \\ \hline
        17\_qtde\_maq\_lavar\_louca & 0,6\% \\ \hline
        22\_qtde\_celular & 0,6\% \\ \hline
        19\_qtde\_tv & 0,6\% \\ \hline
        20\_flag\_aparelho\_dvd & 0,6\% \\ \hline
        21\_flag\_tv\_assinatura & 0,6\% \\ \hline
        24\_qtde\_computadores & 0,6\% \\ \hline
        23\_flag\_telefone\_fixo & 0,6\% \\ \hline
        25\_flag\_internet & 0,6\% \\ \hline
        nacionalidade & 0,05\% \\ \hline
    \end{tabular}
    \begin{center}
        \footnotesize Fonte: elaborado pelo autor.
    \end{center}
\end{table}

A quantidade de nulos nas variáveis varia significativamente, com algumas apresentando mais de 70\% de valores nulos, enquanto outras possuem menos de 1\%. Variáveis com uma alta proporção de valores nulos podem comprometer a análise se for realizado alguma imputação de valores.

Nas variáveis de notas, foi observado que há valores preenchidos com zero e valores nulos. Após análise conjunta com as variáveis de presença nas respecitivas provas, constatou-se que os valores nulos indicam ausência ou eliminação do participante na prova, enquanto os valores zero indicam que o participante esteve presente na prova, mas obteve nota zero. Dessa forma, as observações com notas nulas foram removidas, enquanto as observações com notas zero foram mantidas.

Após a aplicação dos tratamentos descritos na Seção \ref{sec_metodologia_tratamento_nulos}, restaram 12.241.442 observações e 45 variáveis no conjunto de dados integrado.

\subsection{Separação dos Conjuntos de Dados por Variável Resposta}\label{sec_resultados_separacao_conjuntos_dados}

Após a separação dos conjuntos de dados por variável resposta, conforme descrito na Seção \ref{sec_metodologia_separacao_conjuntos_dados}, foram criados cinco conjuntos de dados distintos. A Tabela \ref{tab_conjunto_variaveis_resposta} apresenta a quantidade de observações e variáveis em cada conjunto de dados.


\begin{table}[h]

    \centering

    \caption{Quantidade de observações por conjunto de dados}

    \label{tab_conjunto_variaveis_resposta}

    \begin{tabular}{|l|c|c|}
        \hline
        \textbf{Conjunto de Dados} & \textbf{Observações} & \textbf{Variáveis} \\ \hline
        Ciências Humanas & 7.895.093 & 36 \\ \hline
        Ciências Natureza & 7.500.050 & 36 \\ \hline
        Linguagem e Código & 7.895.093 & 36 \\ \hline
        Matemática & 7.500.050 & 36 \\ \hline
        Redação & 7.895.093 & 36 \\ \hline
    \end{tabular}
    \begin{center}
        \footnotesize Fonte: elaborado pelo autor.
    \end{center}
\end{table}

\subsection{Exploração Inicial}\label{sec_resultados_exploracao_inicial}

