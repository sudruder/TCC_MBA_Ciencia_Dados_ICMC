
\chapter{Resultados}\label{cap_resultados}

Este capítulo apresenta os resultados obtidos a partir da aplicação da metodologia descrita no Capítulo \ref{cap_metodologia} - Metodologia. Os resultado serão apresentados na mesma ordem das etapas descritas na metodologia.

\section{Entendimento de Negócio}\label{resultados_entendimento_negocio}

Conforme mencionado no Capítulo \ref{cap_fundamentacao} - Fundamentação Teórica, o ENEM é um exame de grande relevância no contexto educacional brasileiro e compreender os fatores que impactam o desempenho dos estudantes é crucial para a formulação de políticas educacionais eficazes.

Trabalhos anteriores citam algumas variáveis socioeconômicas como discriminadores de performance no ENEM. A Tabela \ref{tab_trabalhos_anteriores} apresenta essas variáveis identificadas na literatura, juntamente com suas respectivas referências.

\begin{longtable}{|m{7cm}|m{4cm}|}

    % --- TÍTULO E RÓTULO (1ª Página) ---
    \caption{Variáveis socioeconômicas e suas referências} \label{tab_trabalhos_anteriores} \\
    \hline
    \textbf{Variável socioeconômica} & \textbf{Referência} \\ \hline
    \endfirsthead

    % --- CABEÇALHO (Páginas Seguintes) ---
    \hline
    \textbf{Variável socioeconômica} & \textbf{Referência} \\ \hline
    \endhead

    % --- RODAPÉ (Páginas Intermediárias) ---
    \hline
    \multicolumn{2}{r}{\footnotesize\textit{Continua na próxima página...}} \\
    \endfoot

    % --- RODAPÉ FINAL (Última Página) ---
    \hline
    \multicolumn{2}{c}{\footnotesize Fonte: elaborado pelo autor.} \\
    \endlastfoot

    % --- DADOS DA TABELA ---
    Renda familiar & Melo \textit{et al.} \cite{ref_01} \newline Vasconcellos \cite{ref_06} \\ \hline
    
    Raça / Cor & Melo \textit{et al.} \cite{ref_01} \newline Moraes \textit{et al.} \cite{ref_03} \\ \hline
    
    Sexo & Moraes \textit{et al.} \cite{ref_03} \\ \hline
    
    Idade / Atraso Escolar & Jaloto e Primi \cite{ref_12} \\ \hline
    
    Administração: Pública vs. Privada & Moraes \textit{et al.} \cite{ref_03} \newline Jaloto e Primi \cite{ref_12} \newline Ortega \textit{et al.} \cite{ref_05} \\ \hline
    
    Atributos Escolares & Moraes \textit{et al.} \cite{ref_03} \\

\end{longtable}

Assim, ao avaliarmos os trabalhos anteriores disponíveis, concluímos que há uma variedade de fatores socioeconômicos que podem influenciar o desempenho dos estudantes no ENEM. Com base nisso, foram formuladas as seguintes perguntas de pesquisa:

\begin{itemize}
    \item \textbf{Pergunta 1:} Quais são os principais fatores socioeconômicos que influenciam o desempenho dos estudantes no ENEM?
    \item \textbf{Pergunta 2:} Qual é a magnitude da influência de cada um desses conjuntos de fatores nas notas dos participantes?
\end{itemize}

\section{Entendimento dos dados}\label{resultados_entendimento_dados}

\subsection{Escolha e Coleta dos Dados}\label{resultados_escolha_coleta_dados}

Como descrito no Capítulo \ref{cap_metodologia} - Metodologia, foi necessário identificar dados que fossem relevantes para responder as pergunras de pesquisa formuladas. Foi realizada uma busca por bases de dados públicas que contivessem informações detalhadas sobre os participantes do ENEM, incluindo suas características socioeconômicas e desempenho no exame.

Os microdados do ENEM, disponibilizados anualmente pelo Instituto Nacional de Estudos e Pesquisas Educacionais Anísio Teixeira (INEP), foram escolhidos como a principal fonte de dados para este trabalho e podem ser acessados através do portal do INEP \cite{ref_08}.

No mesmo portal, também estão disponíveis os dados do Censo Escolar, que fornecem informações adicionais sobre as escolas de todo o território nacional \cite{ref_07}. Esses foram escolhidos como fonte complementar por fornecerem um contexto mais amplo sobre o ambiente educacional.

Foram então selecionadas as edições de 2020 a 2024 (as últimas cinco edições disponíveis) de ambos os conjuntos de dados e os arquivos disponibilizados foram baixados através de download simples e armazenados localmente para posterior leitura e manipulação.

Como os dados escolhidos são públicos e anonimizados por quem os distribui, entendeu-se que não há limitações éticas para o uso desses dados neste trabalho e não foi necessário submeter o projeto a um comitê de ética em pesquisa.

\subsection{Compreensão Inicial dos Dados}\label{resultados_compreensao_inicial_dados}

Os arquivos de microdados do ENEM e do Censo Escolar são disponibilizados em formato compactado (.zip), separados pelo ano de aplicação do exame/censo.

Dentre os arquivos existentes nos arquivos compactados dos microdados do ENEM, foram selecionados os arquivos CSV (\textit{Comma-Separated Values}) que contêm as informações dos participantes e suas notas e os os dicionários de dados de cada edição em formato XLSX (Formato nativo do \textit{Microsoft Excel}), que foi utilizado para interpretar os valores categóricos e identificar variáveis importantes.

Para os arquivos compactados do Censo Escolar, foram selecionados os arquivos CSV que contêm as informações das escolas e os dicionários de dados em formato XLSX.

\subsubsection{Edicão de 2024 do ENEM e LGPD}\label{resultados_enem_2024_lgpd}

Na edição de 2024 dos microdados do ENEM, foi feita uma alteração no formato de disponibilização dos dados dos participantes e das notas, que passaram a ser disponibilizados em arquivos separados.

Isso se deu "Devido à vigência da Lei Geral de Proteção de Dados (LGPD), incorporada ao ordenamento jurídico
brasileiro por meio da Lei nº 13.709, de 14 de agosto de 2018" \cite{Inep2024LeiaMeEnem}, conforme descrito no arquivo auxiliar ¨Leia-Me" \cite{Inep2024LeiaMeEnem} disponibilizado junto com os microdados do ENEM 2024 .

Assim, o formato dos arquivos de microdados do ENEM 2024 difere das edições anteriores, por mais que as informações contidas permanecem as mesmas. Houve a separação dos dados dos participantes e das notas em arquivos distintos e sem uma chave primária que permita a junção dos dois conjuntos de dados. Dessa forma, os dados da edição de 2024 do ENEM não puderam ser utilizados para este trabalho

\subsection{Análise dos Dicionários de Dados}\label{resultados_analise_dicionarios_dados}

Foram analisados os dicionários de dados dos microdados do ENEM e do Censo Escolar para identificar as variáveis disponíveis em cada conjunto de dados. Os dicionários completos estão disponíveis no Apêndice \ref{apendice_dicionario_enem} e \ref{apendice_dicionario_censo_escolar}. A partir dessa análise, foi possível identificar as variáveis que seriam relevantes para responder às perguntas de pesquisa formuladas na Seção \ref{resultados_entendimento_negocio}.

Não foi possível localizar uma variável que permitisse a identificação única das escolas dos participantes do ENEM nos microdados do ENEM, o que impossibilitou correlacionar diretamente os dados dos participantes do ENEM com os dados das escolas do Censo Escolar para agregar informações das escolas aos dados dos participantes. Dessa forma, optou-se por utilizar apenas os dados dos microdados do ENEM para a realização deste trabalho.

\subsection{Definição da Variável Resposta}\label{resultados_definicao_variavel_resposta}

Como esse trabalho pretende avaliar o desempenho dos estudantes no ENEM e os fatores que influenciam esse desempenho, a variável resposta deve refletir esse objetivo. Assim, foram utilizadas como variáveis resposta as notas obtidas pelos estudantes nas quatro provas objetivas e na redação do ENEM.

Ou seja, usamos cinco variáveis resposta distintas para análise: (i) Nota da prova de Ciências da Natureza; (ii) Nota da prova de Ciências Humanas; (iii) Nota da prova de Linguagens e Códigos; (iv) Nota da prova de Matemática; e (v) Nota da Redação.

\section{Preparação dos dados}\label{resultados_preparacao_dados}

\subsection{Preparação do Ambiente Tecnológico e Analítico}\label{resultados_preparacao_ambiente_analitico}

Para a execução desse trabalho, foi utilizado um ambiente baseado em \texttt{Python} versão 3.11 através do gerenciador de ambientes virtuais Miniconda3 \cite{Anaconda_Miniconda_nd}. O computador utilizado possui uma CPU AMD Ryzen 7 9800X3D, 32 GB de memória RAM e uma GPU NVIDIA GeForce RTX 4070 Ti Super, com 16 GB de memória dedicada com sistemas operacionais Ubuntu 24.04 LTS e Windows 11 Pro.

O ambiente foi especificamente configurado com o ecossistema NVIDIA CUDAX \cite{NVIDIA_CUDAX_nd} para posibilitar a execução utilizando a GPU do equipamento, visando acelerar o processamento dos dados e a modelagem. Esta suíte de bibliotecas de software permite executar pipelines de Ciência de Dados e análises inteiramente na GPU, minimizando a transferência de dados entre a CPU e a GPU.

Foram utilizados seus principais componentes: \texttt{cudf} \cite{NVIDIA_cuDF_nd}, uma biblioteca para manipulação de \texttt{DataFrames} na GPU, análoga ao \texttt{pandas} \cite{PandasTeam_PandasDocs_2025}, e \texttt{cuml} \cite{NVIDIA_cuML_2023}, que fornece implementações de algoritmos de \textit{Machine Learning} acelerados por GPU, análoga ao \texttt{scikit-learn} \cite{scikit-learn}. Todo o ambiente foi construído sobre a plataforma CUDA 13.0, com as bibliotecas e dependências gerenciadas diretamente pelo Conda.

O arquivo YML de configuração do ambiente virtual utilizado está disponível no Apêndice \ref{apendice_ambiente_virtual}.

\subsection{Leitura dos Dados}\label{resultados_leitura_dados}

Os arquivos CSV dos microdados do ENEM foram lidos utilizando o método \texttt{read\_csv} da biblioteca \texttt{pandas} especificando o separador como ponto e vírgula (\texttt{sep = ';'}).

Serão utilizados os dados das edições de 2020 a 2023 e possuem as quantidade de observações e variáveis descritas na Tabela \ref{tab_qtde_obs_variaveis_edicao}. As tabelas foram carregadas já desconsiderando colunas que não agregam ao modelo, como o número de inscrição do participante, por exemplo.

\begin{longtable}{|c|c|c|}

    % --- TÍTULO E RÓTULO (Aparece no topo da primeira página) ---
    \caption{Quantidade de observações e variáveis por edição do ENEM} \label{tab_qtde_obs_variaveis_edicao} \\
    \hline
    \textbf{Edição} & \textbf{Observações} & \textbf{Variáveis} \\ \hline
    \endfirsthead

    % --- CABEÇALHO PARA PÁGINAS SEGUINTES (Se quebrar página) ---
    \hline
    \textbf{Edição} & \textbf{Observações} & \textbf{Variáveis} \\ \hline
    \endhead

    % --- RODAPÉ PARA PÁGINAS INTERMEDIÁRIAS (Antes de terminar) ---
    \hline
    \multicolumn{3}{r}{\footnotesize\textit{Continua na próxima página...}} \\
    \endfoot

    % --- RODAPÉ FINAL (Aparece apenas na última página) ---
    \hline
    \multicolumn{3}{c}{\footnotesize Fonte: microdados do INEP; elaborado pelo autor.} \\
    \endlastfoot

    % --- DADOS DA TABELA ---
    2020 & 5.783.109 & 52 \\ \hline
    2021 & 3.389.832 & 52 \\ \hline
    2022 & 3.476.105 & 52 \\ \hline
    2023 & 3.933.955 & 52 \\

\end{longtable}

\subsection{Integração dos Dados}\label{resultados_integracao_dados}

Analisando o dicionário de dados de cada edição, foi possível observar que todas as edições possuem o mesmo esquema, ou seja, as mesmas variáveis com os mesmos nomes estão presentes em todas as edições selecionadas. Assim, a integração entre edições foi realizada por meio da concatenação vertical dos quatro conjuntos de dados, utilizando o método \texttt{concat} da biblioteca \texttt{pandas}.

Em seguida, foi feita uma modificação no nome das variáveis para nomes que fossem mais intuitivos e de compreensão rápida do conteúdo. Essa modificação foi realizada utilizando o método \texttt{rename}, a partir de um dicionário que mapeava os nomes originais para os novos nomes desejados.

Com o dicionário de dados analisado, foi diagnosticado que algumas variáveis categóricas estavam codificadas com valores numéricos que não eram intuitivos. Assim, seus valores foram transformados, substituindo os códigos numéricos por descrições textuais mais compreensíveis através do metodo \texttt{map} da biblioteca \texttt{pandas}, utilizando dicionários de mapeamento construídos especificamente para cada variável categórica que necessitava de transformação.

\subsection{Tratamento de Valores Nulos}\label{resultados_tratamento_valores_nulos}

Inicialmente, foi feito o cálculo do percentual de valores nulos por variável. A Tabela \ref{tab_valores_nulos_inicial} apresenta estes valores do conjunto de dados integrados, antes dos tratamentos.

% ==================================================

% Percentual de valores ausentes por variável

%                          variavel    %_nulos
% 0                 sigla_uf_escola  78.117200
% 1            cod_municipio_escola  78.117200
% 2                   tp_adm_escola  78.117200
% 3            funcionamento_escola  78.117200
% 4                 tp_local_escola  78.117200
% 5                       tp_ensino  69.835984
% 6                       tp_escola  68.248202
% 7      ano_conclusao\_ensino\_medio  51.603048
% 8          nota_ciencias_natureza  40.353944
% 9                 nota_matematica  40.353944
% 10          nota_ciencias_humanas  36.992080
% 11                   nota_redacao  36.992080
% 12         nota_linguagem_codigos  36.992080
% 13                03\_ocupacao\_pai  12.518470
% 14            01_escolaridade_pai   9.838316
% 15                04\_ocupacao\_mae   8.991262
% 16                   estado_civil   4.244533
% 17            02_escolaridade_mae   3.516661
% 18                       cor_raca   1.842121
% 19                  10_qtde_carro   0.578713
% 20              05_qtde_moradores   0.578713
% 21              06_renda_familiar   0.578713
% 22  07_dias_trabalhador_domestico   0.578713
% 23               08\_qtde\_banheiro   0.578713
% 24                 09_qtde_quarto   0.578713
% 25           18\_flag\_aspirador\_po   0.578713
% 26            11_qtde_motocicleta   0.578713
% 27              12_qtde_geladeira   0.578713
% 28                13_qtde_freezer   0.578713
% 29        14_qtde_maq_lavar_roupa   0.578713
% 30        15_qtde_maq_secar_roupa   0.578713
% 31            16_qtde_micro_ondas   0.578713
% 32        17_qtde_maq_lavar_louca   0.578713
% 33                22_qtde_celular   0.578713
% 34                     19_qtde_tv   0.578713
% 35           20_flag_aparelho_dvd   0.578713
% 36          21_flag_tv_assinatura   0.578713
% 37           24\_qtde\_computadores   0.578713
% 38          23_flag_telefone_fixo   0.578713
% 39               25_flag_internet   0.578713
% 40                  nacionalidade   0.046861

% ==================================================

\begin{longtable}{|c|c|}

    % --- TÍTULO E RÓTULO (Aparece no topo da primeira página) ---
    \caption{Percentual de valores nulos por variável} \label{tab_valores_nulos_inicial} \\
    \hline
    \textbf{Variável} & \textbf{Percentual de nulos} \\ \hline
    \endfirsthead

    % --- CABEÇALHO PARA PÁGINAS SEGUINTES (Se quebrar página) ---
    \hline
    \textbf{Variável} & \textbf{Percentual de nulos} \\ \hline
    \endhead

    % --- RODAPÉ PARA PÁGINAS INTERMEDIÁRIAS (Antes de terminar) ---
    \hline
    \multicolumn{2}{r}{\footnotesize\textit{Continua na próxima página...}} \\
    \endfoot

    % --- RODAPÉ FINAL (Aparece apenas na última página) ---
    \hline
    \multicolumn{2}{c}{\footnotesize Fonte: elaborado pelo autor.} \\
    \endlastfoot

    % --- DADOS DA TABELA ---
    sigla\_uf\_escola & 78,1\% \\ \hline
    cod\_municipio\_escola & 78,1\% \\ \hline
    tp\_adm\_escola & 78,1\% \\ \hline
    funcionamento\_escola & 78,1\% \\ \hline
    tp\_local\_escola & 78,1\% \\ \hline
    tp\_ensino & 69,8\% \\ \hline
    tp\_escola & 68,2\% \\ \hline
    ano\_conclusao\_ensino\_medio & 51,6\% \\ \hline
    nota\_ciencias\_natureza & 40,4\% \\ \hline
    nota\_matematica & 40,4\% \\ \hline
    nota\_ciencias\_humanas & 37,0\% \\ \hline
    nota\_redacao & 37,0\% \\ \hline
    nota\_linguagem\_codigos & 37,0\% \\ \hline
    03\_ocupacao\_pai & 12,5\% \\ \hline
    01\_escolaridade\_pai & 9,8\% \\ \hline
    04\_ocupacao\_mae & 9,0\% \\ \hline
    estado\_civil & 4,2\% \\ \hline
    02\_escolaridade\_mae & 3,5\% \\ \hline
    cor\_raca & 1,8\% \\ \hline
    10\_qtde\_carro & 0,6\% \\ \hline
    05\_qtde\_moradores & 0,6\% \\ \hline
    06\_renda\_familiar & 0,6\% \\ \hline
    07\_dias\_trabalhador\_domestico & 0,6\% \\ \hline
    08\_qtde\_banheiro & 0,6\% \\ \hline
    09\_qtde\_quarto & 0,6\% \\ \hline
    18\_flag\_aspirador\_po & 0,6\% \\ \hline
    11\_qtde\_motocicleta & 0,6\% \\ \hline
    12\_qtde\_geladeira & 0,6\% \\ \hline
    13\_qtde\_freezer & 0,6\% \\ \hline
    14\_qtde\_maq\_lavar\_roupa & 0,6\% \\ \hline
    15\_qtde\_maq\_secar\_roupa & 0,6\% \\ \hline
    16\_qtde\_micro\_ondas & 0,6\% \\ \hline
    17\_qtde\_maq\_lavar\_louca & 0,6\% \\ \hline
    22\_qtde\_celular & 0,6\% \\ \hline
    19\_qtde\_tv & 0,6\% \\ \hline
    20\_flag\_aparelho\_dvd & 0,6\% \\ \hline
    21\_flag\_tv\_assinatura & 0,6\% \\ \hline
    24\_qtde\_computadores & 0,6\% \\ \hline
    23\_flag\_telefone\_fixo & 0,6\% \\ \hline
    25\_flag\_internet & 0,6\% \\ \hline
    nacionalidade & 0,05\% \\

\end{longtable}

O percentual de valores nulos varia significativamente, com algumas variáveis apresentando mais de 70\% de valores nulos, enquanto outras possuem menos de 1\%. Variáveis com uma alta proporção de valores nulos podem comprometer a análise se for realizado alguma imputação de valores. Sendo assim, foi decidido remover as variáveis que apresentavam mais de 50\% de valores nulos, resultando na remoção de nove variáveis do conjunto de dados.

Para as variáveis das notas, por serem as variáveis resposta deste trabalho, foi realizada uma análise mais detalhada. Primeiro, verificou-se a existência de valores zerados nessas variáveis e se possuem significado diferentes de valores nulos. Para isso, foi feita uma análise com a presença nas provas e o status da redação. Foi identificado que a nota zerada significa que o participante esteve presente na prova, mas obteve nota zero, enquanto o valor nulo indica que o participante ou não realizou a prova, ou foi eliminado, ou teve sua redação anulada. Dessa forma, optou-se por manter as observações com notas zeradas no conjunto de dados, removendo apenas as observações com notas nulas. A decisão de incorporar ou não as notas zero na análise será discutida na Seção \ref{resultados_modelagem}.

Para as demais variáveis com valores nulos, foi realizada uma análise consolidada, ou seja, foram retiradas as observações que possuíam valor nulo em qualquer uma das variáveis restantes, o que resultou na retirada de 4.341.559 observações do conjunto de dados.

Assim, restaram 12.241.442 observações e 45 variáveis no conjunto de dados após os tratamentos.

\subsection{Separação dos Conjuntos de Dados por Variável Resposta}\label{resultados_separacao_conjuntos_dados}

Após a separação dos conjuntos de dados por variável resposta, conforme descrito na Seção \ref{metodologia_preparacao_dados}, foram criados cinco conjuntos de dados distintos. A Tabela \ref{tab_conjunto_variaveis_resposta} apresenta a quantidade de observações e variáveis em cada conjunto de dados.

% Ciências Humanas tem 7,895,093 linhas e 35 colunas.
% Ciências Natureza tem 7,500,050 linhas e 35 colunas.
% Linguagem e Código tem 7,895,093 linhas e 35 colunas.
% Matemática tem 7,500,050 linhas e 35 colunas.
% Redação tem 7,895,093 linhas e 35 colunas.

\begin{longtable}{|l|c|c|}

    % --- TÍTULO E RÓTULO (1ª Página) ---
    \caption{Observações e variáveis por conjunto de dados} \label{tab_conjunto_variaveis_resposta} \\
    \hline
    \textbf{Conjunto de Dados} & \textbf{Observações} & \textbf{Variáveis} \\ \hline
    \endfirsthead

    % --- CABEÇALHO (Páginas Seguintes) ---
    \hline
    \textbf{Conjunto de Dados} & \textbf{Observações} & \textbf{Variáveis} \\ \hline
    \endhead

    % --- RODAPÉ (Páginas Intermediárias) ---
    \hline
    \multicolumn{3}{r}{\footnotesize\textit{Continua na próxima página...}} \\
    \endfoot

    % --- RODAPÉ FINAL (Última Página) ---
    \hline
    \multicolumn{3}{c}{\footnotesize Fonte: elaborado pelo autor.} \\
    \endlastfoot

    % --- DADOS DA TABELA ---
    Ciências Humanas & 7.895.093 & 35 \\ \hline
    Ciências da Natureza & 7.500.050 & 35 \\ \hline
    Linguagem e Código & 7.895.093 & 35 \\ \hline
    Matemática & 7.500.050 & 35 \\ \hline
    Redação & 7.895.093 & 35 \\

\end{longtable}

A estrutura de dicionários foi utilizada para manter o controle dos conjuntos de dados, suas respectivas variáveis resposta e variáveis preditoras ao longo do trabalho.

\section{Modelagem}\label{resultados_modelagem}

\subsection{Análise Exploratória dos Dados - Variáveis Resposta}\label{resultados_analise_exploratoria_dados_vars_resposta}

\subsubsection{Distribuições}\label{resultados_analise_exploratoria_dados_vars_resposta_distribuicoes}

O primeiro passo da análise exploratória foi entender o domínio das variáveis respostas e foi constatado que para as notas das provas objetivas (Ciências Humanas, Ciências da Natureza, Linguagens e Códigos e Matemática) haviam mais de cinco mil notas distintas, com variações pequenas entre elas (décimos de pontos). Já para a nota da redação, o número de notas distintas era significativamente menor (apenas 50 notas) com variação de pontos de 20 em 20 pontos.

A tabela \ref{tab_describe_notas} apresenta as estatísticas descritivas das notas de cada prova, obtido através do método \texttt{describe}.

% Ciências Humanas  Ciências Natureza  Linguagem e Código    Matemática  \
% count      7.895093e+06       7.500050e+06        7.895093e+06  7.500050e+06   
% mean       5.238126e+02       4.968878e+02        5.190258e+02  5.386558e+02   
% std        9.131224e+01       8.146093e+01        7.693933e+01  1.214854e+02   
% min        0.000000e+00       0.000000e+00        0.000000e+00  0.000000e+00   
% 25%        4.601000e+02       4.373000e+02        4.701000e+02  4.416000e+02   
% 50%        5.286000e+02       4.903000e+02        5.252000e+02  5.260000e+02   
% 75%        5.882000e+02       5.519000e+02        5.736000e+02  6.249000e+02   
% max        8.626000e+02       8.753000e+02        8.261000e+02  9.857000e+02   

%             Redação  
% count  7.895093e+06  
% mean   6.166102e+02  
% std    2.046583e+02  
% min    0.000000e+00  
% 25%    5.200000e+02  
% 50%    6.200000e+02  
% 75%    7.600000e+02  
% max    1.000000e+03

\begin{longtable}{|l|c|c|c|c|c|}

    % --- TÍTULO E RÓTULO (1ª Página) ---
    \caption{Estatísticas descritivas por conjunto de dados} \label{tab_describe_notas} \\
    \hline
    \textbf{Estatística} & \textbf{Humanas} & \textbf{Natureza} & \textbf{Linguagem} & \textbf{Matemática} & \textbf{Redação} \\ \hline
    \endfirsthead

    % --- CABEÇALHO (Páginas Seguintes) ---
    \hline
    \textbf{Estatística} & \textbf{Humanas} & \textbf{Natureza} & \textbf{Linguagem} & \textbf{Matemática} & \textbf{Redação} \\ \hline
    \endhead

    % --- RODAPÉ (Páginas Intermediárias) ---
    \hline
    \multicolumn{6}{r}{\footnotesize\textit{Continua na próxima página...}} \\
    \endfoot

    % --- RODAPÉ FINAL (Última Página) ---
    \hline
    \multicolumn{6}{c}{\footnotesize Fonte: elaborado pelo autor.} \\
    \endlastfoot

    % --- DADOS DA TABELA ---
    Contagem & 7.895.093 & 7.500.050 & 7.895.093 & 7.500.050 & 7.895.093 \\ \hline
    Média & 523,8 & 496,9 & 519,0 & 538,7 & 616,6 \\ \hline
    Desvio Padrão & 91,3 & 81,4 & 91,3 & 121,5 & 204,7 \\ \hline
    Mínimo & 0,0 & 0,0 & 0,0 & 0,0 & 0,0 \\ \hline
    25º Percentil & 460,1 & 437,3 & 460,1 & 441,6 & 520 \\ \hline
    50º Percentil & 528,6 & 490,3 & 528,6 & 526 & 620 \\ \hline
    75º Percentil & 588,2 & 551,9 & 588,2 & 624,9 & 760 \\ \hline
    Máximo & 862,6 & 875,3 & 826,1 & 985,7 & 1000 \\

\end{longtable}

Em seguida, foram construídos histogramas de cada nota para entender a distribuição das notas. As Figuras \ref{fig:hist_humanas}, \ref{fig:hist_natureza}, \ref{fig:hist_linguagem_codigo}, \ref{fig:hist_matematica} e \ref{fig:hist_redacao} apresentam os histogramas das notas de cada prova. 

\begin{figure}[H]
    \centering
    \caption{Histograma das notas - Ciências Humanas} \label{fig:hist_humanas}
    \includegraphics[width=0.7\linewidth]{imagens/histograma_humanas.png}
    \par\vspace{0.1cm}
    % {\footnotesize Fonte: elaborado pelo autor.}
\end{figure}

\begin{figure}[H]
    \centering
    \caption{Histograma das notas - Ciências da Natureza} \label{fig:hist_natureza}
    \includegraphics[width=0.7\linewidth]{imagens/histograma_natureza.png}
    \par\vspace{0.1cm}
    % {\footnotesize Fonte: elaborado pelo autor.}
\end{figure}

\begin{figure}[H]
    \centering
    \caption{Histograma das notas - Linguagem e Código} \label{fig:hist_linguagem_codigo}
    \includegraphics[width=0.7\linewidth]{imagens/histograma_linguagem_codigo.png}
    \par\vspace{0.1cm}
    % {\footnotesize Fonte: elaborado pelo autor.}
\end{figure}

\begin{figure}[H]
    \centering
    \caption{Histograma das notas - Matemática} \label{fig:hist_matematica}
    \includegraphics[width=0.7\linewidth]{imagens/histograma_matematica.png}
    \par\vspace{0.1cm}
    % {\footnotesize Fonte: elaborado pelo autor.}
\end{figure}

\begin{figure}[H]
    \centering
    \caption{Histograma das notas - Redação} \label{fig:hist_redacao}
    \includegraphics[width=0.7\linewidth]{imagens/histograma_redacao.png}
    \par\vspace{0.1cm}
    {\footnotesize Fonte: elaborado pelo autor.}
\end{figure}

A tabela \ref{tab_assimetria_curtose} apresenta os valores de assimetria e curtose das notas de cada prova, obtidos através dos métodos \texttt{skew} e \texttt{kurtosis} da biblioteca \texttt{pandas}, e o percentual de notas zero em cada conjunto de dados.

% Ciências Humanas - Assimetria: -0.3408 | Curtose: 1.1269
% Ciências Natureza - Assimetria: 0.0321 | Curtose: 1.8372
% Linguagem e Código - Assimetria: -0.5113 | Curtose: 1.1780
% Matemática - Assimetria: 0.3138 | Curtose: 0.0850
% Redação - Assimetria: -0.7457 | Curtose: 1.0488

\begin{longtable}{|l|c|c|c|}

    % --- TÍTULO E RÓTULO (1ª Página) ---
    \caption{Assimetria, Curtose e Notas zeradas} \label{tab_assimetria_curtose} \\
    \hline
    \textbf{Variável} & \textbf{Assimetria} & \textbf{Curtose} & \textbf{Notas zeradas} \\ \hline
    \endfirsthead

    % --- CABEÇALHO (Páginas Seguintes) ---
    \hline
    \textbf{Variável} & \textbf{Assimetria} & \textbf{Curtose} & \textbf{Notas zeradas} \\ \hline
    \endhead

    % --- RODAPÉ (Páginas Intermediárias) ---
    \hline
    \multicolumn{4}{r}{\footnotesize\textit{Continua na próxima página...}} \\
    \endfoot

    % --- RODAPÉ FINAL (Última Página) ---
    \hline
    \multicolumn{4}{c}{\footnotesize Fonte: elaborado pelo autor.} \\
    \endlastfoot

    % --- DADOS DA TABELA ---
    Humanas & -0,3408 & 1,1269 & 0,18\%\\ \hline
    Natureza & 0,0321 & 1,8372 & 0,17\%\\ \hline
    Linguagem & -0,5113 & 1,1780 & 0,08\%\\ \hline
    Matemática & 0,3138 & 0,0850 & 0,17\%\\ \hline
    Redação & -0,7457 & 1,0488 & 3,56\%\\ \hline
\end{longtable}

Analisando os valores de assimetria e curtose, é possível observar que as distribuições das notas possuem diferentes características.

A assimetria da nota de redação é a mais negativa, o que indica que os alunos tivereram, em geral, o melhor desempenho, assim como nas provas de Linguagens e Códigos e Ciências Humanas, que também apresentam assimetria negativa, porém com valores menores. Na prova de Ciências da Natureza, a assimetria é praticamente nula, indicando uma distribuição mais simétrica das notas, enquanto a nota de matemática apresenta a assimetria mais positiva, indicando um desempenho relativamente pior dos alunos nessa prova.

Analisando os valores da curtose, a prova de matemática foi a única a apresentar uma curtose próxima de zero, indicando uma distribuição mais próxima da normalidade. As outras provas apresentaram valores maiores que 1 indicando distribuições com caudas mais pesadas e picos mais acentuados.

\subsubsection{Teste de Hipótese}\label{resultados_analise_exploratoria_dados_vars_resposta_teste_hipotese}

Foi realizado o teste de hipótese ANOVA com nível de significância de $0,1\%$ para comparar as médias das notas por edição do ENEM em cada conjunto de dados, onde a hipótese nula $H_0$ é de que as médias são iguais entre as edições. A tabela \ref{tab_anova_notas_edicao} apresenta os valores de F, p-valor e a métrica SMD (\textit{Standardized Mean Difference}) obtidos para cada conjunto de dados.

% Ciências Humanas | Estatística F: 13161.5488 | Rejeita-se H0: Sim | p-valor: 0.0000e+00 | SMD: 0.1850 | Insignificante
% Ciências Natureza | Estatística F: 3431.3132 | Rejeita-se H0: Sim | p-valor: 0.0000e+00 | SMD: 0.0866 | Insignificante
% Linguagem e Código | Estatística F: 26507.5898 | Rejeita-se H0: Sim | p-valor: 0.0000e+00 | SMD: 0.2686 | Pequeno
% Matemática | Estatística F: 13041.6191 | Rejeita-se H0: Sim | p-valor: 0.0000e+00 | SMD: 0.1968 | Insignificante
% Redação | Estatística F: 27498.2207 | Rejeita-se H0: Sim | p-valor: 0.0000e+00 | SMD: 0.2422 | Pequeno

\begin{longtable}{|l|c|c|c|c|c|}

    % --- TÍTULO E RÓTULO (1ª Página) ---
    \caption{Teste ANOVA das médias das notas por edição} \label{tab_anova_notas_edicao} \\
    \hline
    \textbf{Variável} & \textbf{Valor F} & \textbf{Rejeita-se $H_0$?} & \textbf{p-valor} & \textbf{SMD} & \textbf{Tamanho do efeito} \\ \hline
    \endfirsthead

    % --- CABEÇALHO (Páginas Seguintes) ---
    \hline
    \textbf{Variável} & \textbf{Valor F} & \textbf{Rejeita-se $H_0$?} & \textbf{p-valor} & \textbf{SMD} & \textbf{Tamanho do efeito} \\ \hline
    \endhead

    % --- RODAPÉ (Páginas Intermediárias) ---
    \hline
    \multicolumn{6}{r}{\footnotesize\textit{Continua na próxima página...}} \\
    \endfoot

    % --- RODAPÉ FINAL (Última Página) ---
    \hline
    \multicolumn{6}{c}{\footnotesize Fonte: elaborado pelo autor.} \\
    \endlastfoot

    % --- DADOS DA TABELA ---
    Humanas & 13.161 & Sim & 0,0000 & 0,185 & Insignificante \\ \hline
    Natureza & 3.431 & Sim & 0,0000 & 0,087 & Insignificante \\ \hline
    Linguagem & 26.506 & Sim & 0,0000 & 0,269 & Pequeno \\ \hline
    Matemática & 13.041 & Sim & 0,0000 & 0,197 & Insignificante \\ \hline
    Redação & 27.498 & Sim & 0,0000 & 0,242 & Pequeno \\ \hline

\end{longtable}

Foi decidido manter todas as edições do ENEM no conjunto de dados para a modelagem preditiva e sem a necessidade de segmentação por edição, uma vez que o tamanho do efeito é insignificante ou pequeno.

\subsubsection{Análise de Outliers}\label{resultados_analise_exploratoria_dados_vars_resposta_outliers}

Para a análise dos outliers, foram utilizados os boxplots das notas de cada prova, apresentados nas Figuras \ref{fig:boxplot_humanas}, \ref{fig:boxplot_natureza}, \ref{fig:boxplot_linguagem_codigo}, \ref{fig:boxplot_matematica} e \ref{fig:boxplot_redacao}.

\begin{figure}[H]
    \centering
    \caption{Boxplot das notas por edição - Ciências Humanas} \label{fig:boxplot_humanas}
    \includegraphics[width=0.7\linewidth]{imagens/boxplot_humanas.png}
    \par\vspace{0.1cm}
    % {\footnotesize Fonte: elaborado pelo autor.}
\end{figure}

\begin{figure}[H]
    \centering
    \caption{Boxplot das notas - Ciências da Natureza} \label{fig:boxplot_natureza}
    \includegraphics[width=0.7\linewidth]{imagens/boxplot_natureza.png}
    \par\vspace{0.1cm}
    % {\footnotesize Fonte: elaborado pelo autor.}
\end{figure}

\begin{figure}[H]
    \centering
    \caption{Boxplot das notas - Linguagem e Código} \label{fig:boxplot_linguagem_codigo}
    \includegraphics[width=0.7\linewidth]{imagens/boxplot_linguagem_codigo.png}
    \par\vspace{0.1cm}
    % {\footnotesize Fonte: elaborado pelo autor.}
\end{figure}

\begin{figure}[H]
    \centering
    \caption{Boxplot das notas - Matemática} \label{fig:boxplot_matematica}
    \includegraphics[width=0.7\linewidth]{imagens/boxplot_matematica.png}
    \par\vspace{0.1cm}
    % {\footnotesize Fonte: elaborado pelo autor.}
\end{figure}

\begin{figure}[H]
    \centering
    \caption{Boxplot das notas - Redação} \label{fig:boxplot_redacao}
    \includegraphics[width=0.7\linewidth]{imagens/boxplot_redacao.png}
    \par\vspace{0.1cm}
    {\footnotesize Fonte: elaborado pelo autor.}
\end{figure}

Foi utilizado o critério do intervalor interquartil (\textit{Interquartile Range} - IQR) para identificar os outliers nas notas de cada prova. Foram considerados outliers os valores que estavam abaixo de $Q1 - 1,5 \times IQR$ ou acima de $Q3 + 1,5 \times IQR$, onde Q1 é o primeiro quartil, Q3 é o terceiro quartil e IQR = Q3 - Q1. A Tabela \ref{tab_outliers_notas} apresenta o limite inferior, o limite superior, a quantidade e o percentual de outliers identificados em cada conjunto de dados.

% ==================================================

% Outliers - Ciências Humanas

% Limite Superior: 780.35
% Limite Inferior: 267.95

% Superiores: 5,627 (0.07%)
% Inferiores: 13,969 (0.18%)
% Total: 19,596 (0.25%)

% ==================================================

% Outliers - Ciências Natureza

% Limite Superior: 723.80
% Limite Inferior: 265.40

% Superiores: 36,402 (0.49%)
% Inferiores: 12,615 (0.17%)
% Total: 49,017 (0.65%)

% ==================================================

% Outliers - Linguagem e Código

% Limite Superior: 728.85
% Limite Inferior: 314.85

% Superiores: 3,195 (0.04%)
% Inferiores: 32,803 (0.42%)
% Total: 35,998 (0.46%)

% ==================================================

% Outliers - Matemática

% Limite Superior: 899.85
% Limite Inferior: 166.65

% Superiores: 15,096 (0.20%)
% Inferiores: 12,848 (0.17%)
% Total: 27,944 (0.37%)

% ==================================================

% Outliers - Redação

% Limite Superior: 1000.00
% Limite Inferior: 160.00

% Superiores: 0 (0.00%)
% Inferiores: 282,438 (3.58%)
% Total: 282,438 (3.58%)

% ==================================================

\begin{longtable}{|l|c|c|c|}

    % --- TÍTULO E RÓTULO (1ª Página) ---
    \caption{Quantidade e percentual de outliers nas notas} \label{tab_outliers_notas} \\
    \hline
    \textbf{Variável} & \textbf{Limite Inferior} & \textbf{Limite Superior} & \textbf{Outliers} \\ \hline
    \endfirsthead

    % --- CABEÇALHO (Páginas Seguintes) ---
    \hline
    \textbf{Variável} & \textbf{Limite Inferior} & \textbf{Limite Superior} & \textbf{Outliers} \\ \hline
    \endhead

    % --- RODAPÉ (Páginas Intermediárias) ---
    \hline
    \multicolumn{4}{r}{\footnotesize\textit{Continua na próxima página...}} \\
    \endfoot

    % --- RODAPÉ FINAL (Última Página) ---
    \hline
    \multicolumn{4}{c}{\footnotesize Fonte: elaborado pelo autor.} \\
    \endlastfoot

    % --- DADOS DA TABELA ---
    Humanas & 267,95 & 780,35 & 19.596 (0,25\%) \\ \hline
    Natureza & 265,40 & 723,80 & 49.017 (0,65\%) \\ \hline
    Linguagem & 314,85 & 728,85 & 35.998 (0,46\%) \\ \hline
    Matemática & 166,65 & 899,85 & 27.944 (0,37\%) \\ \hline
    Redação & 160,00 & 1000,00 & 282.438 (3,58\%) \\ \hline

\end{longtable}

Ao analisarmos os intervalos dos outliers da nota da Redação, vemos que a nota máxima (1.000 pontos) não foi considerada outlier. Porém, tendo o contexto do ENEM em mente e analisando a distribuição das notas, é razoável retirarmos as notas máximas do conjunto de dados, uma vez que são notas extremamente raras. Isso resulta em mais 116 observações removidas do conjunto de dados da nota de Redação.

\subsection{Análise Exploratória - Variáveis Preditoras}\label{resultados_analise_exploratoria_dados_vars_preditoras}

A análise exploratória das variáveis preditoras seguiu três etapas: (i) concentração, (ii) correlação e (iii) \textit{Permutration Importance}, conforme descrito na Seção \ref{metodologia_exploracao_dados}.

\subsubsection{Concentração}\label{resultados_analise_exploratoria_dados_vars_preditoras_concentracao}

Ao calcularmos a proporção de observações para cada categoria das variáveis categóricas, foi possível identifcar três variáveis que apresentavam uma concentração acima de 93\%. Devido a essa alta concentração, foi decidido remover essas variáveis do conjunto de dados. As tabelas \ref{tab_concentracao_vars_categoricas_humanas} a \ref{tab_concentracao_vars_categoricas_redacao} apresentam as cinco variáveis categóricas com maior concentração e suas respectivas proporções da categoria de maior Concentração para cada conjunto de dados.

% ===================== Ciências Humanas =====================

%                             variavel  concentracao
% 0              12\_qtde\_geladeira      0.928419
% 1  07\_dias\_trabalhador\_domestico      0.902118
% 2               25\_flag\_internet      0.898890
% 3        15\_qtde\_maq\_secar\_roupa      0.865083
% 4          23\_flag\_telefone\_fixo      0.841536

\begin{longtable}{|l|c|}

    % --- TÍTULO E RÓTULO (1ª Página) ---
    \caption{Cinco maiores concentrações - Humanas} \label{tab_concentracao_vars_categoricas_humanas} \\
    \hline
    \textbf{Variável} & \textbf{Maior Concentração} \\ \hline
    \endfirsthead

    % --- CABEÇALHO (Páginas Seguintes) ---
    \hline
    \textbf{Variável} & \textbf{Maior Concentração} \\ \hline
    \endhead

    % --- RODAPÉ (Páginas Intermediárias) ---
    \hline
    \multicolumn{2}{r}{\footnotesize\textit{Continua na próxima página...}} \\
    \endfoot

    % --- RODAPÉ FINAL (Última Página) ---
    \hline
    \multicolumn{2}{c}{\footnotesize Fonte: elaborado pelo autor.} \\
    \endlastfoot

    % --- DADOS DA TABELA ---
    12\_qtde\_geladeira & 92.8\% \\ \hline
    07\_dias\_trabalhador\_domestico & 90.2\% \\ \hline
    25\_flag\_internet & 89.9\% \\ \hline
    15\_qtde\_maq\_secar\_roupa & 86.5\% \\ \hline
    23\_flag\_telefone\_fixo & 84.1\% \\ \hline

\end{longtable}

% ==================== Ciências Natureza =====================

%                             variavel  concentracao
% 0              12\_qtde\_geladeira      0.928660
% 1  07\_dias\_trabalhador\_domestico      0.902912
% 2               25\_flag\_internet      0.899944
% 3        15\_qtde\_maq\_secar\_roupa      0.865654
% 4          23\_flag\_telefone\_fixo      0.840437

\begin{longtable}{|l|c|}

    % --- TÍTULO E RÓTULO (1ª Página) ---
    \caption{Cinco maiores concentrações - Natureza} \label{tab_concentracao_vars_categoricas_natureza} \\
    \hline
    \textbf{Variável} & \textbf{Maior Concentração} \\ \hline
    \endfirsthead

    % --- CABEÇALHO (Páginas Seguintes) ---
    \hline
    \textbf{Variável} & \textbf{Maior Concentração} \\ \hline
    \endhead

    % --- RODAPÉ (Páginas Intermediárias) ---
    \hline
    \multicolumn{2}{r}{\footnotesize\textit{Continua na próxima página...}} \\
    \endfoot

    % --- RODAPÉ FINAL (Última Página) ---
    \hline
    \multicolumn{2}{c}{\footnotesize Fonte: elaborado pelo autor.} \\
    \endlastfoot

    % --- DADOS DA TABELA ---
    12\_qtde\_geladeira & 92.9\% \\ \hline
    07\_dias\_trabalhador\_domestico & 90.3\% \\ \hline
    25\_flag\_internet & 89.9\% \\ \hline
    15\_qtde\_maq\_secar\_roupa & 86.6\% \\ \hline
    23\_flag\_telefone\_fixo & 84.0\% \\ \hline

\end{longtable}

% ==================== Linguagem e Código ====================

%                             variavel  concentracao
% 0              12\_qtde\_geladeira      0.928376
% 1  07\_dias\_trabalhador\_domestico      0.901928
% 2               25\_flag\_internet      0.899254
% 3        15\_qtde\_maq\_secar\_roupa      0.864880
% 4          23\_flag\_telefone\_fixo      0.841141

\begin{longtable}{|l|c|}

    % --- TÍTULO E RÓTULO (1ª Página) ---
    \caption{Cinco maiores concentrações - Linguagem} \label{tab_concentracao_vars_categoricas_linguagem} \\
    \hline
    \textbf{Variável} & \textbf{Maior Concentração} \\ \hline
    \endfirsthead

    % --- CABEÇALHO (Páginas Seguintes) ---
    \hline
    \textbf{Variável} & \textbf{Maior Concentração} \\ \hline
    \endhead

    % --- RODAPÉ (Páginas Intermediárias) ---
    \hline
    \multicolumn{2}{r}{\footnotesize\textit{Continua na próxima página...}} \\
    \endfoot

    % --- RODAPÉ FINAL (Última Página) ---
    \hline
    \multicolumn{2}{c}{\footnotesize Fonte: elaborado pelo autor.} \\
    \endlastfoot

    % --- DADOS DA TABELA ---
    12\_qtde\_geladeira & 92.9\% \\ \hline
    07\_dias\_trabalhador\_domestico & 90.2\% \\ \hline
    25\_flag\_internet & 89.9\% \\ \hline
    15\_qtde\_maq\_secar\_roupa & 86.5\% \\ \hline
    23\_flag\_telefone\_fixo & 84.1\% \\ \hline

\end{longtable}

% ======================== Matemática ========================

%                             variavel  concentracao
% 0              12\_qtde\_geladeira      0.928426
% 1  07\_dias\_trabalhador\_domestico      0.902230
% 2               25\_flag\_internet      0.900222
% 3        15\_qtde\_maq\_secar\_roupa      0.865169
% 4          23\_flag\_telefone\_fixo      0.839908

\begin{longtable}{|l|c|}

    % --- TÍTULO E RÓTULO (1ª Página) ---
    \caption{Cinco maiores concentrações - Matemática} \label{tab_concentracao_vars_categoricas_matematica} \\
    \hline
    \textbf{Variável} & \textbf{Maior Concentração} \\ \hline
    \endfirsthead

    % --- CABEÇALHO (Páginas Seguintes) ---
    \hline
    \textbf{Variável} & \textbf{Maior Concentração} \\ \hline
    \endhead

    % --- RODAPÉ (Páginas Intermediárias) ---
    \hline
    \multicolumn{2}{r}{\footnotesize\textit{Continua na próxima página...}} \\
    \endfoot

    % --- RODAPÉ FINAL (Última Página) ---
    \hline
    \multicolumn{2}{c}{\footnotesize Fonte: elaborado pelo autor.} \\
    \endlastfoot

    % --- DADOS DA TABELA ---
    12\_qtde\_geladeira & 92.8\% \\ \hline
    07\_dias\_trabalhador\_domestico & 90.2\% \\ \hline
    25\_flag\_internet & 90.0\% \\ \hline
    15\_qtde\_maq\_secar\_roupa & 86.5\% \\ \hline
    23\_flag\_telefone\_fixo & 84.0\% \\ \hline

\end{longtable}

% ========================= Redação ==========================

%                             variavel  concentracao
% 0              12\_qtde\_geladeira      0.927998
% 1               25\_flag\_internet      0.901566
% 2  07\_dias\_trabalhador\_domestico      0.900670
% 3        15\_qtde\_maq\_secar\_roupa      0.863696
% 4          23\_flag\_telefone\_fixo      0.839008

\begin{longtable}{|l|c|}

    % --- TÍTULO E RÓTULO (1ª Página) ---
    \caption{Cinco maiores concentrações - Redação} \label{tab_concentracao_vars_categoricas_redacao} \\
    \hline
    \textbf{Variável} & \textbf{Maior Concentração} \\ \hline
    \endfirsthead

    % --- CABEÇALHO (Páginas Seguintes) ---
    \hline
    \textbf{Variável} & \textbf{Maior Concentração} \\ \hline
    \endhead

    % --- RODAPÉ (Páginas Intermediárias) ---
    \hline
    \multicolumn{2}{r}{\footnotesize\textit{Continua na próxima página...}} \\
    \endfoot

    % --- RODAPÉ FINAL (Última Página) ---
    \hline
    \multicolumn{2}{c}{\footnotesize Fonte: elaborado pelo autor.} \\
    \endlastfoot

    % --- DADOS DA TABELA ---
    12\_qtde\_geladeira & 92.8\% \\ \hline
    07\_dias\_trabalhador\_domestico & 90.1\% \\ \hline
    25\_flag\_internet & 90.1\% \\ \hline
    15\_qtde\_maq\_secar\_roupa & 86.4\% \\ \hline
    23\_flag\_telefone\_fixo & 83.9\% \\ \hline

\end{longtable}

\subsubsection{Correlação Phik}\label{resultados_analise_exploratoria_dados_vars_preditoras_correlacao_phik}

A próxima etapa da análise exploratória das variáveis preditoras foi a análise de correlação utilizando a métrica Phik. As tabelas \ref{tab_correlacao_phik_humanas} a \ref{tab_correlacao_phik_redacao} apresentam as cinco variáveis com maior correlação Phik com a variável resposta em cada conjunto de dados.

% ===================== Ciências Humanas =====================

%                         var_1      phik
% 898  24\_qtde\_computadores  0.445248
% 278       03\_ocupacao\_pai  0.396672
% 309       04\_ocupacao\_mae  0.375381
% 433      08\_qtde\_banheiro  0.356501
% 712  18\_flag\_aspirador\_po  0.354332


\begin{longtable}{|l|c|}

    % --- TÍTULO E RÓTULO (1ª Página) ---
    \caption{Cinco maiores correlações Phik - Humanas} \label{tab_correlacao_phik_humanas} \\

    \hline
    \textbf{Variável} & \textbf{Correlação Phik} \\ \hline
    \endfirsthead

    % --- CABEÇALHO (Páginas Seguintes) ---
    \hline
    \textbf{Variável} & \textbf{Correlação Phik} \\ \hline
    \endhead

    % --- RODAPÉ (Páginas Intermediárias) ---
    \hline
    \multicolumn{2}{r}{\footnotesize\textit{Continua na próxima página...}} \\
    \endfoot

    % --- RODAPÉ FINAL (Última Página) ---
    \hline
    \multicolumn{2}{c}{\footnotesize Fonte: elaborado pelo autor.} \\
    \endlastfoot

    % --- DADOS DA TABELA ---
    24\_qtde\_computadores & 44,5\% \\ \hline
    03\_ocupacao\_pai & 39,7\% \\ \hline
    04\_ocupacao\_mae & 37,6\% \\ \hline
    08\_qtde\_banheiro & 35,7\% \\ \hline
    18\_flag\_aspirador\_po & 35,4\% \\ \hline

\end{longtable}

% ==================== Ciências Natureza =====================

%                         var_1      phik
% 898  24\_qtde\_computadores  0.445987
% 278       03\_ocupacao\_pai  0.401680
% 309       04\_ocupacao\_mae  0.379906
% 433      08\_qtde\_banheiro  0.374039
% 712  18\_flag\_aspirador\_po  0.364154

% \vspace{1cm} % para quebrar a página
\pagebreak

\begin{longtable}{|l|c|}

    % --- TÍTULO E RÓTULO (1ª Página) ---
    \caption{Cinco maiores correlações Phik - Natureza} \label{tab_correlacao_phik_natureza} \\
    \hline
    \textbf{Variável} & \textbf{Correlação Phik} \\ \hline
    \endfirsthead

    % --- CABEÇALHO (Páginas Seguintes) ---
    \hline
    \textbf{Variável} & \textbf{Correlação Phik} \\ \hline
    \endhead

    % --- RODAPÉ (Páginas Intermediárias) ---
    \hline
    \multicolumn{2}{r}{\footnotesize\textit{Continua na próxima página...}} \\
    \endfoot

    % --- RODAPÉ FINAL (Última Página) ---
    \hline
    \multicolumn{2}{c}{\footnotesize Fonte: elaborado pelo autor.} \\
    \endlastfoot

    % --- DADOS DA TABELA ---
    24\_qtde\_computadores & 44,6\% \\ \hline
    03\_ocupacao\_pai & 40,2\% \\ \hline
    04\_ocupacao\_mae & 38,0\% \\ \hline
    08\_qtde\_banheiro & 37,4\% \\ \hline
    18\_flag\_aspirador\_po & 36,4\% \\ \hline

\end{longtable}

% ==================== Linguagem e Código ====================

%                         var_1      phik
% 898  24\_qtde\_computadores  0.440769
% 278       03\_ocupacao\_pai  0.417061
% 309       04\_ocupacao\_mae  0.398863
% 185    lingua_estrangeira  0.367523
% 712  18\_flag\_aspirador\_po  0.361188

\begin{longtable}{|l|c|}

    % --- TÍTULO E RÓTULO (1ª Página) ---
    \caption{Cinco maiores correlações Phik - Linguagem} \label{tab_correlacao_phik_linguagem} \\
    \hline
    \textbf{Variável} & \textbf{Correlação Phik} \\ \hline
    \endfirsthead

    % --- CABEÇALHO (Páginas Seguintes) ---
    \hline
    \textbf{Variável} & \textbf{Correlação Phik} \\ \hline
    \endhead

    % --- RODAPÉ (Páginas Intermediárias) ---
    \hline
    \multicolumn{2}{r}{\footnotesize\textit{Continua na próxima página...}} \\
    \endfoot

    % --- RODAPÉ FINAL (Última Página) ---
    \hline
    \multicolumn{2}{c}{\footnotesize Fonte: elaborado pelo autor.} \\
    \endlastfoot

    % --- DADOS DA TABELA ---
    24\_qtde\_computadores & 44,1\% \\ \hline
    03\_ocupacao\_pai & 41,7\% \\ \hline
    04\_ocupacao\_mae & 39,9\% \\ \hline
    08\_qtde\_banheiro & 36,8\% \\ \hline
    18\_flag\_aspirador\_po & 36,1\% \\ \hline

\end{longtable}

% ======================== Matemática ========================

%                         var_1      phik
% 898  24\_qtde\_computadores  0.478993
% 278       03\_ocupacao\_pai  0.447099
% 309       04\_ocupacao\_mae  0.425178
% 433      08\_qtde\_banheiro  0.419609
% 712  18\_flag\_aspirador\_po  0.404733

\begin{longtable}{|l|c|}

    % --- TÍTULO E RÓTULO (1ª Página) ---
    \caption{Cinco maiores correlações Phik - Matemática} \label{tab_correlacao_phik_matematica} \\
    \hline
    \textbf{Variável} & \textbf{Correlação Phik} \\ \hline
    \endfirsthead

    % --- CABEÇALHO (Páginas Seguintes) ---
    \hline
    \textbf{Variável} & \textbf{Correlação Phik} \\ \hline
    \endhead

    % --- RODAPÉ (Páginas Intermediárias) ---
    \hline
    \multicolumn{2}{r}{\footnotesize\textit{Continua na próxima página...}} \\
    \endfoot

    % --- RODAPÉ FINAL (Última Página) ---
    \hline
    \multicolumn{2}{c}{\footnotesize Fonte: elaborado pelo autor.} \\
    \endlastfoot

    % --- DADOS DA TABELA ---
    24\_qtde\_computadores & 47,9\% \\ \hline
    03\_ocupacao\_pai & 44,7\% \\ \hline
    04\_ocupacao\_mae & 42,5\% \\ \hline
    08\_qtde\_banheiro & 41,9\% \\ \hline
    18\_flag\_aspirador\_po & 40,5\% \\ \hline

\end{longtable}

% ========================= Redação ==========================

%                         var_1      phik
% 278       03\_ocupacao\_pai  0.359434
% 898  24\_qtde\_computadores  0.355526
% 309       04\_ocupacao\_mae  0.348788
% 433      08\_qtde\_banheiro  0.332394
% 495         10_qtde_carro  0.293415

\begin{longtable}{|l|c|}

    % --- TÍTULO E RÓTULO (1ª Página) ---
    \caption{Cinco maiores correlações Phik - Redação} \label{tab_correlacao_phik_redacao} \\
    \hline
    \textbf{Variável} & \textbf{Correlação Phik} \\ \hline
    \endfirsthead

    % --- CABEÇALHO (Páginas Seguintes) ---
    \hline
    \textbf{Variável} & \textbf{Correlação Phik} \\ \hline
    \endhead

    % --- RODAPÉ (Páginas Intermediárias) ---
    \hline
    \multicolumn{2}{r}{\footnotesize\textit{Continua na próxima página...}} \\
    \endfoot

    % --- RODAPÉ FINAL (Última Página) ---
    \hline
    \multicolumn{2}{c}{\footnotesize Fonte: elaborado pelo autor.} \\
    \endlastfoot

    % --- DADOS DA TABELA ---a
    03\_ocupacao\_pai & 35,9\% \\ \hline
    24\_qtde\_computadores & 35,6\% \\ \hline
    04\_ocupacao\_mae & 34,9\% \\ \hline
    08\_qtde\_banheiro & 33,2\% \\ \hline
    10\_qtde\_carro & 29,3\% \\ \hline

\end{longtable}

Ao analisarmos as matrizes de correlação Phik completas para cada conjunto de dados, foi possível identificar um par de variáveis que apresentavam uma correlação perfeita (Phik = 1.0): flag de treineiro e status da conclusão do ensino médio.

Devido a essa correlação perfeita, analisamos a distribuição cruzadas das categorias dessas duas variáveis, onde foi possível observar que 100\% das observações da categoria "Treineiro" da variável "flag\_treineiro" estavam associadas à categoria "Termina o ensino médio após o ano da prova" da variável "conclusao\_ensino\_medio". Diante isso, dado que a variável "conclusao\_ensino\_medio" apresenta mais categorias e, portanto, mais informações, foi decidido manter essa variável no conjunto de dados e remover a variável "flag\_treineiro".

\subsubsection{\textit{Permutation Importance}}\label{resultados_analise_exploratoria_dados_vars_preditoras_permutation_importance}

Conforme descrito na Seção \ref{metodologia_exploracao_dados}, a última etapa da análise exploratória das variáveis preditoras foi a análise de importância utilizando a métrica \textit{Permutation Importance}. Realizadas as separações dos conjuntos de dados em treino e teste, foi treinado um modelo de \textit{Random Forest Regressor} em cada conjunto de dados para em seguida calcularmos o \textit{Permutation Importance} de cada variável preditora.

As Figuras \ref{fig_permutation_importance_humanas} a \ref{fig_permutation_importance_redacao} apresentam os gráficos de importância das dez variáveis mais importantes para cada conjunto de dados.

\begin{figure}[H]
    \centering
    \caption{Dez maiores \textit{Permutation Importance} - Humanas} \label{fig_permutation_importance_humanas}
    \includegraphics[width=0.7\linewidth]{imagens/permutation_importance_humanas.png}
    \par\vspace{0.1cm}
    % {\footnotesize Fonte: elaborado pelo autor.}
\end{figure}

\begin{figure}[H]
    \centering
    \caption{Dez maiores \textit{Permutation Importance} - Natureza} \label{fig_permutation_importance_natureza}
    \includegraphics[width=0.7\linewidth]{imagens/permutation_importance_natureza.png}
    \par\vspace{0.1cm}
    % {\footnotesize Fonte: elaborado pelo autor.}
\end{figure}

\begin{figure}[H]
    \centering
    \caption{Dez maiores \textit{Permutation Importance} - Linguagem} \label{fig_permutation_importance_linguagem_codigo}
    \includegraphics[width=0.7\linewidth]{imagens/permutation_importance_linguagem_codigo.png}
    \par\vspace{0.1cm}
    % {\footnotesize Fonte: elaborado pelo autor.}
\end{figure}

\begin{figure}[H]
    \centering
    \caption{Dez maiores \textit{Permutation Importance} - Matemática} \label{fig_permutation_importance_matematica}
    \includegraphics[width=0.7\linewidth]{imagens/permutation_importance_matematica.png}
    \par\vspace{0.1cm}
    % {\footnotesize Fonte: elaborado pelo autor.}
\end{figure}

\begin{figure}[H]
    \centering
    \caption{Dez maiores \textit{Permutation Importance} - Redação} \label{fig_permutation_importance_redacao}
    \includegraphics[width=0.7\linewidth]{imagens/permutation_importance_redacao.png}
    \par\vspace{0.1cm}
    {\footnotesize Fonte: elaborado pelo autor.}
\end{figure}

\subsubsection{Seleção de Variáveis}\label{resultados_analise_exploratoria_dados_vars_preditoras_selecao_variaveis}

Até o momento já aplicamos dois critérios para a seleção das variáveis preditoras: (i) concentração, removendo variáveis com concentração de categoria maior que 93\% e (ii) correlação perfeita com outra variável preditora, resultando em quatro variáveis removidas do conjunto de dados: (i) \texttt{nacionalidade}, \texttt{17\_qtde\_maq\_lavar\_louca} e \texttt{estado\_civil} e (ii) \texttt{flag\_treineiro}.



Com as informações de correlação Phik e \textit{Permutation Importance}, aplicaremos mais dois critérios para a seleção das variáveis preditoras: (i) correlação baixa com a variável resposta (Phik < 0,05\%) e (ii) um critério duplo de correlação alta (Phik > 85\%) com outras variáveis preditoras e menor \textit{Permutation Importance} entre as variáveis correlacionadas.

No critério de baixa correlação com a variável resposta, nenhuma variável preditora foi removida, uma vez que todas apresentaram Phik maior que 0,05\%, sendo o menor valor, entre todos os conjuntos de dados, de 3.64\% (variável de quantidade de motociclestas com a nota da Redação).

No critério duplo, foi apresentado apenas uma dupla de variáveis com correlação alta (Phik > 85\%) em cada conjunto de dados: status da conclusão do ensino médio e a faixa etária do participante. Apenas no conjunto de dados da Redação, a variável de faixa etária apresentou maior \textit{Permutation Importance} e então seria mantida no conjunto de dados, enquanto a variável de status da conclusão do ensino médio seria removida.

Porém, considerando a concentração cruzada das duas variáveis (apresentada na Tabela \ref{tab_concentracao_cruzada_status_conclusao_ensino_medio_faixa_etaria} para o conjunto de dados da Redação), foi decido manter ambas as variáveis no conjunto de dados por escolha do autor.

% fx_etaria | ja_terminou_ensino | nao_concluiu_nem_cursa | termina_apos_ano_da_prova | termina_no_ano_da_prova
% 17 & 0.3\% & 15.9\% & 6.6\% & 0.0\%  \\ \hline
% 18 & 6.3\% & 16.0\% & 0.6\% & 0.0\%  \\ \hline
% 19 & 8.8\% & 2.4\% & 0.1\% & 0.0\%  \\ \hline
% 20 & 6.2\% & 0.6\% & 0.0\% & 0.0\%  \\ \hline
% 21 & 4.3\% & 0.2\% & 0.0\% & 0.0\%  \\ \hline
% 22 & 3.1\% & 0.1\% & 0.0\% & 0.0\%  \\ \hline
% 23 & 2.4\% & 0.0\% & 0.0\% & 0.0\%  \\ \hline
% 24 & 1.9\% & 0.0\% & 0.0\% & 0.0\%  \\ \hline
% 25 & 1.5\% & 0.0\% & 0.0\% & 0.0\%  \\ \hline
% 26_30 & 4.5\% & 0.1\% & 0.0\% & 0.0\%  \\ \hline
% 31_35 & 2.4\% & 0.0\% & 0.0\% & 0.0\%  \\ \hline
% 36_40 & 1.7\% & 0.0\% & 0.0\% & 0.0\%  \\ \hline
% 41_45 & 1.2\% & 0.0\% & 0.0\% & 0.0\%  \\ \hline
% 46_50 & 0.7\% & 0.0\% & 0.0\% & 0.0\%  \\ \hline
% 51_55 & 0.4\% & 0.0\% & 0.0\% & 0.0\%  \\ \hline
% 56_60 & 0.2\% & 0.0\% & 0.0\% & 0.0\%  \\ \hline
% 61_65 & 0.1\% & 0.0\% & 0.0\% & 0.0\%  \\ \hline
% 66_70 & 0.0\% & 0.0\% & 0.0\% & 0.0\%  \\ \hline
% <_17 & 0.0\% & 0.4\% & 10.1\% & 0.2\%  \\ \hline
% >_70 & 0.0\% & 0.0\% & 0.0\% & 0.0\%  \\ \hline

\begin{longtable}{|c|c|c|c|c|}

    % --- TÍTULO E RÓTULO (1ª Página) ---
    \caption{Concentração cruzada - Redação} \label{tab_concentracao_cruzada_status_conclusao_ensino_medio_faixa_etaria} \\
    \hline
    \textbf{Faixa etária} & \textbf{Código 1} & \textbf{Código 2} & \textbf{Código 3} & \textbf{Código 4}\\ \hline
    \endfirsthead

    % --- CABEÇALHO (Páginas Seguintes) ---
    \hline
    \textbf{Faixa etária} & \textbf{Código 1} & \textbf{Código 2} & \textbf{Código 3} & \textbf{Código 4}\\ \hline
    \endhead

    % --- RODAPÉ (Páginas Intermediárias) ---
    \hline
    \multicolumn{5}{r}{\footnotesize\textit{Continua na próxima página...}} \\
    \endfoot

    % --- RODAPÉ FINAL (Última Página) ---
    \hline
    \multicolumn{5}{c}{\footnotesize Fonte: elaborado pelo autor.} \\
    \endlastfoot

    % --- DADOS DA TABELA ---a
    menor de 17 anos & - & 0.4\% & 10.1\% & 0.2\%  \\ \hline
    17 anos & 0.3\% & 15.9\% & 6.6\% & -  \\ \hline
    18 anos & 6.3\% & 16.0\% & 0.6\% & -  \\ \hline
    19 anos & 8.8\% & 2.4\% & 0.1\% & -  \\ \hline
    20 anos & 6.2\% & 0.6\% & - & -  \\ \hline
    21 anos & 4.3\% & 0.2\% & - & -  \\ \hline
    22 anos & 3.1\% & 0.1\% & - & -  \\ \hline
    23 anos & 2.4\% & - & - & -  \\ \hline
    24 anos & 1.9\% & - & - & -  \\ \hline
    25 anos & 1.5\% & - & - & -  \\ \hline
    26 a 30 anos & 4.5\% & 0.1\% & - & -  \\ \hline
    31 a 35 anos & 2.4\% & - & - & -  \\ \hline
    36 a 40 anos & 1.7\% & - & - & -  \\ \hline
    41 a 45 anos & 1.2\% & - & - & -  \\ \hline
    46 a 50 anos & 0.7\% & - & - & -  \\ \hline
    51 a 55 anos & 0.4\% & - & - & -  \\ \hline
    56 a 60 anos & 0.2\% & - & - & -  \\ \hline
    61 a 65 anos & 0.1\% & - & - & -  \\ \hline
    66 a 70 anos & - & - & - & -  \\ \hline
    maior de 70 anos & - & - & - & -  \\ \hline

\end{longtable}

\section{Treinamento dos Modelos}\label{resultados_treinamento_modelos}

\subsection{Ajuste dos Hiperparâmetros}\label{resultados_treinamento_modelos_ajuste_hiperparametros}

A primeira etapa do treinamento dos modelos foi o ajuste dos hiperparâmetros utilizando a técnica de \textit{Grid Search}. Inicialmente, foi utilizado o método \texttt{GridSearchCV} da biblioteca \texttt{scikit-learn} \cite{scikit-learn} para realizar o ajuste dos hiperparâmetros dos modelos. Porém, a execução do código foi interrompida subitamente algumas vezes, possivelmente devido ao alto consumo de memória da GPU.

Assim, para contornar esse problema, o \textit{Grid Search} foi implementado manualmente. Foi
estabelecido um dicionário com os hiperparâmetros e seus respectivos valores a serem testados para cada modelo e gerada uma lista com todas as combinações possíveis desses hiperparâmetros. Em seguida, através de um \textit{loop}, cada combinação de hiperparâmetros foi utilizada para instanciar cada modelo, treinar o modelo com os dados de treino e avaliar o desempenho do modelo com os dados de validação.

Para gerar o conjunto de dados de validação, foi feita uma separação no conjunto de dados de treino de forma que o conjunto de validação tenha 10\% dos dados originais, considerando o conjunto de teste que já foi separado. O desempenho do modelo foi avaliado utilizando a métrica \textit{Root Mean Squared Error} - RMSE (raiz quadrada do erro quadrático médio).

Para cada algortimo, foi estabelecido um conjunto de hiperparâmetros e seus respectivos valores a serem testados, assim como hiperparâmetros fixos em valores pré-definidos. As tabelas \ref{tab_hiperparametros_grid_xgb} a \ref{tab_hiperparametros_grid_rf} apresentam os hiperparâmetros e seus respectivos valores a serem testados para cada modelo, bem como os hiperparâmetros fixados em valores pré-definidos.

% -------------------- XGBoost ---------------------
%                      0     1     2
% learning_rate     0.05  0.10   0.2
% max_depth         6.00  8.00  10.0
% min_child_weight  1.00  5.00  10.0
% colsample_bytree  0.70  0.85   1.0
% subsample         0.70  0.85   1.0
                              
% n_estimators               100
% objective     reg:squarederror
% tree_method               hist
% device                    cuda
% eval_metric               rmse

\begin{longtable}{|c|c|}

    % --- TÍTULO E RÓTULO (1ª Página) ---
    \caption{\textit{Grid Search} - \textit{XGBoost}} \label{tab_hiperparametros_grid_xgb} \\
    \hline
    \textbf{Hiperparâmetro} & \textbf{Valores} \\ \hline
    \endfirsthead

    % --- CABEÇALHO (Páginas Seguintes) ---
    \hline
    \textbf{Hiperparâmetro} & \textbf{Valores} \\ \hline
    \endhead

    % --- RODAPÉ (Páginas Intermediárias) ---
    \hline
    \multicolumn{2}{r}{\footnotesize\textit{Continua na próxima página...}} \\
    \endfoot

    % --- RODAPÉ FINAL (Última Página) ---
    \hline
    \multicolumn{2}{c}{\footnotesize Fonte: elaborado pelo autor.} \\
    \endlastfoot

    % --- DADOS DA TABELA ---
    \texttt{learning\_rate} & \texttt{[0.05, 0.10, 0.20]} \\ \hline
    \texttt{max\_depth} & \texttt{[6, 8, 10]} \\ \hline
    \texttt{min\_child\_weight} & \texttt{[1, 5, 10]} \\ \hline
    \texttt{colsample\_bytree} & \texttt{[0.70, 0.85, 1.0]} \\ \hline
    \texttt{subsample} & \texttt{[0.70, 0.85, 1.0]} \\ \hline
    \texttt{n\_estimators} & \texttt{100} \\ \hline
    \texttt{objective} & \texttt{"reg:squarederror"} \\ \hline
    \texttt{tree\_method} & \texttt{"hist"} \\ \hline
    \texttt{device} & \texttt{"cuda"} \\ \hline
    \texttt{eval\_metric} & \texttt{"rmse"} \\ \hline

\end{longtable}

% -------------------- LightGBM --------------------
%                        0      1      2
% num_leaves         31.00  63.00  127.0
% learning_rate       0.05   0.10    0.2
% min_child_samples  20.00  50.00  100.0
% colsample_bytree    0.70   0.85    1.0
% subsample           0.70   0.85    1.0
                        
% n_estimators         100
% objective     regression
% metric              rmse
% device               cpu
% n_jobs                -1


\begin{longtable}{|c|c|}

    % --- TÍTULO E RÓTULO (1ª Página) ---
    \caption{\textit{Grid Search} - \textit{LightGBM}} \label{tab_hiperparametros_grid_lgbm} \\
    \hline
    \textbf{Hiperparâmetro} & \textbf{Valores} \\ \hline
    \endfirsthead

    % --- CABEÇALHO (Páginas Seguintes) ---
    \hline
    \textbf{Hiperparâmetro} & \textbf{Valores} \\ \hline
    \endhead

    % --- RODAPÉ (Páginas Intermediárias) ---
    \hline
    \multicolumn{2}{r}{\footnotesize\textit{Continua na próxima página...}} \\
    \endfoot

    % --- RODAPÉ FINAL (Última Página) ---
    \hline
    \multicolumn{2}{c}{\footnotesize Fonte: elaborado pelo autor.} \\
    \endlastfoot

    % --- DADOS DA TABELA ---
    \texttt{num\_leaves} & \texttt{[31, 63, 127]} \\ \hline
    \texttt{learning\_rate} & \texttt{[0.05, 0.10, 0.20]} \\ \hline
    \texttt{min\_child\_samples} & \texttt{[20, 50, 100]} \\ \hline
    \texttt{colsample\_bytree} & \texttt{[0.70, 0.85, 1.0]} \\ \hline
    \texttt{subsample} & \texttt{[0.70, 0.85, 1.0]} \\ \hline
    \texttt{n\_estimators} & \texttt{100} \\ \hline
    \texttt{objective} & \texttt{"regression"} \\ \hline
    \texttt{metric} & \texttt{"rmse"} \\ \hline
    \texttt{device} & \texttt{"cpu"} \\ \hline
    \texttt{n\_jobs} & \texttt{-1} \\ \hline

\end{longtable}

% ----------------- Random Forest ------------------
%                  0     1     2
% max_depth     10.0  15.0  20.0
% max_features   0.7   0.9   1.0
% max_samples    0.8   0.9   1.0
                      
% split_criterion    mse
% bootstrap         True
% n_bins             256
% min_samples_leaf    15
% n_streams            4
% n_estimators       100

\begin{longtable}{|c|c|}

    % --- TÍTULO E RÓTULO (1ª Página) ---
    \caption{\textit{Grid Search} - \textit{Random Forest}} \label{tab_hiperparametros_grid_rf} \\
    \hline
    \textbf{Hiperparâmetro} & \textbf{Valores} \\ \hline
    \endfirsthead

    % --- CABEÇALHO (Páginas Seguintes) ---
    \hline
    \textbf{Hiperparâmetro} & \textbf{Valores} \\ \hline
    \endhead

    % --- RODAPÉ (Páginas Intermediárias) ---
    \hline
    \multicolumn{2}{r}{\footnotesize\textit{Continua na próxima página...}} \\
    \endfoot

    % --- RODAPÉ FINAL (Última Página) ---
    \hline
    \multicolumn{2}{c}{\footnotesize Fonte: elaborado pelo autor.} \\
    \endlastfoot

    % --- DADOS DA TABELA ---
    \texttt{max\_depth} & \texttt{[10, 15, 20]} \\ \hline
    \texttt{max\_features} & \texttt{[0.7, 0.9, 1.0]} \\ \hline
    \texttt{max\_samples} & \texttt{[0.8, 0.9, 1.0]} \\ \hline
    \texttt{split\_criterion} & \texttt{"mse"} \\ \hline
    \texttt{bootstrap} & \texttt{True} \\ \hline
    \texttt{n\_bins} & \texttt{256} \\ \hline
    \texttt{min\_samples\_leaf} & \texttt{15} \\ \hline
    \texttt{n\_streams} & \texttt{4} \\ \hline
    \texttt{n\_estimators} & \texttt{100} \\ \hline

\end{longtable}



Foram então treinados 243 combinações de hiperparâmetros para os algoritmos de \textit{XGBoost} e \textit{LightGBM} (5 hiperparâmetros com 3 valores cada) e 27 combinações de hiperparâmetros para o algoritmo de \textit{Random Forest} (3 hiperparâmetros com 3 valores cada)totalizando 270 modelos treinados para cada conjunto de dados (Humanas, Natureza, Linguagem, Matemática e Redação) e 1.350 modelos treinados no total, o que levou aproximadamente 5 horas para ser executado.

Os gráficos \ref{fig_erro_grid_xgb} a \ref{fig_erro_grid_rf} apresentam o erro RMSE para cada combinação de hiperparâmetros testada para cada modelo e conjunto de dados, com as interações do \textit{Grid Search} ordenadas em ordem decrescente de erro.

\begin{figure}[H]
    \centering
    \caption{Erro do \textit{Grid Search} - \textit{XGBoost}} \label{fig_erro_grid_xgb}
    \includegraphics[width=0.7\linewidth]{imagens/erro_grid_xgb.png}
    \par\vspace{0.1cm}
    % {\footnotesize Fonte: elaborado pelo autor.}
    {\footnotesize Obs.: apenas metade dos dados foi plotada para melhor visualização.}
\end{figure}

\begin{figure}[H]
    \centering
    \caption{Erro do \textit{Grid Search} - \textit{LightGBM}} \label{fig_erro_grid_lgbm}
    \includegraphics[width=0.7\linewidth]{imagens/erro_grid_lgbm.png}
    \par\vspace{0.1cm}
    % {\footnotesize Fonte: elaborado pelo autor.}
    {\footnotesize Obs.: apenas metade dos dados foi plotada para melhor visualização.}
\end{figure}

\begin{figure}[H]
    \centering
    \caption{Erro do \textit{Grid Search} - \textit{Random Forest}} \label{fig_erro_grid_rf}
    \includegraphics[width=0.7\linewidth]{imagens/erro_grid_rf.png}
    \par\vspace{0.1cm}
    {\footnotesize Fonte: elaborado pelo autor.}
\end{figure}

A redução no erro RMSE para as melhores combinações de hiperparâmetros acabou não sendo tão significativa para os modelos de \textit{XGBoost} e \textit{LightGBM}, apresentando uma redução menor que 1\% em relação ao maior erro RMSE encontrado na \textit{Grid Search}. Já para o modelo de \textit{Random Forest}, a redução no erro RMSE foi mais significativa, apresentando uma redução de aproximadamente 2,2\% em relação ao maior erro RMSE encontrado na \textit{Grid Search}.

Até este momento, não ajustamos o hiperparâmetro \texttt{n\_estimators} (número de estimadores), uma vez que a execução do código para as combinações com esse parâmetro maior que 100 foi interrompida subitamente algumas vezes, também possivelmente devido ao alto consumo de memória da GPU. Dessa forma, o ajuste desse hiperparâmetro será realizado na próxima etapa, juntamente com o treinamento final dos modelos.

As tabelas \ref{tab_hiperparametros_ajustados_xgb} a \ref{tab_hiperparametros_ajustados_rf} apresentam os melhores hiperparâmetros encontrados para cada modelo em cada conjunto de dados, bem como o erro RMSE correspondente a cada combinação de hiperparâmetros.

% ========================= XGBoost ==========================

%                   Ciências Humanas  Ciências Natureza  Linguagem e Código  \
% rmse                     75.351906          65.120758           63.051487   
% learning_rate             0.100000           0.100000            0.100000   
% max_depth                10.000000          10.000000           10.000000   
% min_child_weight         10.000000          10.000000            1.000000   
% colsample_bytree          0.700000           0.700000            0.700000   
% subsample                 1.000000           1.000000            1.000000   

%                   Matemática     Redação  
% rmse               95.976074  148.119446  
% learning_rate       0.100000    0.100000  
% max_depth          10.000000   10.000000  
% min_child_weight    5.000000    5.000000  
% colsample_bytree    0.700000    0.700000  
% subsample           1.000000    1.000000  

\begin{longtable}{|l|c|c|c|c|c|}

    % --- TÍTULO E RÓTULO (1ª Página) ---
    \caption{Hiperparâmetros Ajustados - \textit{XGBoost}} \label{tab_hiperparametros_ajustados_xgb} \\
    \hline
    \textbf{Hiperparâmetro} & \textbf{Humanas} & \textbf{Natureza} & \textbf{Linguagem} & \textbf{Matemática} & \textbf{Redação} \\ \hline
    \endfirsthead

    % --- CABEÇALHO (Páginas Seguintes) ---
    \hline
    \textbf{Hiperparâmetro} & \textbf{Humanas} & \textbf{Natureza} & \textbf{Linguagem} & \textbf{Matemática} & \textbf{Redação} \\ \hline
    \endhead

    % --- RODAPÉ (Páginas Intermediárias) ---
    \hline
    \multicolumn{6}{r}{\footnotesize\textit{Continua na próxima página...}} \\
    \endfoot

    % --- RODAPÉ FINAL (Última Página) ---
    \hline
    % \multicolumn{6}{c}{\footnotesize Fonte: elaborado pelo autor.} \\
    \endlastfoot

    % --- DADOS DA TABELA ---a
    \texttt{learning\_rate} & \texttt{0.1} & \texttt{0.1} & \texttt{0.1} & \texttt{0.1} & \texttt{0.1} \\ \hline
    \texttt{max\_depth} & \texttt{10.0} & \texttt{10.0} & \texttt{10.0} & \texttt{10.0} & \texttt{10.0} \\ \hline
    \texttt{min\_child\_weight} & \texttt{10.0} & \texttt{10.0} & \texttt{1.0} & \texttt{5.0} & \texttt{5.0} \\ \hline
    \texttt{colsample\_bytree} & \texttt{0.7} & \texttt{0.7} & \texttt{0.7} & \texttt{0.7} & \texttt{0.7} \\ \hline
    \texttt{subsample} & \texttt{1.0} & \texttt{1.0} & \texttt{1.0} & \texttt{1.0} & \texttt{1.0} \\ \hline
    RMSE & 75,4 & 65,1 & 63,1 & 96,0 & 148,1 \\ \hline

\end{longtable}

% ========================= LightGBM =========================

%                    Ciências Humanas  Ciências Natureza  Linguagem e Código  \
% rmse                       75.36182          65.139965           63.074187   
% num_leaves                127.00000         127.000000          127.000000   
% learning_rate               0.20000           0.200000            0.200000   
% min_child_samples         100.00000          50.000000           50.000000   
% colsample_bytree            0.70000           0.700000            0.850000   
% subsample                   1.00000           0.700000            1.000000   

%                    Matemática     Redação  
% rmse                96.013921  148.150517  
% num_leaves         127.000000  127.000000  
% learning_rate        0.200000    0.200000  
% min_child_samples  100.000000  100.000000  
% colsample_bytree     0.700000    0.700000  
% subsample            1.000000    1.000000  

\begin{longtable}{|l|c|c|c|c|c|}

    % --- TÍTULO E RÓTULO (1ª Página) ---
    \caption{Hiperparâmetros Ajustados - \textit{LightGBM}} \label{tab_hiperparametros_ajustados_lgbm} \\
    \hline
    \textbf{Hiperparâmetro} & \textbf{Humanas} & \textbf{Natureza} & \textbf{Linguagem} & \textbf{Matemática} & \textbf{Redação} \\ \hline
    \endfirsthead

    % --- CABEÇALHO (Páginas Seguintes) ---
    \hline
    \textbf{Hiperparâmetro} & \textbf{Humanas} & \textbf{Natureza} & \textbf{Linguagem} & \textbf{Matemática} & \textbf{Redação} \\ \hline
    \endhead

    % --- RODAPÉ (Páginas Intermediárias) ---
    \hline
    \multicolumn{6}{r}{\footnotesize\textit{Continua na próxima página...}} \\
    \endfoot

    % --- RODAPÉ FINAL (Última Página) ---
    \hline
    % \multicolumn{6}{c}{\footnotesize Fonte: elaborado pelo autor.} \\
    \endlastfoot

    % --- DADOS DA TABELA ---a
    \texttt{num\_leaves} & \texttt{127.0} & \texttt{127.0} & \texttt{127.0} & \texttt{127.0} & \texttt{127.0} \\ \hline
    \texttt{learning\_rate} & \texttt{0.2} & \texttt{0.2} & \texttt{0.2} & \texttt{0.2} & \texttt{0.2} \\ \hline
    \texttt{min\_child\_samples} & \texttt{100.0} & \texttt{50.0} & \texttt{50.0} & \texttt{100.0} & \texttt{100.0} \\ \hline
    \texttt{colsample\_bytree} & \texttt{0.7} & \texttt{0.7} & \texttt{0.85} & \texttt{0.7} & \texttt{0.7} \\ \hline
    \texttt{subsample} & \texttt{1.0} & \texttt{0.7} & \texttt{1.0} & \texttt{1.0} & \texttt{1.0} \\ \hline
    RMSE & 75,4 & 65,1 & 63,1 & 96,0 & 148,2 \\ \hline

\end{longtable}

% ====================== Random Forest =======================

%               Ciências Humanas  Ciências Natureza  Linguagem e Código  \
% rmse                 75.642021          65.390625           63.319668   
% max_depth            20.000000          20.000000           20.000000   
% max_features          0.700000           0.700000            0.700000   
% max_samples           0.800000           0.800000            0.800000   

%               Matemática     Redação  
% rmse           96.473152  148.628891  
% max_depth      20.000000   20.000000  
% max_features    0.700000    0.700000  
% max_samples     0.800000    0.800000

\begin{longtable}{|l|c|c|c|c|c|}

    % --- TÍTULO E RÓTULO (1ª Página) ---
    \caption{Hiperparâmetros Ajustados - \textit{Random Forest}} \label{tab_hiperparametros_ajustados_rf} \\
    \hline
    \textbf{Hiperparâmetro} & \textbf{Humanas} & \textbf{Natureza} & \textbf{Linguagem} & \textbf{Matemática} & \textbf{Redação} \\ \hline
    \endfirsthead

    % --- CABEÇALHO (Páginas Seguintes) ---
    \hline
    \textbf{Hiperparâmetro} & \textbf{Humanas} & \textbf{Natureza} & \textbf{Linguagem} & \textbf{Matemática} & \textbf{Redação} \\ \hline
    \endhead

    % --- RODAPÉ (Páginas Intermediárias) ---
    \hline
    \multicolumn{6}{r}{\footnotesize\textit{Continua na próxima página...}} \\
    \endfoot

    % --- RODAPÉ FINAL (Última Página) ---
    \hline
    \multicolumn{6}{c}{\footnotesize Fonte: elaborado pelo autor.} \\
    \endlastfoot

    % --- DADOS DA TABELA ---a
    \texttt{max\_depth} & \texttt{20.0} & \texttt{20.0} & \texttt{20.0} & \texttt{20.0} & \texttt{20.0} \\ \hline
    \texttt{max\_features} & \texttt{0.7} & \texttt{0.7} & \texttt{0.7} & \texttt{0.7} & \texttt{0.7} \\ \hline
    \texttt{max\_samples} & \texttt{0.8} & \texttt{0.8} & \texttt{0.8} & \texttt{0.8} & \texttt{0.8} \\ \hline
    RMSE & 75,6 & 65,4 & 63,3 & 96,4 & 148,6 \\ \hline

\end{longtable}

\section{Avaliação}\label{resultados_avaliacao}
\section{Implantação}\label{resultados_implantacao}