
% !TEX root = ./TCC_Impresso.tex
% !TEX root = ./TCC_Online.tex

\chapter{Metodologia}\label{cap_metodologia}

Este capítulo detalha a metodologia de trabalho utilizada, apresentando o delineamento da pesquisa, da coleta e processamento dos dados e as técnicas analíticas empregadas para responder às perguntas de pesquisa.

A estrutura metodológica adotada será baseada no modelo CRISP-DM (\textit{Cross-Industry Standard Process for Data Mining}) \cite{Chapman2000CRISPDM}, contendo as etapas de (i) entendimento de negócio, (ii) entendimento dos dados, (iii) preparação dos dados, (iv) modelagem, (v) avaliação e (vi) implantação.

\begin{figure}[H]
    \centering
    \caption{Modelo CRISP-DM}
    \includegraphics[width = 0.7\textwidth]{imagens/crisp-dm.png}
    \label{fig:crisp-dm}
\begin{center}
    \footnotesize Fonte: modificado de Chapman \textit{et al.} \cite{Chapman2000CRISPDM}.
\end{center}
\end{figure}

\section{Entendimento de Negócio}\label{metodologia_entendimento_negocio}

A etapa de entendimento de negócio envolve a definição clara dos objetivos do projeto, a compreensão do contexto em que a pesquisa está inserida, a identificação das partes interessadas, a formulação das perguntas de pesquisa que guiarão a análise dos dados e os resultados que espera-se alcançar.

O foco principal deste trabalho será a formulação de hipóteses relacionadas aos fatores que influenciam o desempenho dos estudantes no ENEM, através da análise dos dados de performance dos participantes e suas características socioeconômicas.

Sendo assim, foi necessário formular perguntas de pesquisa específicas que possam ser respondidas através da análise dos dados disponíveis.

\section{Entendimento dos dados}\label{metodologia_entendimento_dados}

Com as perguntas de pesquisa definidas, a próxima etapa foi encontrar dados que sejam adequados para responder a essas perguntas e estabelecer, primeiramente, uma forma consistente de coleta e armazenamento para em seguida realizar uma rápida compreensão da estrutura dos dados.

Há depender dos tipos de dados a serem utilizados, é necessário submeter o projeto a um comitê de ética em pesquisa para aprovação, garantindo que todos os aspectos éticos relacionados ao uso dos dados sejam devidamente considerados.

Após essa etapa, foi possível identifcar quais os arquivos são relevantes para a análise e quais variáveis dentro desses arquivos serão utilizadas como variáveis preditoras e como variáveis resposta.

\section{Preparação dos dados}\label{metodologia_preparacao_dados}

Com os arquivos relevantes selecionados, passa-se para a etapa de preparação dos dados, que envolve a leitura, limpeza, transformação e integração dos dados para torná-los adequados para a modelagem. Essa etapa é crucial, pois a qualidade dos dados impacta diretamente na eficácia dos modelos preditivos que serão construídos posteriormente.

Para a execução dessa e das etapas posteriores, é necessário preparar um ambiente tecnológico e analítico adequado que permita a manipulação eficiente dos dados e a construção dos modelos preditivos.

Após a leitura dos dados, estes foram integrados em único conjunto de dados para facilitar a análise e a modelagem. Foram realizados os ajustes necessários no esquema dos dados para garantir a consistência e a integridade das informações.

Em seguida, as variáveis foram renomeadas para nomes mais intuitivos e de fácil compreensão e foi analisado se seria necessário criar variáveis contendo alguma transformação das variáveis originais, por exemplo, criar uma variável que transforme os códigos numéricos de variáveis categóricas em rótulos textuais.

Posteriormente, foi realizada uma análise para identificar e tratar valores nulos, removendo ou imputando valores conforme apropriado. Após o tratamento dos valores nulos, os dados foram separados em diferentes conjuntos de dados, cada um correspondente a uma variável resposta específica, garantindo que cada conjunto contenha apenas as observações relevantes para a análise daquela variável e sem valores nulos.

\section{Modelagem}\label{metodologia_modelagem}



\section{Avaliação}\label{metodologia_avaliacao}

\section{Implantação}\label{metodologia_implantacao}




























% \subsection{Coleta, Leitura e Integração dos Dados}\label{sec_metodologia_coleta_integracao_dados}


% Em seguida, foi feita uma modificação no nome das variáveis para nomes que fossem mais intuitivos e de compreensão rápida do conteúdo. Essa modificação foi realizada utilizando o método \texttt{rename}, a partir de um dicionário que mapeava os nomes originais para os novos nomes desejados.

% \subsection{Transformação das Variáveis Categóricas}\label{sec_metodologia_transformacao_variaveis_categoricas}

% Analisando o dicionário de dados, foi possível identificar diversas variáveis categóricas representadas por códigos numéricos. Essas variáveis precisam ser transformadas para garantir que os modelos de \textit{Machine Learning} as interpretem corretamente como categóricas e não como variáveis numéricas.

% Os códigos numéricos foram então substituídos por rótulos textuais mais descritivos utilizando o método \texttt{map} com um dicionário de mapeamento específico para cada variável categórica. Alguns códigos represenram ausência de informação e foram substituídos por \texttt{None} para um futuro tratamento de valores ausentes adequado.

% \subsection{Tratamentos dos Valores Nulos}\label{sec_metodologia_tratamento_nulos}

% Após a integração dos dados, foi realizada uma análise inicial para identificar a presença de valores nulos em cada variável. Dependendo do percentual de valores nulos e da natureza da variável, diferentes estratégias de tratamento foram adotadas.

% As variáveis com mais de 50\% de valores nulos foram removidas do conjunto de dados, pois a alta proporção de valores ausentes poderia comprometer a análise e a modelagem subsequente.

% As variáveis de notas das provas objetivas e da redação passaram por uma análise mais detalhada de valores nulos em conjunto com as variáveis de presença nas respecitivas provas para entender o significado dos valores nulos e decidir o tratamento adequado. No fim, as observações com notas nulas foram removidas, enquanto as observações com notas zero foram mantidas, pois representam casos distintos.

% As demais variáveis com valores nulos foram consideradas de maneira consolidada, ou seja, caso uma observação possua valor ausente em qualquer uma dessas variáveis, ela será considerada como tendo valor ausente. Essas observações foram então removidas do conjunto de dados.

% \subsection{Separação dos Conjuntos de Dados por Variável Resposta}\label{sec_metodologia_separacao_conjuntos_dados}

% Como possuímos cinco variáveis resposta distintas, é necessário preparar os dados para cada uma delas. Assim, foram criados cinco conjuntos de dados distintos, um para cada variável resposta. O filtro aplicado é o mesmo para todos os conjuntos de dados: foram mantidas apenas as observações que possuem presença na prova correspondente. Para a nota da redação, foi utilizada a variável de presença na prova de Linguagens e Códigos, que é quando a redação é aplicada.

% ALém disso, foram retiradas as variáveis de nota das demais provas, assim como as variáveis de presença nas demais provas, para evitar qualquer tipo de \textit{data leakage} durante a modelagem.

% Foi utilizado a estrutura de dicionário para armazenar os cinco conjuntos de dados distintos, facilitando o acesso e a manipulação dos dados durante a análise.

% \subsection{Exploração Inicial}\label{sec_metodologia_exploracao_inicial}

% Com os dados no formato desejado, foi realizada uma exploração inicial para compreender a estrutura dos dados e e detectar possíveis inconsistências.

% Primeiramente, foi analisada a distriuição das notas para cada variável resposta, utilizando histogramas e boxplots. Foi constatado que a nota da redação é aplicada em intervalos de 20 pontos, enquanto que as demais notas são aplicadas de maneira contínua. Isso se dá pela quantidade muito diferente de notas distintas existentes em cada prova: a redação possui 50 valores distintos, enquanto que as demais provas possuem mais de 5.200 valores distintos cada.








% \section{Análise de correlação das Variáveis Preditoras}\label{analise_correlacao_variaveis}

% Antes de iniciar a modelagem, foi realizada uma análise de correlação entre as variáveis preditoras para identificar possíveis multicolinearidades que poderiam afetar o desempenho dos modelos. A correlação foi avaliada utilizando a biblioteca \texttt{phik} \cite{KPMG_PhiK_2024} \cite{Baak2020Phik}, que permite calcular a correlação entre variáveis categóricas e numéricas.

% \subsection{Imputação de Valores Ausentes}\label{imputacao_valores_ausentes}

% Para as variáveis com valores ausentes que não foram removidas, foi adotado o método de \textit{KNN Imputer}, que utiliza a média/moda dos \textit{k} vizinhos mais próximos para preencher os valores ausentes. Esse método foi escolhido por sua capacidade de preservar a distribuição dos dados e considerar a correlação entre as variáveis.

% O \textit{KNN Imputer} foi implementado utilizando a classe \texttt{KNNImputer} da biblioteca \texttt{sklearn.impute}. Para se determinar o hiperparâmetro \texttt{k}, foi feita uma rápida análise de sensibilidade, testando diferentes valores de \texttt{k} e avaliando o impacto na distribuição das variáveis imputadas e na performance dos modelos preditivos. Para cada variável a ser imputada, foi selecionado o valor de \texttt{k} que apresentou o melhor equilíbrio entre acurácia e estabilidade.

% \section{Modelagem}\label{modelagem}

% A modelagem é a etapa onde os dados preparados são utilizados para construir modelos preditivos que possam responder às perguntas de pesquisa formuladas na Seção \ref{entendimento_negocio}. Nesta seção, serão detalhadas as técnicas de modelagem empregadas e os critérios utilizados para a seleção dos modelos.

% \subsection{Seleção de Modelos}\label{selecao_modelos}

% Como as nossas variáveis respostas são numéricas e contínuas, estamos em um problema de regressão de aprendizado supervisionado. Assim, é preciso selecionar modelos de regressão, que são os adequados para prever variáveis contínuas. Os modelos escolhidos foram a Regressão Linear com Regularizações (como \textit{benchmark}) e o XGBoost Regressor \cite{XGBoostDevelopers_Tutorials_2025}, ambos amplamente utilizados devido à sua eficácia e interpretabilidade.

% \subsection{Otimização dos Hiperparâmetros}\label{otimizacao_hiperparametros}

% Para otimizar o desempenho dos modelos selecionados, foi realizada uma busca em grade (\textit{Grid Search}) para identificar os melhores hiperparâmetros. A busca foi realizada utilizando validação cruzada para garantir que os resultados fossem robustos e generalizáveis.

% \subsection{Treino dos Modelos}\label{treino_modelos}

% Definidos os hiperparâmetros ótimos, os modelos foram treinados utilizando o conjunto de dados preparado na Seção \ref{preparacao_dados}. O treinamento foi realizado em validação cruzada de cinco \textit{folds}, utilizando a biblioteca \texttt{cuml} para se beneficiar da aceleração por GPU. Foi selecionado como modelo final aquele modelo que apresentou o melhor desempenho médio nos \textit{folds} de validação para cada variável resposta.

% \subsection{Avaliação dos Modelos}\label{avaliacao_modelos}

% A avaliação dos modelos foi realizada utilizando métricas de desempenho apropriadas para problemas de regressão, como a Raiz do Erro Quadrático Médio (RMSE) e o Coeficiente de Determinação (R²). Essas métricas fornecem uma visão clara da precisão das previsões dos modelos em relação aos valores reais das notas do ENEM.

% Também foram realizadas análises de resíduos para verificar a adequação dos modelos e identificar possíveis padrões não capturados pelas previsões, bem como a análise gráfica das previsões versus os valores reais para avaliar visualmente o desempenho dos modelos.

% \section{Magnitude da Influência das Variáveis Preditoras}\label{medicao_efeito_variaveis}

% Para medir o efeito de cada variável preditora nas notas do ENEM, foram utilizadas técnicas de interpretação de modelos, que permitem identificar quais variáveis têm maior impacto nas previsões dos modelos e como elas influenciam as notas dos estudantes. Entre as técnicas empregadas estão a importância das características (\textit{feature importance}), a análise dos valores SHAP (\textit{SHapley Additive exPlanations}) e os gráficos de dependência parcial (\textit{partial dependence plots}).

% Além disso, foi realizada uma análise de sensibilidade para avaliar como mudanças nas variáveis preditoras afetam as previsões dos modelos. Para isso, foi construída uma matriz de cenários hipotéticos, onde cada variável preditora é alterada sistematicamente enquanto as outras são mantidas constantes. As previsões resultantes foram então analisadas para quantificar o impacto de cada variável nas notas do ENEM.

% \section{Limitações e Considerações Éticas}\label{limitacoes_eticas}

% Algumas limitações devem ser destacadas. Primeiro, apesar do grande volume de dados, a presença de vieses de seleção (por exemplo, diferença entre participantes regulares e treineiros, ou entre ausentes e presentes nas provas) pode influenciar as inferências; as análises procuram mitigar esses efeitos, mas não os eliminam completamente. Segundo, variáveis censuradas, inconsistências de registro e códigos especiais para ausência exigiram tratamentos que podem introduzir perdas de informação.

% Quanto às considerações éticas, todos os dados utilizados são microdados públicos disponibilizados pelo INEP, já anonimizados para preservar a privacidade dos participantes. Recomenda-se cautela na interpretação dos resultados para evitar conclusões simplistas sobre individualidades dos estudantes ou estigmatização de grupos e instituições.

% No Capítulo \ref{cap_resultados} serão apresentados os resultados empíricos da metodologia empregada: (i) a descrição dos dados após o pré-processamento, (ii) a análise de correlação entre as variáveis preditoras, (iii) o desempenho dos modelos preditivos na previsão das notas do ENEM, e (iv) a análise da magnitude da influência das variáveis preditoras nas notas dos estudantes.