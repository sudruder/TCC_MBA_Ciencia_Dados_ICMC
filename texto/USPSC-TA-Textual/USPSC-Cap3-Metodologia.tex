
\chapter{Metodologia}\label{cap_metodologia}

Este capítulo detalha a metodologia de trabalho utilizada, apresentando o delineamento da pesquisa, da coleta e processamento dos dados e as técnicas analíticas empregadas para responder às perguntas de pesquisa. A estrutura metodológica adotada será baseada no modelo CRISP-DM (Cross-Industry Standard Process for Data Mining) \cite{Chapman2000CRISPDM}, contendo as etapas de (i) entendimento de negócio, (ii) entendimento dos dados, (iii) preparação dos dados, (iv) modelagem, (v) avaliação e (vi) implantação.

\section{Entendimento de Negócio}\label{entendimento_negocio}

O entendimento de negócio é a primeira etapa do modelo CRISP-DM e envolve a definição clara dos objetivos do projeto, a compreensão do contexto em que a pesquisa está inserida e a identificação das partes interessadas. O foco principal deste trabalho acadêmico será a formulação de hipóteses relacionadas aos fatores que influenciam o desempenho dos estudantes no ENEM, bem como a análise do ``Efeito Escola''.

Conforme já mencionado no Capítulo \ref{cap_fundamentacao} - Fundamentação Teórica, o ENEM é um exame de grande relevância no contexto educacional brasileiro, e compreender os fatores que impactam o desempenho dos estudantes é crucial para a formulação de políticas educacionais eficazes. Trabalhos anteriores citam algumas variáveis socioeconômicas como discriminadores de performance no ENEM. A Tabela \ref{tab_variaveis_socioeconomicas} apresenta as principais variáveis socioeconômicas identificadas na literatura como relevantes para a análise do desempenho dos estudantes no ENEM, juntamente com suas respectivas referências.



\begin{table}[h]

    \centering

    \caption{Variáveis socioeconômicas e suas referências}

    \begin{tabular}{|m{7cm}|m{4cm}|} \hline

        \textbf{Variável socioeconômica} & \textbf{Referência} \\ \hline
        Renda familiar & Melo \textit{et al.} \cite{ref_01} \\ & Vasconcellos \cite{ref_06} \\ \hline
        Raça / Cor & Melo \textit{et al.} \cite{ref_01} \\ & Moraes \textit{et al.} \cite{ref_03} \\ \hline
        Sexo & Moraes \textit{et al.} \cite{ref_03} \\ \hline
        Idade / Atraso Escolar & Jaloto e Primi \cite{ref_12} \\ \hline
        Administração: Pública vs. Privada & Moraes \textit{et al.} \cite{ref_03} \\ & Jaloto e Primi \cite{ref_12} \\ & Ortega \textit{et al.} \cite{ref_05} \\ \hline
        Atributos Escolares & Moraes \textit{et al.} \cite{ref_03} \\ \hline
    \end{tabular}

    \label{tab_variaveis_socioeconomicas}

    \begin{center}
        \footnotesize Fonte: do autor.
    \end{center}

\end{table}



Assim, para este trabalho, formulo as seguintes perguntas de pesquisa para este trabalho:

\begin{itemize}
    \item \textbf{Pergunta 1:} Quais são os principais fatores socioeconômicos, características da escola e particularidades regionais que influenciam o desempenho dos estudantes no ENEM?
    \item \textbf{Pergunta 2:} Qual é a magnitude da influência de cada um desses conjuntos de fatores nas notas dos participantes?
\end{itemize}

Com as perguntas de pesquisa definidas, o próximo passo é compreender os dados disponíveis para análise, conforme descrito na Seção \ref{entendimento_dados}.

\section{Entendimento dos Dados}\label{entendimento_dados}
 
Nesta etapa, o foco será a coleta e compreensão dos dados disponíveis para análise. Utilizarei duas fontes principais de dados: (i) os microdados do ENEM e (ii) e os dados do Censo Escolar, ambos disponibilizados pelo Instituto Nacional de Estudos e Pesquisas Educacionais Anísio Teixeira (INEP).

\subsection{Microdados do ENEM}\label{entendimento_microdados_enem}

Os microdados do ENEM são disponibilizados anualmente pelo INEP em formato aberto e em arquivos CSV (\textit{Comma-Separated Values}) \cite{ref_08}, o que facilita a sua manipulação e análise. Junto dos microdados, é disponibilizado também o dicionário de dados, que descreve detalhadamente cada variável presente no conjunto de dados. Os dados da edição de 2022 e 2023 possuem 76 variáveis e 3,4 milhões e 3,9 milhões de observações, respectivamente, onde cada observação é um candidato que se inscreveu para realizar o exame. O dicionário completo do Censo Escolar pode ser encontrado no Apêndice \ref{apendice_dicionario_enem}. A seguir, apresento um resumo do dicionário de dados do ENEM.

\begin{itemize}

    \item \textbf{Dados do participante:} Número de inscrição mascarado, ano do exame, faixa etária, sexo, estado civil, cor/raça, nacionalidade, situação de conclusão do Ensino Médio, ano de conclusão do Ensino Médio, tipo de escola do Ensino Médio, tipo de instituição que concluiu ou concluirá o Ensino Médio e se o inscrito fez a prova como treineiro.

    \item \textbf{Dados da escola:} Código e nome da escola, código e sigla da Unidade da Federação, código do município, nome do município, dependência administrativa, localização e situação de funcionamento.

    \item \textbf{Dados do local de aplicação da prova:} Código e nome da escola onde a prova foi aplicada, código e sigla da Unidade da Federação, código do município e nome do município.

    \item \textbf{Dados da prova Objetiva:} Presença e código do tipo de prova objetiva, nota das provas objetivas, vetor com as respostas da parte objetiva, língua estrangeira escolhida e vetor com o gabarito da parte objetiva.

    \item \textbf{Dados da redação: } Nota das competências e nota da prova de redação.

    \item \textbf{Dados do questionário socioeconômico:} Respostas do questionário socioeconômico aplicado aos participantes do ENEM.


\end{itemize}

A partir da interpretação do dicionario de dados, já é possível identificar diversas variáveis que podem ser utilizadas para responder às perguntas de pesquisa formuladas na Seção \ref{entendimento_negocio}. Essas variáveis incluem características demográficas dos participantes, informações sobre a escola de origem e dados socioeconômicos coletados por meio do questionário aplicado durante o exame.

\subsection{Censo Escolar}\label{entendimento_censo_escolar}

Os microdados do Censo Escolar são disponibilizados anualmente pelo INEP em formato aberto e em arquivos CSV (\textit{Comma-Separated Values}) \cite{ref_07}, o que facilita manipulação e análise. Junto dos microdados, é disponibilizado também o dicionário de dados, que descreve detalhadamente cada variável presente no conjunto de dados. Os dados da edição de 2022 e 2023 possuem 385 e 408 variáveis e 224 mil e 217 mil observações, respectivamente, onde cada observação é uma instituição de ensino do Brasil. No total, devido a diferenças de formato, temos 456 variáveis únicas entre as duas edições. O dicionário completo do Censo Escolar pode ser encontrado no Apêndice \ref{apendice_dicionario_censo}. A seguir, apresento um resumo do dicionário de dados do Censo Escolar.

\begin{itemize}

    \item \textbf{Dados de Identificação e Localização da Escola:} Ano do Censo, nome e código da escola, endereço completo e localização geográfica.

    \item \textbf{Dados de Administração e Funcionamento:} Dependência administrativa (federal, estadual, municipal, privada), categoria da escola privada, situação de funcionamento, datas de início e término do ano letivo, vinculação a órgãos públicos, detalhes sobre parcerias ou convênios com o poder público, informações da mantenedora das escolas privadas e status de regulamentação.

    \item \textbf{Dados de Infraestrutura e Dependências:} Local de funcionamento (prédio escolar, salas em empresa, unidade prisional, templo etc.), informações sobre abastecimento (água, energia e esgoto), destinação/tratamento do lixo, lista de dependências físicas existentes (biblioteca, cozinha, laboratórios, quadra de esportes, refeitório etc.) e recursos de acessibilidade (banheiro PcD, dependências acessíveis, corrimão, elevador, rampas e pisos táteis).

    \item \textbf{Dados de Equipamentos e Tecnologia:} Quantidade de salas de aula existentes, utilizadas, climatizadas e acessíveis. Equipamentos para uso administrativo e pedagógico. Detalhes sobre computadores e acesso à Internet.

    \item \textbf{Dados de Recursos Humanos:} Quantidade total de funcionários e quantidades específicas por função.

    \item \textbf{Dados Pedagógicos e Oferta de Ensino:} Forma de organização do ensino (série/ano , ciclos , módulos ou alternância), alimentação escolar, materiais pedagógicos e socioculturais disponíveis. Informações sobre oferta de Educação Indígena (incluindo línguas), Atendimento Educacional Especializado (AEE) e Atividade Complementar. Mediação pedagógica (presencial, semipresencial, EAD).

    \item \textbf{Dados de Gestão e Relação com a Comunidade:} Existência de órgãos colegiados (Associação de Pais, Conselho Escolar e Grêmio Estudantil), atualização da proposta pedagógica, existência de exame de seleção e reserva de vagas/cotas. Uso de redes sociais, compartilhamento de espaços e abertura da escola nos finais de semana.

    \item \textbf{Dados Consolidados:} Indicadores de oferta de etapas (Educação Básica, Infantil, Fundamental, Médio, EJA, Profissional e Especial). Quantidade total de matrículas, docentes e turmas, com detalhamento por etapa, turno, tempo integral, sexo, cor/raça, faixa etária e zona de residência. Dados sobre alunos que utilizam transporte escolar público.

\end{itemize}

A partir da interpretação do dicionario de dados, já é possível identificar diversas variáveis que podem ser utilizadas para responder às perguntas de pesquisa formuladas na Seção \ref{entendimento_negocio}. Essas variáveis incluem características administrativas e pedagógicas das escolas, infraestrutura disponível e recursos humanos.

\section{Preparação dos Dados}\label{preparacao_dados}

A preparação dos dados é uma etapa crucial no processo de análise, pois envolve a limpeza, transformação e integração dos dados para torná-los adequados para a modelagem. Nesta seção, detalharei as etapas realizadas para preparar os microdados do ENEM e do Censo Escolar para análise.

\subsection{Coleta e Importação dos Dados}\label{importacao_dados}

A coleta dos dados foi realizada por meio do download dos arquivos CSV disponibilizados pelo INEP para as edições de 2022 e 2023 do ENEM e do Censo Escolar. Os arquivos foram armazenados localmente para facilitar o acesso durante o processo de análise.

A leitura dos dados foi feitra através da biblioteca \texttt{pandas} do Python, que oferece funcionalidades robustas para manipulação de dados tabulares. A leitura foi feita utilizando o método \texttt{read\_csv}, especificando o separador como ponto e vírgula (\texttt{sep = ';'}), a codificação como \texttt{latin1} para lidar com caracteres especiais e o parâmetro \texttt{low\_memory = False} para otimizar a leitura de grandes arquivos.

\subsection{Integração e Modificação de Esquemas}\label{integracao_esquemas}

Nesta fase, foi necessário integrar os microdados do ENEM de cada edição. Os conjuntos de dados do ENEM de 2022 e 2023 possuem o mesmo esquema, ou seja, as mesmas variáveis estão presentes em ambos os anos e com o mesmo nome. Portanto, a integração foi realizada por meio da concatenação vertical dos dois conjuntos de dados, resultando em um único DataFrame contendo todas as observações de ambas as edições.

Em seguida, foi feita uma modificação no nome das variáveis para nomes que me fossem mais intuitivos e de compreensão rápida do conteúdo. Essa modificação foi realizada utilizando o método \texttt{rename} da biblioteca \texttt{pandas}, a partir de um dicionário que mapeava os nomes originais para os novos nomes desejados.

Para o Censo Escolar, as edições de 2022 e 2023 possuem esquemas ligeiramente diferentes, com algumas variáveis presentes em um ano e ausentes no outro. Para integrar esses conjuntos de dados, foi necessário realizar uma união que preservasse todas as variáveis de ambos os anos. Isso foi feito utilizando a função \texttt{merge} do \texttt{pandas} com a opção \texttt{how = 'outer'}, garantindo que todas as observações fossem mantidas, mesmo que algumas variáveis estivessem ausentes em determinada edição.

Após a integração, também foi realizada uma modificação nos nomes das variáveis do Censo Escolar, seguindo o mesmo procedimento adotado para os dados do ENEM.

\subsection{Exploração inicial}\label{exploracao_inicial}

Com os dados no formato desejado, foi realizada uma exploração inicial para compreender a estrutura dos dados, identificar valores ausentes e detectar possíveis inconsistências. Essa exploração incluiu a verificação do número de linhas e colunas de cada conjunto de dados, a análise dos tipos de dados de cada variável e a identificação de valores nulos ou faltantes.

Nessa fasse, também identificado variáveis que poderiam ser descartadas por não serem relevantes para a análise proposta, como variáveis de identificação pessoal dos participantes do ENEM, de local de realização da prova, entre outras. Essas variáveis foram removidas para simplificar o conjunto de dados e focar nas variáveis que realmente contribuiriam para responder às perguntas de pesquisa.

\subsection{Tratamento de Valores Ausentes}\label{tratamento_valores_ausentes}

Para variáveis que possuem valores faltantes, primeiro foi feita uma análise da relevância dessas variáveis para a análise proposta. Caso a variável fosse considerada relevante, foram adotadas estratégias específicas para lidar com os valores ausentes, enquanto que variáveis consideradas irrelevantes foram descartadas.

As variáveis de nota nas quatro provas do ENEM (Ciências da Natureza, Ciências Humanas, Linguagens e Códigos e Matemática) e da redação apresentam valores ausentes para participantes que não compareceram à prova ou que tiveram suas provas anuladas. Essas variáveis serão as variáveis alvo da análise, portanto, observações com valores ausentes nessas variáveis foram removidas do conjunto de dados e cinco conjunto de dados resultantes foram criados, um para cada variável alvo.

