\chapter{Fundamentação Teórica}\label{cap_fundamentacao}

Este capítulo estabelece o contexto teórico e empírico para o estudo, fundamentando a análise no conhecimento acadêmico existente.

% 2.1
\section{O ENEM no Cenário Educacional Brasileiro}\label{enem}

O Exame Nacional do Ensino Médio (ENEM) teve sua primeira edição em 1998, contando com a participação de aproximadamente 115 mil participantes.
Na época, suas notas só eram utilizadas por 2 instituições de ensino superior, número que saltou para 93 instituições no ano seguinte.
A importância do ENEM cresceu com o passar dos anos, alcançando a marca de mais de 1 milhão de participantes na sua quarta edição e tornando-se uma das principais formas de acesso ao ensino superior, com a criação do Programa Universidade Para Todos (ProUni) em 2005 \cite{INEP1}.
Em 2009, com a criação do Sistema de Seleção Unificada (SISU), o ENEM foi reformulado e assumiu o formato que tem hoje: 180 questões objetivas divididas em 4 áreas do conhecimento ((i) Ciências Humanas, (ii) Ciências da Natureza, (iii) Linguagem e Código e (iv) Matemática) e uma Redação.
No ano seguinte, os resultados do ENEM passaram a ser adotados pelo Fundo de Financiamento Estudantil (Fies) e, em 2013, quase todas as instituições federais adotaram o ENEM como critério de seleção.
Duas universidades portuguesas, a Universidade de Coimbra e a Universidade do Algarve, passaram a usar o ENEM como critério de seleção em 2014, número que chegou a 35 instituições portuguesas em 2018 \cite{INEP1}.

É evidente que o ENEM deixou de ser apenas uma ferramenta de avaliação e transformou-se em um instrumento multifacetado que desempenha um papel central na trajetória educacional dos jovens brasileiros.
Além de aferir o desempenho dos estudantes ao final do ensino médio, o ENEM serve como a principal porta de acesso ao ensino superior, sendo a base para o SISU, o ProUni e o Fies \cite{INEP2}.
Essa centralidade significa que qualquer fator que influencie o desempenho no exame tem um impacto direto e significativo nas oportunidades de acesso ao ensino superior e, consequentemente, na mobilidade social dos indivíduos.

Os microdados do ENEM, disponibilizados anualmente pelo Instituto Nacional de Estudos e Pesquisas Educacionais Anísio Teixeira (INEP), representam uma fonte de informação rica e valiosa para pesquisas educacionais \cite{ref_05}.
Esses dados detalhados permitem uma compreensão aprofundada dos padrões de desempenho, das características socioeconômicas dos participantes e dos contextos escolares, possibilitando análises complexas sobre as desigualdades educacionais no país.

% 2.2
\section{Teorias sobre Desigualdades Educacionais: O Capital Cultural de Bourdieu}\label{capital_cultural}

Para compreender a reprodução das desigualdades sociais no sistema educacional, a teoria do capital cultural de Pierre Bourdieu oferece um arcabouço teórico fundamental.
Este argumenta que o sucesso escolar não depende apenas do mérito individual ou da capacidade cognitiva, mas também da posse de diferentes formas de capital: o econômico (posses que o indivíduo tem), o social (relacionamentos que podem ser benéficos ao indivíduo), o simbólico (prestígio/honra) e o cultural (conhecimentos reconhecidos por diplomas e títulos) \cite{ref_10}.

O capital cultural ainda se divide em três estados: (i) o capital cultural incorporado, composto por elementos pessoais como gostos (musicais, artísticos etc.) e domínio de línguas;
(ii) o capital cultural objetivado, composto por posses de livros e obras de arte ou acesso a museus, cinema etc.;
(iii) o capital cultural institucionalizado, caracterizado por diplomas e títulos de conhecimento \cite{ref_10}.

O acúmulo de capital cultural é o que influenciará o desempenho escolar do indivíduo e futuramente seu posicionamento no mercado de trabalho.
Se os dados do ENEM confirmarem a forte influência de variáveis socioeconômicas e de escolaridade parental, isso reforçará a tese da reprodução escolar das desigualdades, sugerindo que o sistema educacional, em vez de ser um equalizador, pode perpetuar as hierarquias sociais.
Isso se manifesta, por exemplo, na forma como a escolaridade da mãe e a renda familiar são fatores relevantes para o desempenho e a dispersão das notas do ENEM \cite{ref_01}.
Oliveira e Cruz (2014) argumentam que a escola, ao reconhecer os alunos mais inteligentes ou aplicados, na verdade está selecionando os alunos com o capital cultural mais diverso e amplo, o que propaga a desigualdade social ao criar os ``mitos de aluno inteligente-brilhante / aluno fracassado-invisível'', fazendo com que ``o próprio oprimido passe a acreditar que não é capaz de ter sucesso por características pessoais e não do sistema.''

% 2.3
\section{Fatores Socioeconômicos e Desempenho no ENEM}\label{fatores_socio_enem}

A literatura é vasta ao associar variáveis socioeconômicas ao desempenho em avaliações de larga escala e o ENEM não é exceção.
As persistentes e quantificáveis desigualdades de desempenho ligadas a fatores socioeconômicos \cite{ref_01} indicam que o acesso a ``experiências educacionais muito mais ricas'' \cite{ref_06} fora do ambiente escolar formal é um preditor poderoso do sucesso no ENEM.
Isso sugere que a escola, por si só, pode não ser capaz de compensar totalmente essas desvantagens de origem e que o campo educacional não é nivelado desde o início.
Estudos sobre o ENEM consistentemente apontam o impacto de diversos fatores:

\begin{itemize}
	\item \textbf{Renda Familiar:} Uma correlação positiva e significativa é observada entre a renda familiar e as notas do ENEM \cite{ref_01}.
    Análises indicam que a diferença na nota de Redação pode ser de até 40\% entre os grupos de menor e maior renda \cite{ref_06}.
	\item \textbf{Raça/Cor:} O desempenho de alunos brancos consistentemente supera o de outros grupos raciais, mesmo quando outras variáveis são controladas \cite{ref_01}.
    Em média, o desempenho de alunos brancos superou o dos demais em menos de 10 pontos nas quatro provas em 2018, controlando outras variáveis \cite{ref_12}.
	\item \textbf{Escolaridade dos Pais/Nível Instrucional da Mãe:} Este é um fator relevante para o desempenho e a dispersão das notas dos estudantes \cite{ref_01}.
    Mães com escolaridade a partir do ensino médio e famílias de renda alta têm um impacto positivo no desempenho \cite{ref_03}.
	\item \textbf{Sexo:} Diferenças de desempenho por sexo são notadas, especialmente na prova de Matemática, com vantagem para os homens (até 36 pontos a mais) \cite{ref_03}.
	\item \textbf{Idade/Atraso Escolar:} O atraso escolar associa-se negativamente ao desempenho.
    Alunos com pelo menos um ano de atraso escolar tiveram, em média, de 16,7 a 29,0 pontos a menos nas provas \cite{ref_12}.
\end{itemize}

% 2.4
\section{Características escolares e o ``Efeito Escola''}\label{carac_escolares}

As características das escolas também exercem influência no desempenho dos estudantes e o conceito de ``Efeito Escola'' busca mensurar a contribuição da instituição de ensino para o desempenho do aluno, além dos fatores individuais e familiares \cite{ref_03}.
Achados relevantes incluem:

\begin{itemize}
	\item \textbf{Dependência Administrativa (Pública vs. Privada):} Alunos de escolas privadas consistentemente superam os de escolas públicas \cite{ref_03}.
    Em Matemática, a diferença pode ser de aproximadamente 83,9 pontos entre alunos de escolas privadas e estaduais \cite{ref_12}.
    Um estudo da UFABC, por exemplo, mostrou que em Matemática, apenas 2,9\% dos estudantes da rede pública atingiram 720 pontos, contra 20\% da rede privada \cite{ref_05}.
	\item \textbf{Atributos Escolares:} Fatores como complexidade de gestão, média de horas-aula, número de alunos por turma, qualidade dos professores (esforço e adequação docente) e o nível socioeconômico médio da escola são importantes \cite{ref_03}.
    O nível socioeconômico médio da escola e a regularidade docente destacam-se como os mais significativos, aumentando a nota em 22,7 pontos para cada nível socioeconômico e em 14,6 para cada nível de regularidade docente em escolas privadas \cite{ref_03}.
\end{itemize}

Embora o ``Efeito Escola'' seja um fator, a literatura sugere que uma grande parte da explicação das notas do ENEM reside em fatores externos ao controle escolar \cite{ref_03}.
Isso significa que, embora a qualidade da escola seja importante, as disparidades socioeconômicas dos alunos e o ambiente familiar podem ter um peso ainda maior.
Isso desafia a ideia de que a escola, por si só, pode reverter completamente as desigualdades de origem, apontando para a necessidade de políticas holísticas que abordem tanto os fatores intraescolares quanto os extraescolares.

% 2.5
\section{Disparidades Regionais e a Participação no ENEM}\label{regiao}

O desempenho no ENEM também exibe variações significativas entre diferentes regiões e unidades da federação \cite{ref_01}.
As disparidades regionais não são apenas geográficas, mas refletem a heterogeneidade socioeconômica e a capacidade de resposta dos sistemas educacionais locais a crises, como a pandemia de COVID-19 \cite{ref_16}.
O período pós-pandemia, em particular, evidenciou um agravamento das desigualdades regionais na participação e no desempenho, com quedas não homogêneas nas taxas de inscrição \cite{ref_15}.
A maior queda proporcional na taxa de inscrição ocorreu na região Sudeste, que de um pico de 63\% em 2016, chegou a apenas 26\% em 2021, tornando-se a região com o menor indicador naquele ano \cite{ref_16}.

% 2.6
\section{Aplicações de Ciência de Dados na Análise do ENEM e resultados obtidos}\label{ciencia_dados_enem}

A aplicação de técnicas de Ciência de Dados e \textit{Machine Learning} na análise dos microdados do ENEM tem se mostrado uma abordagem poderosa para aprofundar a compreensão dos fatores que influenciam o desempenho \cite{ref_01}.
Estudos têm utilizado Regressão Linear, Árvores de Decisão, \textit{Random Forest}, \textit{Boosting}, entre outras técnicas, para predição de notas e identificação de fatores relevantes \cite{ref_03, ref_12, ref_04, ref_01, TCC_Mayra, TCC_Amanda, ref_07}.

Em seu trabalho, Melo \textit{et al.} \cite{ref_01} utilizaram o método de Regressão Linear Múltipla para modelar a média da prova objetiva, média da Redação e as respectivas variâncias.
Seus resultados indicam fortemente que o nível de escolaridade e profissionalização da mãe, a raça do estudante e a renda média da família são relevantes para o desempenho na prova objetiva.
Ao adicionar uma componente espacial, os modelos apresentaram uma melhora, indicando que fatores regionais também influenciam o desempenho do estudante.

Moraes \textit{et al.} \cite{ref_03} também aplicaram o método de Regressão Linear Múltipla para analisar o efeito escola no desempenho em Matemática, considerando variáveis como a quantidade média de alunos por turma, a média de horas-aula por dia e mais algumas variáveis que caracterizam a escola.
Em sua análise exploratória, os autores identificaram as diferenças e semelhanças entre as escolas públicas e privadas, a exemplo do nível socioeconômico médio dos alunos da escola, onde ``87\% das escolas privadas estão nos níveis 5 e 6, enquanto 90\% das escolas públicas possui nível socioeconômico entre os níveis 3 ou 4. Assim, as escolas públicas lidam [...] com alunos com níveis socioeconômicos menores.''

O nível socioeconômico médio dos alunos da escola chega ``a aumentar a nota em 22,7 pontos para cada nível socioeconômico [...] nas escolas privadas e 12,3 pontos [...] nas escolas públicas.'' Essa variável foi construída pelos autores e separada em 6 grupos, onde o grupo 6 reúne as escolas com os alunos de maior nível socioeconômico e o grupo 1 reúne as escolas com os alunos de menor nível socioeconômico.

Os Trabalhos de Conclusão de Curso de Amanda Ferraz \cite{TCC_Amanda} e Mayra Romero \cite{TCC_Mayra}, para este mesmo MBA, aplicaram técnicas mais robustas.
Ferraz utilizou \textit{Random Forest} e \textit{Boosting} para prever a aprovação de participantes do ENEM no SISU para o curso de Medicina, obtendo resultados satisfatórios com Coeficiente de Correlação de Matthews superior a 0,9.
Já Romero desenvolveu e comparou modelos de classificação, incluindo o \textit{Random Forest}, para identificar características socioeconômicas que indicam maior chance de o candidato atingir uma pontuação média acima de 500 pontos no ENEM.
Ela concluiu que o \textit{Random Forest} teve o melhor desempenho e que a renda familiar e o número de computadores são informações que impactam a previsibilidade do modelo.

% 2.7
\section{Métodos de \textit{Machine Learning}}\label{metodos_ml}

Esta seção apresenta, de forma não exaustiva, alguns dos métodos de \textit{Machine Learning} utilizados em trabalhos anteriores relacionados ao tema deste trabalho.
Para isso, foram usadas as referências \cite{ISL, ds_scratch, ml_first_course, rfs} como base para a descrição dos métodos.

\subsection{Regressão Linear}\label{reg_lin} 

A Regressão Linear é um dos pilares do \textit{Machine Learning}, sendo um método fundamental para a modelagem preditiva.
Trata-se de um método paramétrico de aprendizado supervisionado que busca definir um modelo para uma relação linear entre a variável resposta e uma ou mais variáveis preditoras, tendo como objetivo central encontrar a melhor reta (ou hiperplano), em termos de erro na previsão, que descreva essa relação.
A implementação mais básica é expressa pela equação

\begin{equation}
Y = \beta_0 + \beta_1 \times X + \epsilon \label{eq_reg_lin_simples}
\end{equation}

onde $Y$ denota a variável resposta, $X$ a variável preditora, $\beta_0$ o intercepto (o valor de $Y$ quando $X = 0$), $\beta_1$ o coeficiente angular (indicando o impacto de $X$ sobre $Y$) e $\epsilon$ o termo de erro.
Em uma Regressão Linear Múltipla, diversas variáveis independentes são consideradas, cada uma com o seu $\beta_i$ correspondente.
Por trás da Regressão Linear, há algumas premissas adotadas, como a linearidade da relação entre $X$ e $Y$, a independência dos erros, a homocedasticidade e a normalidade dos resíduos.
Essas premissas podem ser interpretadas como desvantagens do modelo de Regressão Linear, por restringir ou até mesmo inviabilizar a sua aplicação.
Já a fácil interpretação, simplicidade e eficiência computacional são algumas das vantagens desse método, que também é muito utilizado como \textit{benchmark} de métodos mais complexos.

\begin{figure}[H]
    \centering
	\caption{Exemplo de uma Regressão Linear simples com dados fictícios} \label{fig:regressao_linear}
    \includegraphics[width=0.7\textwidth]{imagens/regressao_linear.png}
    \par\vspace{0.1cm}
    {\footnotesize Fonte: elaborado pelo autor.}
\end{figure}

\subsection{Árvore de Decisão}\label{arv_dec}

A Árvore de Decisão é um método paramétrico de aprendizado supervisionado que utiliza uma abordagem intuitiva de separação dos dados em grupos semelhantes, através de regras hierárquicas simples e de forma recursiva.
Pode ser utilizado para resolver problemas de regressão, com a média da variável resposta em cada folha, ou de classificação, com a classe mais frequente em cada folha.

O processo de divisão segue uma lógica de ``se-então'': se o dado de entrada tem o valor de uma variável preditora menor ou igual a um limite, então este segue pelo caminho à esquerda;
se não, este segue pelo caminho à direita. É dessa lógica que surge a analogia com árvore, já que as regras usadas para definir o modelo podem ser representadas em um gráfico de árvore binária.
A seleção das melhores divisões é baseada, para os problemas de classificação, em alguma medida de impureza, como a Entropia ou o Índice de Gini.
Já para os problemas de regressão, as divisões são baseadas na redução de alguma medida de erro, como o Erro Quadrático Médio (\textit{Mean Squared Error} - MSE).

Assim como a Regressão Linear, a Árvore de Decisão é um modelo de fácil interpretação, já que as regras de decisão são explícitas e podem ser visualizadas graficamente.
É capaz de lidar com variáveis categóricas e contínuas, o que a torna versátil, não requer normalização dos dados e é robusta a \textit{outliers}.
No entanto, ela é propensa ao \textit{overfitting}, se não aplicadas técnicas de poda, e são instáveis, já que pequenas variações nos dados podem levar a grandes mudanças na estrutura da árvore.

\begin{figure}[H]
	\centering
	\caption{Exemplo de uma Árvore de Decisão com o \textit{dataset Iris}} \label{fig:arvore_decisao}
	\includegraphics[width=0.8\textwidth]{imagens/arvore_decisao_iris.png}	
    \par\vspace{0.1cm}
    \footnotesize Fonte: elaborado pelo autor.
\end{figure}

\subsection{\textit{Random Forest}}\label{rand_for}

O \textit{Random Forest} é um método derivado da Árvore de Decisão, sendo um dos algoritmos mais populares e eficazes em \textit{Machine Learning}.
Ele adota uma abordagem de \textit{ensemble}, ou seja, combina múltiplos modelos para melhorar a precisão e a robustez das previsões.
A ideia central é criar uma ``floresta'' de Árvores de Decisão, onde a decisão final é feita pela média/mediana das previsões para um problema de regressão ou pela classe mais frequente entre todas as árvores no caso de um problema de classificação.

O seu processo de construção envolve duas etapas principais: (i) a amostragem aleatória dos dados, onde cada árvore é treinada em um subconjunto diferente dos dados originais, e (ii) a seleção aleatória de variáveis em cada divisão, o que reduz a correlação entre as árvores e melhora a generalização do modelo. Essa aleatoriedade é crucial para evitar o \textit{overfitting} e aumentar a diversidade entre as árvores.

O \textit{Random Forest} é conhecido por sua alta precisão, capacidade de lidar com grandes conjuntos de dados e variáveis de diferentes tipos, resistência a \textit{outliers} e facilidade de interpretação através da análise da importância das variáveis.
No entanto, ele pode ser computacionalmente intensivo e menos interpretável do que uma única árvore de decisão, já que a combinação de múltiplas árvores torna mais difícil entender as regras subjacentes.

\begin{figure}[H]
	\centering
	\caption{Exemplo de uma \textit{Random Forest} com o \textit{dataset Iris}} \label{fig:random_forest}
	\includegraphics[width = 0.7\textwidth]{imagens/random_forest_iris.png}	
	\par\vspace{0.1cm}
    \footnotesize Fonte: elaborado pelo autor.
\end{figure}

\subsection{\textit{Boosting}}\label{boost}

O \textit{Boosting} é uma técnica de \textit{ensemble}, combinando múltiplos modelos fracos para criar um modelo forte. A ideia central é treinar sequencialmente uma série de modelos, onde cada novo modelo foca em corrigir os erros cometidos pelos modelos anteriores. Alguns algoritmos populares de \textit{Boosting} incluem o \textit{AdaBoost}, \textit{Gradient Boosting} e \textit{XGBoost}.

O \textit{AdaBoost} (\textit{Adaptive Boosting}) foi um dos primeiros algoritmos de \textit{Boosting} e funciona aumentando o peso dos dados de treinamento que foram classificados incorretamente pelos modelos anteriores.
Ao final, as previsões de todos os modelos são combinadas, ponderadas pela precisão de cada modelo.

O \textit{Gradient Boosting} usa uma abordagem de otimização, onde cada novo modelo é treinado especificamente nos resíduos do modelo anterior, buscando minimizá-los.
Os novos aprendizes são adicionados de forma iterativa e geralmente são árvores de decisão de pequeno porte.

O \textit{XGBoost} (\textit{Extreme Gradient Boosting}) é uma implementação otimizada do \textit{Gradient Boosting}, que oferece melhorias significativas em termos de velocidade e desempenho, implementando técnicas de regularização (L1 e L2), tratamento de valores ausentes, paralelização e outras otimizações.

% \begin{figure}[H]
% 	\centering
% 	\caption{Esquema ilustrativo do funcionamento do \textit{AdaBoost}} \label{fig:adaboost}
% 	\includegraphics[width = 1\textwidth]{imagens/adaboost.png}
% 	\par\vspace{0.1cm}
%     \footnotesize Fonte: elaborado pelo autor.
% \end{figure}