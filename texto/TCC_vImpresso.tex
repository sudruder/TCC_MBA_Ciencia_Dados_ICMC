
\documentclass[
% -- opções da classe memoir --
12pt,		% tamanho da fonte
openright,	% capítulos começam em pág ímpar (insere página vazia caso preciso)
twoside,  % para impressão em anverso (frente) e verso. Oposto a oneside - Nota: utilizar \imprimirfolhaderosto*
%oneside, % para impressão em páginas separadas (somente anverso) -  Nota: utilizar \imprimirfolhaderosto
% inclua uma % antes do comando twoside e exclua a % antes do oneside 
a4paper,			% tamanho do papel. 
% -- opções da classe abntex2 --
chapter=TITLE,		% títulos de capítulos convertidos em letras maiúsculas
% -- opções do pacote babel --
english,			% idioma adicional para hifenização
french,				% idioma adicional para hifenização
spanish,			% idioma adicional para hifenização
brazil				% o último idioma é o principal do documento
% {USPSC-classe/USPSC} configura o cabeçalho contendo apenas o número da página
]{USPSC-classe/USPSC}
%]{USPSC-classe/USPSC1}
% Inclua % antes de ]{USPSC-classe/USPSC} e retire a % antes de %]{USPSC-classe/USPSC1} para utilizar o 
% cabeçalho diferenciado para as páginas pares e ímpares:
%- páginas ímpares: com seções ou subseções e o número da página
%- páginas pares: com o número da página e o título do capítulo 
% ---
% ---
% Pacotes básicos - Fundamentais 
% ---
\usepackage[T1]{fontenc}		% Seleção de códigos de fonte.
\usepackage[utf8]{inputenc}		% Codificação do documento (conversão automática dos acentos)

\usepackage{lmodern}			% Usa a fonte Latin Modern

\usepackage{lastpage}			% Usado pela Ficha catalográfica
\usepackage{indentfirst}		% Indenta o primeiro parágrafo de cada seção.
\usepackage{color}				% Controle das cores
\usepackage{graphicx}			% Inclusão de gráficos
\usepackage{float} 				% Fixa tabelas e figuras no local exato
\usepackage{chemfig}            % Para escrever reações químicas
\usepackage{chemmacros}         % Para escrever reações químicas
\usepackage{tikz}				% Para escrever reações químicas e outros
\usetikzlibrary{positioning}
\usepackage{microtype} 			% para melhorias de justificação
\usepackage{pdfpages}
\usepackage{makeidx}			% para gerar índice remissivo
\usepackage{hyphenat}			% Pacote para retirar a hifenizacao DO TEXTO
\usepackage[absolute]{textpos}	% Pacote permite o posicionamento do texto
\usepackage{eso-pic}			% Pacote para incluir imagem de fundo
\usepackage{makebox}			% Pacote para criar caixa de texto
\usepackage{array}				% Necessário para m{} e p{}
\usepackage{ragged2e}			% Para \RaggedRight, melhor que \raggedright em tabelas
\usepackage{refcount}
\usepackage{listings}
\usepackage{xcolor} % Optional, but recommended for code coloring
% ---

\raggedbottom

% Define YAML language for listings
\newcommand\YAMLcolonstyle{\color{red}\mdseries}
\newcommand\YAMLkeystyle{\color{black}\bfseries}
\newcommand\YAMLvaluestyle{\color{blue}\mdseries}

\lstdefinelanguage{yaml}{
  keywords={true,false,null,y,n},
  keywordstyle=\color{darkgray}\bfseries,
  basicstyle=\YAMLkeystyle\tiny, % Changed to tiny to fit page
  sensitive=false,
  comment=[l]{\#},
  morecomment=[s]{/*}{*/},
  commentstyle=\color{purple}\ttfamily,
  stringstyle=\YAMLvaluestyle\ttfamily,
  moredelim=[l][\color{orange}]{\&},
  moredelim=[l][\color{magenta}]{*},
  moredelim=**[il][\YAMLcolonstyle{:}\YAMLvaluestyle]{:},
  morestring=[b]',
  morestring=[b]",
  literate =    {---}{{\ProcessThreeDashes}}3
                {>}{{\textcolor{red}\textgreater}}1     
                {|}{{\textcolor{red}\textbar}}1 
                {\ -\ }{{\mdseries\ -\ }}3,
}

% ---
% Pacotes de citações
% Citações padrão ABNT
% ---
% Sistemas de chamada: autor-data ou numérico.

% Sistema autor-data
% \usepackage[alf, abnt-emphasize=bf, abnt-thesis-year=both, abnt-repeated-author-omit=no, abnt-last-names=abnt, abnt-etal-cite=3, abnt-etal-list=3, abnt-etal-text=it, abnt-and-type=e, abnt-doi=doi, abnt-url-package=none, abnt-verbatim-entry=no]{abntex2cite}
% \bibliographystyle{USPSC-classe/abntex2-alf-USPSC}

% Sistema Numérico
% Para citações numéricas, sistema adotado pelo IFSC, incluir % no início dos comandos acima e retirar a % dos comandos abaixo
\usepackage{cite}              % agrupa citações numéricas consecutivas 
\usepackage[num, abnt-emphasize=bf, abnt-thesis-year=both, abnt-repeated-author-omit=no, abnt-last-names=abnt, abnt-etal-cite=3, abnt-etal-list=3, abnt-etal-text=it, abnt-and-type=e, abnt-doi=doi, abnt-url-package=none, abnt-verbatim-entry=no]{abntex2cite} 
\bibliographystyle{USPSC-classe/abntex2-num-USPSC}
% Se o idioma for o inglês, exclua % no comando acima ou do comando abaixo
%\bibliographystyle{USPSC-classe/abntex2-numeng-USPSC}

% Complementarmente, verifique as instruções abaixo sobre os Pacotes de Nota de rodapé
% ---
% Pacotes de Nota de rodapé
% Configurações de nota de rodapé

% O presente modelo adota o formato numérico para as notas de rodapés quando utiliza o sistema de chamada autor-data para citações e referências. Para utilizar o sistema de chamada numérico para citações e referências, habilitar um dos comandos abaixo.
% Há diversa opções para nota de rodapé no Sistema Numérico.  Para o IFSC, habilitade o comando abaixo.

%\renewcommand{\thefootnote}{\fnsymbol{footnote}}  %Comando para inserção de símbolos em nota de rodapé

% Outras opções para nota de rodapé no Sistema Numérico:
%\renewcommand{\thefootnote}{\alph{footnote}}      %Comando para inserção de letras minúscula em nota de rodapé
%\renewcommand{\thefootnote}{\Alph{footnote}}      %Comando para inserção de letras maiúscula em nota de rodapé
%\renewcommand{\thefootnote}{\roman{footnote}}     %Comando para inserção de números romanos minúsculos  em nota de rodapé
%\renewcommand{\thefootnote}{\Roman{footnote}}     %Comando para inserção de números romanos minúsculos  em nota de rodapé

\renewcommand{\footnotesize}{\small} %Comando para diminuir a fonte das notas de rodapé

% ---
% Pacote para agrupar a citação numérica consecutiva
% Quando for adotado o Sistema Numérico, a exemplo do IFSC, habilite 
% o pacote cite abaixo retirando a porcentagem antes do comando abaixo
%\usepackage[superscript]{cite}	

% ---
% Pacotes adicionais, usados apenas no âmbito do Modelo Canônico do abnteX2
% ---
\usepackage{lipsum}				% para geração de dummy text
% ---

% pacotes de tabelas
\usepackage{multicol}	% Suporte a mesclagens em colunas
\usepackage{multirow}	% Suporte a mesclagens em linhas
\usepackage{longtable}	% Tabelas com várias páginas
\usepackage{threeparttablex}    % notas no longtable
\usepackage{array}

% ----
% Compatibilização com a ABNT NBR 6023:2018 e 10520:2023
% Para tirar <> da URL e tornar as expressões latinas em itálico
\usepackage{USPSC-classe/ABNT6023-10520}
% As demais compatibilizações estão nos arquivos abntex2-alf-USPSC.bst,abntex2-alfeng-USPSC.bst, abntex2-num-USPSC.bst e abntex2-numeng-USPSC.bst, dependendo do idioma do textos e se o sistemas de chamada for autor-data ou numérico, conforme explicitado acima.
% ----

% ---
% DADOS INICIAIS - Define sigla com título, área de concentração e opção do Programa 
% ConsulteDCCp a tabela referente aos Programas, áreas e opções de sua unidade contante do
% arquivo USPSC-Siglas estabelecidas para os Programas de Pós-Graduação nos APÊNDICES B-J
\siglaunidade{ICMC}
\programa{MBACDp}
% Os demais dados deverão ser fornecidos no arquivo USPSC-pre-textual-UUUU ou USPSC-TCC-pre-textual-UUUU, onde UUUU é a sigla da Unidade. 
% Exemplo:USPSC-pre-textual-IFSC.tex
% ---
% Configurações de aparência do PDF final

% alterando o aspecto da cor azul
\definecolor{blue}{RGB}{41,5,195}

% informações do PDF
\makeatletter
\hypersetup{
	%pagebackref=true,
	colorlinks = true,       	% false: boxed links; true: colored links
	linkcolor = black,            	% color of internal links
	citecolor = black,        		% color of links to bibliography
	filecolor = black,      	% color of file links
	urlcolor = black,
	bookmarksdepth = 4	
}
\makeatother
% --- 
% --- 
% Espaçamentos entre linhas e parágrafos 
% --- 

% O tamanho do parágrafo é dado por:
\setlength{\parindent}{1.3cm}

% Controle do espaçamento entre um parágrafo e outro:
\setlength{\parskip}{0.2cm}  % tente também \onelineskip

% ---
% compila o sumário e índice
\makeindex
% ---
% ----
% Início do documento
% ----
\begin{document}

% Seleciona o idioma do documento (conforme pacotes do babel)
\selectlanguage{brazil}
% Se o idioma do texto for inglês, inclua uma % antes do 
%      comando \selectlanguage{brazil} e 
%      retire a % antes do comando abaixo
%\selectlanguage{english}

% Retira espaço extra obsoleto entre as frases.
\frenchspacing 

% --- Formatação dos Títulos
\renewcommand{\ABNTEXchapterfontsize}{\fontsize{12}{12}\bfseries}
\renewcommand{\ABNTEXsectionfontsize}{\fontsize{12}{12}\bfseries}
\renewcommand{\ABNTEXsubsectionfontsize}{\fontsize{12}{12}\normalfont}
\renewcommand{\ABNTEXsubsubsectionfontsize}{\fontsize{12}{12}\normalfont}
\renewcommand{\ABNTEXsubsubsubsectionfontsize}{\fontsize{12}{12}\normalfont}

% ELEMENTOS PRÉ-TEXTUAIS

%Capa do ICMC
\include{USPSC-TA-PreTextual/USPSC-CapaICMC}
\AddToShipoutPicture{\BackgroundBranco}
\includepdf{USPSC-TA-PreTextual/USPSC-PaginaEmBranco.pdf}


% Folha de rosto do ICMC
\imprimirfolharostocar*

% Inserir a ficha catalográfica em pdf
\includepdf{USPSC-TA-PreTextual/USPSC-fichacatalografica.pdf}

% Folha de rosto adicional

\imprimirfolhaderostoadic*

% Inserir errata
% \include{USPSC-TA-PreTextual/USPSC-Errata}


% Inserir folha de aprovação

% A Folha de aprovação é um elemento obrigatório da NBR 4724/2011 (seção 4.2.1.3). 
% Após a defesa/aprovação do trabalho, gere o arquivo folhadeaprovacao.pdf da página assinada pela banca 
% e iclua o arquivo utilizando o comando abaixo:
\includepdf{USPSC-TA-PreTextual/USPSC-folhadeaprovacao.pdf}

% Alternativa para a Folha de Aprovação:
% Se for a sua opção elaborar uma folha de aprovação, insira uma % antes do comando acima que inclui o arquivo folhadeaprovacao.pdf,
% tire o % do comando abaixo e altere o arquivo folhadeaprovacao.tex conforme suas necessidades
% \include{folhadeaprovacao}
% \includepdf{USPSC-TA-PreTextual/USPSC-PaginaEmBranco.pdf}

% Dedicatória
%% USPSC-Dedicatoria.tex
\begin{dedicatoria}
   \vspace*{\fill}
   \centering
   \noindent
   \textit{Dedico este trabalho aos meus pais,\\
   por todo o amor, apoio, incentivos e sacrifícios\\
   que me impulsionaram a trilhar o caminho que trilhei.} \vspace*{\fill}
\end{dedicatoria}
% ---

% Agradecimentos

\begin{agradecimentos}

Agradeço aos meus pais, pelo amor incondicional e pelo incentivo constante aos estudos.

Ao meu orientador, Prof. Prof. Dr. Adriano Kamimura Suzuki, pelas correções precisas e por me ter guiado na estruturação deste trabalho.

Aos professores e colegas do MBA em Ciência de Dados do ICMC-USP, pela troca de experiências e pelo ambiente de aprendizado estimulante que contribuiu imensamente para o meu crescimento profissional.

Aos meus amigos Gêmeos, que me acompanharam nesta jornada desde o começo.

A P., por tudo.

Por fim, agradeço à comunidade de código aberto (\textit{open source}), cujas ferramentas e bibliotecas tornaram este trabalho tecnicamente possível, democratizando o acesso a tecnologias de ponta.

\end{agradecimentos}


% Epígrafe
%% USPSC-Epigrafe.tex
\begin{epigrafe}
    \vspace*{\fill}
	\begin{flushright}
		\textit{``Be yourself, everyone else is already taken.'' \\
		Oscar Wilde}
	\end{flushright}
\end{epigrafe}
% ---

% Resumo
%% USPSC-Resumo.tex

\setlength{\absparsep}{18pt} % ajusta o espaçamento dos parágrafos do resumo		

\begin{resumo}

	\begin{flushleft} 

			\setlength{\absparsep}{0pt} % ajusta o espaçamento da referência	
			\SingleSpacing 
			\imprimirautorabr~~\textbf{\imprimirtituloresumo}.	\imprimirdata. \pageref{LastPage} p. 
			%Substitua p. por f. quando utilizar oneside em \documentclass
			%\pageref{LastPage} f.
			\imprimirtipotrabalho~-~\imprimirinstituicao, \imprimirlocal, \imprimirdata. 
 	\end{flushleft}

\OnehalfSpacing 			

O Exame Nacional do Ensino Médio (ENEM) consolidou-se como a principal porta de entrada para o ensino superior no Brasil, transcendendo seu papel avaliativo para tornar-se um mecanismo central de mobilidade social. No entanto, o desempenho no exame é historicamente marcado por profundas disparidades associadas à origem social dos candidatos. Este trabalho tem como objetivo investigar e quantificar a influência dos fatores socioeconômicos nas notas do ENEM, utilizando técnicas avançadas de Ciência de Dados e \textit{Machine Learning}. Adotando a metodologia CRISP-DM, foram processados e integrados microdados das edições de 2020 a 2023. Foram treinados e avaliados modelos de regressão baseados em árvores de decisão (\textit{Random Forest}, \textit{XGBoost} e \textit{LightGBM}), culminando na construção de um modelo de \textit{ensemble} que apresentou desempenho superior, com erro percentual (MAPE) próximo a 10\% para áreas como Linguagem e Código. A aplicação de técnicas de interpretabilidade, incluindo \textit{Permutation Importance} e curvas de sensibilidade, revelou que a Renda Familiar, a Escolaridade da Mãe e a Quantidade de Computadores são os preditores mais determinantes para o desempenho acadêmico, superando variáveis demográficas isoladas. Os resultados corroboram estatisticamente a teoria do Capital Cultural de Pierre Bourdieu e destacam a exclusão digital como uma barreira contemporânea crítica para o acesso ao ensino superior. Conclui-se que o perfil socioeconômico é um preditor robusto do sucesso escolar no Brasil, evidenciando que o ENEM, embora padronizado, reflete e reproduz as desigualdades estruturais da sociedade. 

 \textbf{Palavras-chave}: \textbf{Palavras-chave}: ENEM. Ciência de Dados. Machine Learning. Desigualdade Educacional. Capital Cultural.
\end{resumo}

% Abstract
%% USPSC-Abstract.tex

\begin{resumo}[Abstract]
 \begin{otherlanguage*}{english}
	\begin{flushleft} 
		\setlength{\absparsep}{0pt} % ajusta o espaçamento dos parágrafos do resumo		
 		\SingleSpacing  		\imprimirautorabr~~\textbf{\imprimirtitleabstract}.	\imprimirdata.  \pageref{LastPage} p. 
		%Substitua p. por f. quando utilizar oneside em \documentclass
		%\pageref{LastPage} f.
		\imprimirtipotrabalhoabs~-~\imprimirinstituicao, \imprimirlocal, 	\imprimirdata. 
 	\end{flushleft}
	\OnehalfSpacing 

The \textit{Exame Nacional do Ensino Médio} - ENEM (National High School Exam) has established itself as the primary gateway to higher education in Brazil, transcending its evaluative role to become a central mechanism of social mobility. However, performance in the exam is historically marked by deep disparities associated with the candidates' social background. This work aims to investigate and quantify the influence of socioeconomic factors on ENEM scores using advanced Data Science and Machine Learning techniques. Adopting the CRISP-DM methodology, microdata from the 2020 to 2023 editions were processed and integrated. Decision tree-based regression models (\textit{Random Forest}, \textit{XGBoost}, and \textit{LightGBM}) were trained and evaluated, culminating in the construction of an \textit{ensemble} model that achieved superior performance, with a Mean Absolute Percentage Error (MAPE) close to 10\% for areas such as Languages and Codes. The application of interpretability techniques, including \textit{Permutation Importance} and sensitivity curves, revealed that Family Income, Mother's Education, and Number of Computers are the most determinant predictors of academic performance, surpassing isolated demographic variables. The results statistically corroborate Pierre Bourdieu's theory of Cultural Capital and highlight the digital divide as a critical contemporary barrier to accessing higher education. It is concluded that the socioeconomic profile is a robust predictor of educational success in Brazil, evidencing that the ENEM, although standardized, reflects and reproduces society's structural inequalities.

   \vspace{\onelineskip}
 
   \noindent 
	\textbf{Keywords}: ENEM. Data Science. Machine Learning. Educational Inequality. Cultural Capital.
 \end{otherlanguage*}
\end{resumo}

% inserir lista de figurass
\pdfbookmark[0]{\listfigurename}{lof}
\listoffigures*
\cleardoublepage

% inserir lista de tabelas
\pdfbookmark[0]{\listtablename}{lot}
\listoftables*
\cleardoublepage

% inserir lista de quadros
\pdfbookmark[0]{\listofquadroname}{loq}
\listofquadro*
\cleardoublepage

% inserir lista de abreviaturas e siglas

\begin{siglas}
	
	\item[ENEM] Exame Nacional do Ensino Médio
	\item[Fies] Fundo de Financiamento Estudantil
	\item[INEP] Instituto Nacional de Estudos e Pesquisas Educacionais Anísio Teixeira
	\item[ProUni] Programa Universidade Para Todos
	\item[SISU] Sistema de Seleção Unificada

\end{siglas}


% inserir lista de símbolos

\begin{simbolos}

  \item[$\alpha$] \textit{Alpha} - Primeiro caractere do alfabeto grego
  \item[$\beta$] \textit{Beta} - Segundo caractere do alfabeto grego
  \item[$\epsilon$] \textit{Epsilon} - Quinto caractere do alfabeto grego
  \item[$\le$] \textit{Menor ou igual a}

\end{simbolos}

% inserir o sumario
\pdfbookmark[0]{\contentsname}{toc}

\tableofcontents*

\cleardoublepage

% ELEMENTOS TEXTUAIS

\textual
% Os capítulos são inseridos como arquivos externos 

% Capítulo 1 - Introdução

\chapter[Introdução]{Introdução}
\label{introducao}


O Exame Nacional do Ensino Médio (ENEM) consolidou-se, na última década, como a principal avaliação educacional do Ensino Médio no Brasil, transcendendo seu papel inicial de termômetro da qualidade da educação básica para se tornar a porta de entrada para o ensino superior em instituições públicas e privadas, através de programas como o Sistema de Seleção Unificada (SISU), o Programa Universidade Para Todos (ProUni) e o Fundo de Financiamento Estudantil (Fies). Sua relevância reside na capacidade de fornecer um panorama detalhado do desempenho dos estudantes, bem como de aspectos socioeconômicos e contextuais que permeiam o ambiente escolar e familiar dos participantes.

Apesar dos esforços contínuos para aprimorar a qualidade da educação no Brasil, persistem desafios significativos, evidenciados pelas variações no desempenho dos estudantes em avaliações de larga escala como o ENEM. A literatura acadêmica aponta para a influência de múltiplos fatores nesse desempenho, que vão desde as condições socioeconômicas das famílias até as características estruturais e pedagógicas das escolas, além das peculiaridades regionais \cite{ref_01}.  A análise estatística de microdados do ENEM entre 2021 e 2023, por exemplo, revela desigualdades estruturais marcantes entre estudantes de escolas públicas e privadas \cite{ref_05}.  A persistência dessas disparidades indica que as desigualdades educacionais no Brasil não são meramente aleatórias, mas profundamente associadas às desigualdades sociais \cite{ref_04}.

A análise aprofundada dos microdados do ENEM, portanto, constitui uma oportunidade ímpar para desvendar a complexa interação entre os fatores socioeconômicos, as características do ambiente escolar e as peculiaridades regionais que moldam o desempenho dos estudantes. Isso permite ir além da simples constatação das disparidades, oferecendo um panorama mais claro de como um instrumento concebido para democratizar o acesso ao ensino superior pode, na prática, atuar como um espelho das desigualdades sociais estruturais e, em certos contextos, até mesmo contribuir para a sua perpetuação, um fenômeno consistentemente observado em análises de dados históricos \cite{ref_05}.  A compreensão desses mecanismos é vital para a formulação de políticas públicas que não apenas mitiguem as lacunas, mas que atuem nas causas-raiz das iniquidades educacionais.

Nesse contexto, este Trabalho de Conclusão de Curso propõe investigar e quantificar a influência dos principais fatores socioeconômicos, características da escola e particularidades regionais no desempenho dos estudantes no ENEM. A pergunta central que guia esta pesquisa é: ``Quais são os principais fatores socioeconômicos, características da escola e particularidades regionais que influenciam o desempenho dos estudantes no ENEM e qual a magnitude da influência de cada um desses conjuntos de fatores nas notas dos participantes?''. O objetivo geral é utilizar os microdados do exame para fornecer \textit{insights} robustos sobre a qualidade da educação básica no Brasil, contribuindo para a identificação de áreas que necessitam de maior atenção e investimento. A quantificação da influência dos fatores, por meio de modelos preditivos e análise de importância de variáveis \cite{ref_05}, é um diferencial crucial. Não se trata apenas de identificar a existência de correlações, mas de medir o grau de impacto, o que é fundamental para a formulação de políticas públicas eficazes e direcionadas.

Para tanto, buscam-se os seguintes objetivos específicos: i) Coletar, pré-processar e realizar uma análise exploratória dos microdados do ENEM \cite{ref_08} e do Censo Escolar \cite{ref_07}, selecionando as variáveis relevantes; ii) Identificar padrões, tendências e correlações entre as variáveis selecionadas e o desempenho dos estudantes; iii) Aplicar técnicas de Ciência de Dados para construir modelos preditivos e determinar a importância relativa de cada grupo de fatores; e iv) Discutir os resultados obtidos, correlacionando-os com a literatura existente e extraindo dados práticos.

A relevância desta pesquisa reside na sua capacidade de oferecer uma análise quantitativa detalhada das correlações entre múltiplos fatores e o desempenho educacional, utilizando uma vasta base de dados. Os dados gerados podem servir como subsídio para educadores, formuladores de políticas públicas e pesquisadores, auxiliando na compreensão das raízes das desigualdades educacionais e na elaboração de estratégias direcionadas para a melhoria do ensino médio no país. A pesquisa não se limita a um exercício acadêmico; ela tem um potencial transformador social ao fornecer dados concretos para subsidiar políticas públicas mais justas e fortalecer a rede pública de ensino \cite{ref_05}. 

% Capítulo 2 - Fundamentação

\chapter{Fundamentação Teórica}\label{cap_exemplos}

Este capítulo estabelece o contexto teórico e empírico para o estudo, fundamentando a análise no conhecimento acadêmico existente.

\section{O ENEM no Cenário Educacional Brasileiro}\label{sec-divisoes}

O Exame Nacional do Ensino Médio (ENEM) teve sua primeira edição em 1998, contando com a participação de aproximadamente 115 mil participantes. Na época, suas notas só eram utilizadas por 2 instituições de ensino superior, saltando para 93 instituições no ano seguinte. A importância do ENEM cresce com o passar dos anos, alcançando a marca de mais de 1 milhão de participantes na sua quarta edição e tornando-se uma das principais formas de acesso ao ensino superior, com a criação do Programa Universidade Para Todos (ProUni) em 2005. \cite{INEP1}

Em 2009, com a criação do Sistema de Seleção Unificada (SISU), o ENEM foi reformulado e assume o formato que tem hoje: 180 questões objetivas divididas em 4 áreas do conhecimento e uma redação. No ano seguinte, os resultados do ENEM passaram a ser adotados pelo Fundo de Financiamento Estudantil (Fies) e em 2013, quase todas as instituições federais adotam o ENEM como critério de seleção. 2 universidade portuguesas, a Universidade de Coimbra e Universidade de Algrave, adotam o ENEM como critério de seleção em 2014, número que chega a 35 instituições portuguesas em 2018. \cite{INEP1}

É evidente que o ENEM deixa de ser apenas uma ferramenta de avaliação e transforma em um instrumento multifacetado que desempenha um papel central na trajetória educacional dos jovens brasileiros. Além de aferir o desempenho dos estudantes ao final do ensino médio, o ENEM serve como a principal porta de acesso ao ensino superior, sendo a base para o SISU, o ProUni e o Fies. \cite{INEP2} Essa centralidade significa que qualquer fator que influencie o desempenho no exame tem um impacto direto e significativo nas oportunidades de acesso ao ensino superior e, consequentemente, na mobilidade social dos indivíduos.

Os microdados do ENEM, disponibilizados anualmente pelo Instituto Nacional de Estudos e Pesquisas Educacionais Anísio Teixeira (INEP), representam uma fonte de informação rica e valiosa para pesquisas educacionais. \cite{ref_05} Esses dados detalhados permitem uma compreensão aprofundada dos padrões de desempenho, das características socioeconômicas dos participantes e dos contextos escolares, possibilitando análises complexas sobre as disparidades educacionais no país.

\section{Teorias sobre Desigualdades Educacionais: O Capital Cultural de Bourdieu}\label{sec-divisoes}

Para compreender a reprodução das desigualdades sociais no sistema educacional, a teoria do capital cultural de Pierre Bourdieu oferece um arcabouço teórico fundamental. \cite{ref_01} Bourdieu argumenta que o sucesso escolar não depende apenas do mérito individual ou da capacidade cognitiva, mas também da posse de diferentes formas de capital: o econômico (posses que o indivíduo tem), social ()




O capital cultural, em particular, manifesta-se em três estados: incorporado (disposições duradouras do corpo e da mente, como hábitos e habilidades adquiridas na família), objetivado (bens culturais como livros, obras de arte) e institucionalizado (títulos acadêmicos, diplomas). \cite{ref_10}

A posse de capital cultural, juntamente com o capital social (redes de relacionamentos e recursos que delas advêm), pode influenciar significativamente o desempenho escolar e o acesso a bens educacionais. Se os dados do ENEM confirmarem a forte influência de variáveis socioeconômicas e de escolaridade parental, isso reforçará a tese da reprodução escolar das desigualdades, sugerindo que o sistema educacional, em vez de ser um equalizador, pode perpetuar as hierarquias sociais. Isso se manifesta, por exemplo, na forma como a escolaridade da mãe e a renda familiar são fatores relevantes para o desempenho e a dispersão das notas do ENEM. \cite{ref_01}













































% exemplo






\section{Resultados de comandos}\label{sec-divisoes}

% ---
\subsection{Tabelas e quadros}

O \textbf{Tutorial do Pacote USPSC para modelos de trabalhos de acad\^emicos em LaTeX - vers\~ao 3.2} apresenta orientações completas e diversas formatações de tabelas, dentre elas a \autoref{tab-ibge}, que é um exemplo de tabela alinhada que pode ser longa ou curta, conforme padrão do Instituto Brasileiro de Geografia e Estatística (IBGE).

%\begin{table}[H]
\begin{table}[htb]
	\IBGEtab{%
		\caption{Frequência anual por categoria de usuários}%
		\label{tab-ibge}
	}{%
		\begin{tabular}{ccc}
			\toprule
			Categoria de Usuários & Frequência de Usuários \\
			\midrule \midrule
			Graduação & 72\% \\
			\midrule 
			Pós-Graduação & 15\% \\
			\midrule 
			Docente & 10\% \\
			\midrule 
			Outras & 3\% \\
			\bottomrule
		\end{tabular}%
	}{%
		\fonte{Elaborada pelos autores.}%
		\nota{Exemplo de uma nota.}%
		\nota[Anotações]{Uma anotação adicional, que pode ser seguida de várias
			outras.}%
		
	}
\end{table}


A formatação do quadro é similar à tabela, mas deve ter suas laterais fechadas e conter as linhas horizontais.
\newpage

% o comando \newpage foi utilizado para forçar a quebra de página

\begin{quadro}[htb]
	\caption{\label{quadro_modelo}Níveis de investigação}
	\begin{tabular}{|p{2.6cm}|p{6.0cm}|p{2.25cm}|p{3.40cm}|}
		\hline
		\textbf{Nível de Investigação} & \textbf{Insumos}  & \textbf{Sistemas de Investigação}  & \textbf{Produtos}  \\
		\hline
		Meta-nível & Filosofia\index{filosofia} da Ciência  & Epistemologia &
		Paradigma  \\
		\hline
		Nível do objeto & Paradigmas do metanível e evidências do nível inferior &
		Ciência  & Teorias e modelos \\
		\hline
		Nível inferior & Modelos e métodos do nível do objeto e problemas do nível inferior & Prática & Solução de problemas \\ 
		\hline
	\end{tabular}
	\begin{flushleft}
		%\fonte{\citeonline{van1986}}
		Fonte: \citeonline{van1986}
	\end{flushleft}
\end{quadro} 


No \textbf{Tutorial do Pacote USPSC para modelos de trabalhos de acad\^emicos em LaTeX - vers\~ao 3.2} são apresentados mais exemplos de quadros.

% ---
\subsection{Figuras}\label{sec_figuras}
% ---
\index{figuras}Figuras podem ser criadas diretamente em \LaTeX,
como o exemplo da \autoref{fig_circulo}.

\begin{figure}[htb]
	\caption{\label{fig_circulo}A delimitação do espaço}
	\begin{center}
		\setlength{\unitlength}{9cm}
		\begin{picture}(1,1)
		\put(0,0){\line(0,1){1}}
		\put(0,0){\line(1,0){1}}
		\put(0,0){\line(1,1){1}}
		\put(0,0){\line(1,2){.5}}
		\put(0,0){\line(1,3){.3333}}
		\put(0,0){\line(1,4){.25}}
		\put(0,0){\line(1,5){.2}}
		\put(0,0){\line(1,6){.1667}}
		\put(0,0){\line(2,1){1}}
		\put(0,0){\line(2,3){.6667}}
		\put(0,0){\line(2,5){.4}}
		\put(0,0){\line(3,1){1}}
		\put(0,0){\line(3,2){1}}
		\put(0,0){\line(3,4){.75}}
		\put(0,0){\line(3,5){.6}}
		\put(0,0){\line(4,1){1}}
		\put(0,0){\line(4,3){1}}
		\put(0,0){\line(4,5){.8}}
		\put(0,0){\line(5,1){1}}
		\put(0,0){\line(5,2){1}}
		\put(0,0){\line(5,3){1}}
		\put(0,0){\line(5,4){1}}
		\put(0,0){\line(5,6){.8333}}
		\put(0,0){\line(6,1){1}}
		\put(0,0){\line(6,5){1}}
		\end{picture}
	\end{center}
	\legend{Fonte: \citeonline{equipeabntex2}}
\end{figure}

Consulte o \textbf{Tutorial do Pacote USPSC para modelos de trabalhos de acad\^emicos em LaTeX - vers\~ao 3.2} para conhecer mais recursos referentes à figuras. 

% ---
\section{Divisões do documento}\label{sec-divisoes-b}
Esta seção exemplifica o uso de divisões de documentos em conformidade com a ABNT NBR 6024  \cite{nbr6024}.
% ---
% ---
\subsection{Divisões do documento: subseção}\label{sec-divisoes-subsection}
% ---

Um exemplo de seção é a \autoref{sec-divisoes-b}. Esta é a \autoref{sec-divisoes-subsection}.

\subsubsection{Divisões do documento: subsubseção}\label{sec-divisoes-subsubsection}

Isto é uma \texttt{subsubsection} do \LaTeX, mas é denominada de ``subseção'' porque no português não temos a palavra ``subsubseção''.

\subsubsection{Divisões do documento: subsubseção}

Isto é outra subsubseção.

\subsection{Divisões do documento: subseção}\label{sec-exemplo-subsec}

Isto é uma subseção.

\subsubsection{Divisões do documento: subsubseção}

Isto é mais uma subsubseção da \autoref{sec-exemplo-subsec}.


\subsubsubsection{Esta é uma subseção de quinto
nível}\label{sec-exemplo-subsubsubsection}

Esta é uma seção de quinto nível. Ela é produzida com o seguinte comando:

\begin{verbatim}
\subsubsubsection{Esta é uma subseção de quinto
nível}\label{sec-exemplo-subsubsubsection}
\end{verbatim}

\subsubsubsection{Esta é outra subseção de quinto nível}\label{sec-exemplo-subsubsubsection-outro}

Esta é outra seção de quinto nível.


\paragraph{Este é um parágrafo numerado}\label{sec-exemplo-paragrafo}

Este é um exemplo de parágrafo numerado. Ele é produzido com o comando de
parágrafo:

\begin{verbatim}
\paragraph{Este é um parágrafo numerado}\label{sec-exemplo-paragrafo}
\end{verbatim}

A numeração entre parágrafos numerados e subsubsubseções são contínuas.

\paragraph{Esta é outro parágrafo numerado}\label{sec-exemplo-paragrafo-outro}

Este é outro parágrafo numerado.

% ---
\subsection{Este é um exemplo de nome de subseção longa que se aplica a seções e demais divisões do documento. Ele deve estar alinhado à esquerda e a segunda e demais linhas devem iniciar logo abaixo da primeira palavra da primeira linha} 

Observe que o alinhamento do título obedece esta regra também no sumário.
	








% Capítulo 3 - Metodologia

% !TEX root = ./TCC_Impresso.tex
% !TEX root = ./TCC_Online.tex

\chapter{Metodologia}\label{cap_metodologia}

Este capítulo detalha a metodologia de trabalho utilizada, apresentando o delineamento da pesquisa, da coleta e processamento dos dados e as técnicas analíticas empregadas para responder às perguntas de pesquisa.

A estrutura metodológica adotada será baseada no modelo CRISP-DM (\textit{Cross-Industry Standard Process for Data Mining}) \cite{Chapman2000CRISPDM}, contendo as etapas de (i) entendimento de negócio, (ii) entendimento dos dados, (iii) preparação dos dados, (iv) modelagem, (v) avaliação e (vi) implantação.

\begin{figure}[H]
    \centering
    \caption{Modelo CRISP-DM}
    \includegraphics[width = 0.7\textwidth]{imagens/crisp-dm.png}
    \label{fig:crisp-dm}
\begin{center}
    \footnotesize Fonte: modificado de Chapman \textit{et al.} \cite{Chapman2000CRISPDM}.
\end{center}
\end{figure}

\section{Entendimento de Negócio}\label{metodologia_entendimento_negocio}

A etapa de entendimento de negócio envolve a definição clara dos objetivos do projeto, a compreensão do contexto em que a pesquisa está inserida, a identificação das partes interessadas, a formulação das perguntas de pesquisa que guiarão a análise dos dados e os resultados que espera-se alcançar.

O foco principal deste trabalho será a formulação de hipóteses relacionadas aos fatores que influenciam o desempenho dos estudantes no ENEM, através da análise dos dados de performance dos participantes e suas características socioeconômicas.

Sendo assim, foi necessário formular perguntas de pesquisa específicas que possam ser respondidas através da análise dos dados disponíveis.

\section{Entendimento dos dados}\label{metodologia_entendimento_dados}

Com as perguntas de pesquisa definidas, a próxima etapa foi encontrar dados que sejam adequados para responder a essas perguntas e estabelecer, primeiramente, uma forma consistente de coleta e armazenamento para em seguida realizar uma rápida compreensão da estrutura dos dados.

Há depender dos tipos de dados a serem utilizados, é necessário submeter o projeto a um comitê de ética em pesquisa para aprovação, garantindo que todos os aspectos éticos relacionados ao uso dos dados sejam devidamente considerados.

Após essa etapa, foi possível identifcar quais os arquivos são relevantes para a análise e quais variáveis dentro desses arquivos serão utilizadas como variáveis preditoras e como variáveis resposta.

\section{Preparação dos dados}\label{metodologia_preparacao_dados}

Com os arquivos relevantes selecionados, passa-se para a etapa de preparação dos dados, que envolve a leitura, limpeza, transformação e integração dos dados para torná-los adequados para a modelagem. Essa etapa é crucial, pois a qualidade dos dados impacta diretamente na eficácia dos modelos preditivos que serão construídos posteriormente.

Para a execução dessa e das etapas posteriores, é necessário preparar um ambiente tecnológico e analítico adequado que permita a manipulação eficiente dos dados e a construção dos modelos preditivos.

Após a leitura dos dados, estes foram integrados em único conjunto de dados para facilitar a análise e a modelagem. Foram realizados os ajustes necessários no esquema dos dados para garantir a consistência e a integridade das informações.

Em seguida, as variáveis foram renomeadas para nomes mais intuitivos e de fácil compreensão e foi analisado se seria necessário criar variáveis contendo alguma transformação das variáveis originais, por exemplo, criar uma variável que transforme os códigos numéricos de variáveis categóricas em rótulos textuais.

Posteriormente, foi realizada uma análise para identificar e tratar valores nulos, removendo ou imputando valores conforme apropriado. Após o tratamento dos valores nulos, os dados foram separados em diferentes conjuntos de dados, cada um correspondente a uma variável resposta específica, garantindo que cada conjunto contenha apenas as observações relevantes para a análise daquela variável e sem valores nulos.

\section{Modelagem}\label{metodologia_modelagem}



\section{Avaliação}\label{metodologia_avaliacao}

\section{Implantação}\label{metodologia_implantacao}




























% \subsection{Coleta, Leitura e Integração dos Dados}\label{sec_metodologia_coleta_integracao_dados}


% Em seguida, foi feita uma modificação no nome das variáveis para nomes que fossem mais intuitivos e de compreensão rápida do conteúdo. Essa modificação foi realizada utilizando o método \texttt{rename}, a partir de um dicionário que mapeava os nomes originais para os novos nomes desejados.

% \subsection{Transformação das Variáveis Categóricas}\label{sec_metodologia_transformacao_variaveis_categoricas}

% Analisando o dicionário de dados, foi possível identificar diversas variáveis categóricas representadas por códigos numéricos. Essas variáveis precisam ser transformadas para garantir que os modelos de \textit{Machine Learning} as interpretem corretamente como categóricas e não como variáveis numéricas.

% Os códigos numéricos foram então substituídos por rótulos textuais mais descritivos utilizando o método \texttt{map} com um dicionário de mapeamento específico para cada variável categórica. Alguns códigos represenram ausência de informação e foram substituídos por \texttt{None} para um futuro tratamento de valores ausentes adequado.

% \subsection{Tratamentos dos Valores Nulos}\label{sec_metodologia_tratamento_nulos}

% Após a integração dos dados, foi realizada uma análise inicial para identificar a presença de valores nulos em cada variável. Dependendo do percentual de valores nulos e da natureza da variável, diferentes estratégias de tratamento foram adotadas.

% As variáveis com mais de 50\% de valores nulos foram removidas do conjunto de dados, pois a alta proporção de valores ausentes poderia comprometer a análise e a modelagem subsequente.

% As variáveis de notas das provas objetivas e da redação passaram por uma análise mais detalhada de valores nulos em conjunto com as variáveis de presença nas respecitivas provas para entender o significado dos valores nulos e decidir o tratamento adequado. No fim, as observações com notas nulas foram removidas, enquanto as observações com notas zero foram mantidas, pois representam casos distintos.

% As demais variáveis com valores nulos foram consideradas de maneira consolidada, ou seja, caso uma observação possua valor ausente em qualquer uma dessas variáveis, ela será considerada como tendo valor ausente. Essas observações foram então removidas do conjunto de dados.

% \subsection{Separação dos Conjuntos de Dados por Variável Resposta}\label{sec_metodologia_separacao_conjuntos_dados}

% Como possuímos cinco variáveis resposta distintas, é necessário preparar os dados para cada uma delas. Assim, foram criados cinco conjuntos de dados distintos, um para cada variável resposta. O filtro aplicado é o mesmo para todos os conjuntos de dados: foram mantidas apenas as observações que possuem presença na prova correspondente. Para a nota da redação, foi utilizada a variável de presença na prova de Linguagens e Códigos, que é quando a redação é aplicada.

% ALém disso, foram retiradas as variáveis de nota das demais provas, assim como as variáveis de presença nas demais provas, para evitar qualquer tipo de \textit{data leakage} durante a modelagem.

% Foi utilizado a estrutura de dicionário para armazenar os cinco conjuntos de dados distintos, facilitando o acesso e a manipulação dos dados durante a análise.

% \subsection{Exploração Inicial}\label{sec_metodologia_exploracao_inicial}

% Com os dados no formato desejado, foi realizada uma exploração inicial para compreender a estrutura dos dados e e detectar possíveis inconsistências.

% Primeiramente, foi analisada a distriuição das notas para cada variável resposta, utilizando histogramas e boxplots. Foi constatado que a nota da redação é aplicada em intervalos de 20 pontos, enquanto que as demais notas são aplicadas de maneira contínua. Isso se dá pela quantidade muito diferente de notas distintas existentes em cada prova: a redação possui 50 valores distintos, enquanto que as demais provas possuem mais de 5.200 valores distintos cada.








% \section{Análise de correlação das Variáveis Preditoras}\label{analise_correlacao_variaveis}

% Antes de iniciar a modelagem, foi realizada uma análise de correlação entre as variáveis preditoras para identificar possíveis multicolinearidades que poderiam afetar o desempenho dos modelos. A correlação foi avaliada utilizando a biblioteca \texttt{phik} \cite{KPMG_PhiK_2024} \cite{Baak2020Phik}, que permite calcular a correlação entre variáveis categóricas e numéricas.

% \subsection{Imputação de Valores Ausentes}\label{imputacao_valores_ausentes}

% Para as variáveis com valores ausentes que não foram removidas, foi adotado o método de \textit{KNN Imputer}, que utiliza a média/moda dos \textit{k} vizinhos mais próximos para preencher os valores ausentes. Esse método foi escolhido por sua capacidade de preservar a distribuição dos dados e considerar a correlação entre as variáveis.

% O \textit{KNN Imputer} foi implementado utilizando a classe \texttt{KNNImputer} da biblioteca \texttt{sklearn.impute}. Para se determinar o hiperparâmetro \texttt{k}, foi feita uma rápida análise de sensibilidade, testando diferentes valores de \texttt{k} e avaliando o impacto na distribuição das variáveis imputadas e na performance dos modelos preditivos. Para cada variável a ser imputada, foi selecionado o valor de \texttt{k} que apresentou o melhor equilíbrio entre acurácia e estabilidade.

% \section{Modelagem}\label{modelagem}

% A modelagem é a etapa onde os dados preparados são utilizados para construir modelos preditivos que possam responder às perguntas de pesquisa formuladas na Seção \ref{entendimento_negocio}. Nesta seção, serão detalhadas as técnicas de modelagem empregadas e os critérios utilizados para a seleção dos modelos.

% \subsection{Seleção de Modelos}\label{selecao_modelos}

% Como as nossas variáveis respostas são numéricas e contínuas, estamos em um problema de regressão de aprendizado supervisionado. Assim, é preciso selecionar modelos de regressão, que são os adequados para prever variáveis contínuas. Os modelos escolhidos foram a Regressão Linear com Regularizações (como \textit{benchmark}) e o XGBoost Regressor \cite{XGBoostDevelopers_Tutorials_2025}, ambos amplamente utilizados devido à sua eficácia e interpretabilidade.

% \subsection{Otimização dos Hiperparâmetros}\label{otimizacao_hiperparametros}

% Para otimizar o desempenho dos modelos selecionados, foi realizada uma busca em grade (\textit{Grid Search}) para identificar os melhores hiperparâmetros. A busca foi realizada utilizando validação cruzada para garantir que os resultados fossem robustos e generalizáveis.

% \subsection{Treino dos Modelos}\label{treino_modelos}

% Definidos os hiperparâmetros ótimos, os modelos foram treinados utilizando o conjunto de dados preparado na Seção \ref{preparacao_dados}. O treinamento foi realizado em validação cruzada de cinco \textit{folds}, utilizando a biblioteca \texttt{cuml} para se beneficiar da aceleração por GPU. Foi selecionado como modelo final aquele modelo que apresentou o melhor desempenho médio nos \textit{folds} de validação para cada variável resposta.

% \subsection{Avaliação dos Modelos}\label{avaliacao_modelos}

% A avaliação dos modelos foi realizada utilizando métricas de desempenho apropriadas para problemas de regressão, como a Raiz do Erro Quadrático Médio (RMSE) e o Coeficiente de Determinação (R²). Essas métricas fornecem uma visão clara da precisão das previsões dos modelos em relação aos valores reais das notas do ENEM.

% Também foram realizadas análises de resíduos para verificar a adequação dos modelos e identificar possíveis padrões não capturados pelas previsões, bem como a análise gráfica das previsões versus os valores reais para avaliar visualmente o desempenho dos modelos.

% \section{Magnitude da Influência das Variáveis Preditoras}\label{medicao_efeito_variaveis}

% Para medir o efeito de cada variável preditora nas notas do ENEM, foram utilizadas técnicas de interpretação de modelos, que permitem identificar quais variáveis têm maior impacto nas previsões dos modelos e como elas influenciam as notas dos estudantes. Entre as técnicas empregadas estão a importância das características (\textit{feature importance}), a análise dos valores SHAP (\textit{SHapley Additive exPlanations}) e os gráficos de dependência parcial (\textit{partial dependence plots}).

% Além disso, foi realizada uma análise de sensibilidade para avaliar como mudanças nas variáveis preditoras afetam as previsões dos modelos. Para isso, foi construída uma matriz de cenários hipotéticos, onde cada variável preditora é alterada sistematicamente enquanto as outras são mantidas constantes. As previsões resultantes foram então analisadas para quantificar o impacto de cada variável nas notas do ENEM.

% \section{Limitações e Considerações Éticas}\label{limitacoes_eticas}

% Algumas limitações devem ser destacadas. Primeiro, apesar do grande volume de dados, a presença de vieses de seleção (por exemplo, diferença entre participantes regulares e treineiros, ou entre ausentes e presentes nas provas) pode influenciar as inferências; as análises procuram mitigar esses efeitos, mas não os eliminam completamente. Segundo, variáveis censuradas, inconsistências de registro e códigos especiais para ausência exigiram tratamentos que podem introduzir perdas de informação.

% Quanto às considerações éticas, todos os dados utilizados são microdados públicos disponibilizados pelo INEP, já anonimizados para preservar a privacidade dos participantes. Recomenda-se cautela na interpretação dos resultados para evitar conclusões simplistas sobre individualidades dos estudantes ou estigmatização de grupos e instituições.

% No Capítulo \ref{cap_resultados} serão apresentados os resultados empíricos da metodologia empregada: (i) a descrição dos dados após o pré-processamento, (ii) a análise de correlação entre as variáveis preditoras, (iii) o desempenho dos modelos preditivos na previsão das notas do ENEM, e (iv) a análise da magnitude da influência das variáveis preditoras nas notas dos estudantes.

% Capítulo 4 - Resultados


\chapter{Resultados}\label{cap_resultados}

% Capítulo N - Conclusão
%% USPSC-Cap3-Conclusao.tex
% ---
% Conclusão
% ---
\chapter{Conclusão}
% ---
% O comando abaixo insere parágrafos aleatórios só para exemplificar
xxxxxxxxxxxxxxxxxxxxxxxxxxxxxxxxxxxxxxxxxxxxxxx



% ----------------------------------------------------------

% ELEMENTOS PÓS-TEXTUAIS
\postextual

% Referências bibliográficas
\bibliography{USPSC-bib/USPSC-modelo-references}

% Apêndices
%% USPSC-Apendice.tex

\begin{apendicesenv}

\partapendices

\chapter{EXEMPLO 1}

\chapter{EXEMPLO 2}

\chapter{EXEMPLO 3}

\end{apendicesenv}
% ---

% Anexos
% \include{USPSC-TA-PosTextual/USPSC-Anexos}

\end{document}
