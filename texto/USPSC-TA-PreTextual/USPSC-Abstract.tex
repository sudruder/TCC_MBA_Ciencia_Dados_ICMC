%% USPSC-Abstract.tex

\begin{resumo}[Abstract]
 \begin{otherlanguage*}{english}
	\begin{flushleft} 
		\setlength{\absparsep}{0pt} % ajusta o espaçamento dos parágrafos do resumo		
 		\SingleSpacing  		\imprimirautorabr~~\textbf{\imprimirtitleabstract}.	\imprimirdata.  \pageref{LastPage} p. 
		%Substitua p. por f. quando utilizar oneside em \documentclass
		%\pageref{LastPage} f.
		\imprimirtipotrabalhoabs~-~\imprimirinstituicao, \imprimirlocal, 	\imprimirdata. 
 	\end{flushleft}
	\OnehalfSpacing 

The \textit{Exame Nacional do Ensino Médio} - ENEM (National High School Exam) has established itself as the primary gateway to higher education in Brazil, transcending its evaluative role to become a central mechanism of social mobility. However, performance in the exam is historically marked by deep disparities associated with the candidates' social background. This work aims to investigate and quantify the influence of socioeconomic factors on ENEM scores using advanced Data Science and Machine Learning techniques. Adopting the CRISP-DM methodology, microdata from the 2020 to 2023 editions were processed and integrated. Decision tree-based regression models (\textit{Random Forest}, \textit{XGBoost}, and \textit{LightGBM}) were trained and evaluated, culminating in the construction of an \textit{ensemble} model that achieved superior performance, with a Mean Absolute Percentage Error (MAPE) close to 10\% for areas such as Languages and Codes. The application of interpretability techniques, including \textit{Permutation Importance} and sensitivity curves, revealed that Family Income, Mother's Education, and Number of Computers are the most determinant predictors of academic performance, surpassing isolated demographic variables. The results statistically corroborate Pierre Bourdieu's theory of Cultural Capital and highlight the digital divide as a critical contemporary barrier to accessing higher education. It is concluded that the socioeconomic profile is a robust predictor of educational success in Brazil, evidencing that the ENEM, although standardized, reflects and reproduces society's structural inequalities.

   \vspace{\onelineskip}
 
   \noindent 
	\textbf{Keywords}: ENEM. Data Science. Machine Learning. Educational Inequality. Cultural Capital.
 \end{otherlanguage*}
\end{resumo}