%% USPSC-Resumo.tex

\setlength{\absparsep}{18pt} % ajusta o espaçamento dos parágrafos do resumo		

\begin{resumo}

	\begin{flushleft} 

			\setlength{\absparsep}{0pt} % ajusta o espaçamento da referência	
			\SingleSpacing 
			\imprimirautorabr~~\textbf{\imprimirtituloresumo}.	\imprimirdata. \pageref{LastPage} p. 
			%Substitua p. por f. quando utilizar oneside em \documentclass
			%\pageref{LastPage} f.
			\imprimirtipotrabalho~-~\imprimirinstituicao, \imprimirlocal, \imprimirdata. 
 	\end{flushleft}

\OnehalfSpacing 			

O Exame Nacional do Ensino Médio (ENEM) consolidou-se como a principal porta de entrada para o ensino superior no Brasil, transcendendo seu papel avaliativo para tornar-se um mecanismo central de mobilidade social. No entanto, o desempenho no exame é historicamente marcado por profundas disparidades associadas à origem social dos candidatos. Este trabalho tem como objetivo investigar e quantificar a influência dos fatores socioeconômicos nas notas do ENEM, utilizando técnicas avançadas de Ciência de Dados e \textit{Machine Learning}. Adotando a metodologia CRISP-DM, foram processados e integrados microdados das edições de 2020 a 2023. Foram treinados e avaliados modelos de regressão baseados em árvores de decisão (\textit{Random Forest}, \textit{XGBoost} e \textit{LightGBM}), culminando na construção de um modelo de \textit{ensemble} que apresentou desempenho superior, com erro percentual (MAPE) próximo a 10\% para áreas como Linguagem e Código. A aplicação de técnicas de interpretabilidade, incluindo \textit{Permutation Importance} e curvas de sensibilidade, revelou que a Renda Familiar, a Escolaridade da Mãe e a Quantidade de Computadores são os preditores mais determinantes para o desempenho acadêmico, superando variáveis demográficas isoladas. Os resultados corroboram estatisticamente a teoria do Capital Cultural de Pierre Bourdieu e destacam a exclusão digital como uma barreira contemporânea crítica para o acesso ao ensino superior. Conclui-se que o perfil socioeconômico é um preditor robusto do sucesso escolar no Brasil, evidenciando que o ENEM, embora padronizado, reflete e reproduz as desigualdades estruturais da sociedade. 

 \textbf{Palavras-chave}: \textbf{Palavras-chave}: ENEM. Ciência de Dados. Machine Learning. Desigualdade Educacional. Capital Cultural.
\end{resumo}