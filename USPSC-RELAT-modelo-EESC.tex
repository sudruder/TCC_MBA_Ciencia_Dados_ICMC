%% USPSC-RELAT-modelo-EESC.tex
% ---------------------------------------------------------------
% USPSC: Modelo de Trabalho Academico (tese de doutorado, dissertacao de
% mestrado e trabalhos monograficos em geral) em conformidade com 
% ABNT NBR 14724:2011: Informacao e documentacao - Trabalhos academicos -
% Apresentacao
%----------------------------------------------------------------
%% Esta é uma customização do abntex2-modelo-trabalho-academico.tex de v-1.9.5 laurocesar 
%% para as Unidades do Campus USP de São Carlos:
%% EESC - Escola de Engenharia de São Carlos
%% IAU - Instituto de Arquitetura e Urbanismo
%% ICMC - Instituto de Ciências Matemáticas e de Computação
%% IFSC - Instituto de Física de São Carlos
%% IQSC - Instituto de Química de São Carlos
%%
%% Este trabalho utiliza a classe USPSC.cls que é mantida pela seguinte equipe:
%% 
%% Coordenação e Programação:
%%   - Marilza Aparecida Rodrigues Tognetti - marilza@sc.usp.br (PUSP-SC)
%%   - Ana Paula Aparecida Calabrez - aninha@sc.usp.br (PUSP-SC)
%% Normalização:
%%   - Brianda de Oliveira Ordonho Sigolo - brianda@usp.br (IAU)
%%   - Eduardo Graziosi Silva - edu.gs@sc.usp.br (EESC)
%%   - Eliana de Cássia Aquareli Cordeiro - eliana@iqsc.usp.br (IQSC)
%%   - Flávia Helena Cassin - cassinp@sc.usp.br (EESC)
%%   - Maria Cristina Cavarette Dziabas - mcdziaba@ifsc.usp.br (IFSC)
%%   - Regina Célia Vidal Medeiros - rcvmat@icmc.usp.br (ICMC)
%%
%% O USPSC-modelo.tex e USPSC-TCC-modelo.tex utilizam diversos arquivos relacionado em 
%% 2.1 Pacote USPSC: Classe USPSC e modelos de trabalhos acadêmicos	do Tutorial do Pascote 
%%  USPSC para modelos de trabalhos de acadêmicos em LaTeX - versão 3.1


%----------------------------------------------------------------
%% Sobre a classe abntex2.cls:
%% abntex2.cls, v-1.9.5 laurocesar
%% Copyright 2012-2015 by abnTeX2 group at https://www.abntex.net.br/ 
%%
%----------------------------------------------------------------
%\documentclass[french,12pt,openany,oneside,openright]{memoir}

\documentclass[
% -- opções da classe memoir --
openany % Capítulos na mesma página
12pt,		% tamanho da fonte
openright,	% capítulos começam em pág ímpar (insere página vazia caso preciso)
%twoside,  % para impressão em anverso (frente) e verso. Oposto a oneside - Nota: utilizar \imprimirfolhaderosto*
oneside, % para impressão em páginas separadas (somente anverso) -  Nota: utilizar \imprimirfolhaderosto
% inclua uma % antes do comando twoside e exclua a % antes do oneside 
a4paper,			% tamanho do papel. 
% -- opções da classe abntex2 --
chapter=TITLE,		% títulos de capítulos convertidos em letras maiúsculas
% -- opções do pacote babel --
english,			% idioma adicional para hifenização
french,				% idioma adicional para hifenização
spanish,			% idioma adicional para hifenização
brazil				% o último idioma é o principal do documento
% {USPSC-classe/USPSC} configura o cabeçalho contendo apenas o número da página
]{USPSC-classe/USPSC}
%]{USPSC-classe/USPSC1}
% Inclua % antes de ]{USPSC-classe/USPSC} e retire a % antes de %]{USPSC-classe/USPSC1} para utilizar o 
% cabeçalho diferenciado para as páginas pares e ímpares:
%- páginas ímpares: com seções ou subseções e o número da página
%- páginas pares: com o número da página e o título do capítulo 
% ---
% ---
% Pacotes básicos - Fundamentais 
% ---
\usepackage[T1]{fontenc}		% Seleção de códigos de fonte.
\usepackage[utf8]{inputenc}		% Codificação do documento (conversão automática dos acentos)
\usepackage{lmodern}			% Usa a fonte Latin Modern
% Para utilizar a fonte Times New Roman, inclua uma % no início do comando acima  "\usepackage{lmodern}"
% Abaixo, tire a % antes do comando  \usepackage{times}
%\usepackage{times}		    	% Usa a fonte Times New Roman	
% Para usar a fonte , lembre-se de tirar a % do comando %\renewcommand{\ABNTEXchapterfont}{\rmfamily}, localizado mais abaixo, logo após "Outras opções para nota de rodapé no Sistema Numérico" 					
\usepackage{lastpage}			% Usado pela Ficha catalográfica
\usepackage{indentfirst}		% Indenta o primeiro parágrafo de cada seção.
\usepackage{color}				% Controle das cores
\usepackage{graphicx}			% Inclusão de gráficos
\usepackage{float} 				% Fixa tabelas e figuras no local exato
\usepackage{chemfig}            % Para escrever reações químicas
\usepackage{chemmacros}         % Para escrever reações químicas
\usepackage{tikz}				% Para escrever reações químicas e outros
\usetikzlibrary{positioning}
\usepackage{microtype} 			% para melhorias de justificação
\usepackage{pdfpages}
\usepackage{makeidx}            % para gerar índice remissivo
\usepackage{hyphenat}          % Pacote para retirar a hifenizacao do texto
\usepackage[absolute]{textpos} % Pacote permite o posicionamento do texto
\usepackage{eso-pic}           % Pacote para incluir imagem de fundo
\usepackage{makebox}           % Pacote para criar caixa de texto
% ---

% ---
% Pacotes de citações
% Citações padrão ABNT
% ---
% Sistemas de chamada: autor-data ou numérico.
% Sistema autor-data
\usepackage[alf, abnt-emphasize=bf, abnt-thesis-year=both, abnt-repeated-author-omit=no, abnt-last-names=abnt, abnt-etal-cite=3, abnt-etal-list=3, abnt-etal-text=it, abnt-and-type=e, abnt-doi=doi, abnt-url-package=none, abnt-verbatim-entry=no]{abntex2cite}
\bibliographystyle{USPSC-classe/abntex2-alf-USPSC}
% Se o idioma for o inglês, inclua % no comando acima e exclua o % do comando abaixo
%\bibliographystyle{USPSC-classe/abntex2-alfeng-USPSC}

% Para o IQSC, que indica todos os autores nas referências, incluir % no início dos comandos acima e retirar a % dos comandos abaixo 
%\usepackage[alf, abnt-emphasize=bf, abnt-thesis-year=both, abnt-repeated-author-omit=no, abnt-last-names=abnt, abnt-etal-cite=3, abnt-etal-list=0, abnt-etal-text=it, abnt-and-type=e, abnt-doi=doi, abnt-url-package=none, abnt-verbatim-entry=no]{abntex2cite} 
%\bibliographystyle{USPSC-classe/abntex2-alf-USPSC}
% Se o idioma for o inglês, exclua % no comando acima ou do comando abaixo
%\bibliographystyle{USPSC-classe/abntex2-alfeng-USPSC}

% Sistema Numérico
% Para citações numéricas, sistema adotado pelo IFSC, incluir % no início dos comandos acima e retirar a % dos comandos abaixo 
%\usepackage{cite}              % agrupa citações numéricas consecutivas
%\usepackage[num, abnt-emphasize=bf, abnt-thesis-year=both, abnt-repeated-author-omit=no, abnt-last-names=abnt, abnt-etal-cite=3, abnt-etal-list=3, abnt-etal-text=it, abnt-and-type=e, abnt-doi=doi, abnt-url-package=none, abnt-verbatim-entry=no]{abntex2cite} 
%\bibliographystyle{USPSC-classe/abntex2-num-USPSC}
% Se o idioma for o inglês, exclua % no comando acima ou do comando abaixo
%\bibliographystyle{USPSC-classe/abntex2-numeng-USPSC}

% Complementarmente, verifique as instruções abaixo sobre os Pacotes de Nota de rodapé
% ---
% Pacotes de Nota de rodapé
% Configurações de nota de rodapé

% O presente modelo adota o formato numérico para as notas de rodapés quando utiliza o sistema de chamada autor-data para citações e referências. Para utilizar o sistema de chamada numérico para citações e referências, habilitar um dos comandos abaixo.
% Há diversa opções para nota de rodapé no Sistema Numérico.  Para o IFSC, habilitade o comando abaixo.

%\renewcommand{\thefootnote}{\fnsymbol{footnote}}  %Comando para inserção de símbolos em nota de rodapé

% Outras opções para nota de rodapé no Sistema Numérico:
%\renewcommand{\thefootnote}{\alph{footnote}}      %Comando para inserção de letras minúscula em nota de rodapé
%\renewcommand{\thefootnote}{\Alph{footnote}}      %Comando para inserção de letras maiúscula em nota de rodapé
%\renewcommand{\thefootnote}{\roman{footnote}}     %Comando para inserção de números romanos minúsculos  em nota de rodapé
%\renewcommand{\thefootnote}{\Roman{footnote}}     %Comando para inserção de números romanos minúsculos  em nota de rodapé

\renewcommand{\footnotesize}{\small} %Comando para diminuir a fonte das notas de rodapé
%Para utilizar a fonte Times New Roman, inclua retire % do início do comando abaixo 
%\renewcommand{\ABNTEXchapterfont}{\rmfamily}

% ---
% Pacote para agrupar a citação numérica consecutiva
% Quando for adotado o Sistema Numérico, a exemplo do IFSC, habilite 
% o pacote cite abaixo retirando a porcentagem antes do comando abaixo
%\usepackage[superscript]{cite}	

% ---
% Pacotes adicionais, usados apenas no âmbito do Modelo Canônico do abnteX2
% ---
\usepackage{lipsum}				% para geração de dummy text
% ---

% pacotes de tabelas
\usepackage{multicol}	% Suporte a mesclagens em colunas
\usepackage{multirow}	% Suporte a mesclagens em linhas
\usepackage{longtable}	% Tabelas com várias páginas
\usepackage{threeparttablex}    % notas no longtable
\usepackage{array}

% ----
% Compatibilização com a ABNT NBR 6023:2018 e 10520:2023
% Para tirar <> da URL e tornar as expressões latinas em itálico
\usepackage{USPSC-classe/ABNT6023-10520}
% As demais compatibilizações estão nos arquivos abntex2-alf-USPSC.bst,abntex2-alfeng-USPSC.bst, abntex2-num-USPSC.bst e abntex2-numeng-USPSC.bst, dependendo do idioma do textos e se o sistemas de chamada for autor-data ou numérico, conforme explicitado acima.
% ----

% ---
% DADOS INICIAIS - Define sigla com título, área de concentração e opção do Programa 
% Consulte a tabela referente aos Programas, áreas e opções de sua unidade contante do
% arquivo USPSC-Siglas estabelecidas para os Programas de Pós-Graduação por Unidade.xlsx 
% ou nos APÊNDICES A-F
\siglaunidade{EESC-RELAT}
\programa{EAMB}
% Os demais dados deverão ser fornecidos no arquivo USPSC-pre-textual-UUUU ou USPSC-TCC-pre-textual-UUUU, onde UUUU é a sigla da Unidade. 
% Exemplo:USPSC-pre-textual-IFSC.tex
% ---
% Configurações de aparência do PDF final
% alterando o aspecto da cor azul
\definecolor{blue}{RGB}{41,5,195}

% informações do PDF
\makeatletter
\hypersetup{
	%pagebackref=true,
	pdftitle={\@title}, 
	pdfauthor={\@author},
	pdfsubject={\imprimirpreambulo},
	pdfcreator={LaTeX with abnTeX2},
	pdfkeywords={abnt}{latex}{abntex}{USPSC}{trabalho acadêmico}, 
	colorlinks=true,       		% false: boxed links; true: colored links
	linkcolor=black,          	% color of internal links
	citecolor=black,        		% color of links to bibliography
	filecolor=black,      		% color of file links
	urlcolor=black,
	%Para habilitar as cores dos links, retire a % antes dos comandos abaixo e inclua a % antes das 4 linhas de comando acima 
	%linkcolor=blue,            	% color of internal links
	%citecolor=blue,        		% color of links to bibliography
	%filecolor=magenta,      		% color of file links
	%urlcolor=blue,
	bookmarksdepth=4	
}
\makeatother
% --- 
% ----
% Início do documento
% ----
\begin{document}

% Seleciona o idioma do documento (conforme pacotes do babel)
\selectlanguage{brazil}
% Se o idioma do texto for inglês, inclua uma % antes do 
%      comando \selectlanguage{brazil} e 
%      retire a % antes do comando abaixo
%\selectlanguage{english}

% Retira espaço extra obsoleto entre as frases.
\frenchspacing 

% --- Formatação dos Títulos
\renewcommand{\ABNTEXchapterfontsize}{\fontsize{12}{12}\bfseries}
\renewcommand{\ABNTEXsectionfontsize}{\fontsize{12}{12}\bfseries}
\renewcommand{\ABNTEXsubsectionfontsize}{\fontsize{12}{12}\normalfont}
\renewcommand{\ABNTEXsubsubsectionfontsize}{\fontsize{12}{12}\normalfont}
\renewcommand{\ABNTEXsubsubsubsectionfontsize}{\fontsize{12}{12}\normalfont}

% ----------------------------------------------------------
% ELEMENTOS PRÉ-TEXTUAIS
% ----------------------------------------------------------
% ---
% Folha de rosto do relatório
% (o * indica impressão em anverso (frente) e verso )
% ---
\imprimirfolhaderostoalt*
%\imprimirfolhaderostoalt

% ---
% ---
% Inserir informações no verso da página 
% Indicar a equipe técnica, elemento opcional, indica a comissão de estudo, colaboradores, coordenação geral entre outros.
%NOTA Pode ser incluída na folha subsequente à folha de rosto, a ficha catalográfica em pdf
% ---
% Altere o arquivo USPSC-VersoPaginaDeRosto-Relatorio.doc com as informações da equipe
% e salve-o como PDF na pasta USPSC-TA-PreTextual 
% e inclua o arquivo utilizando o comando abaixo:

\includepdf{USPSC-TA-PreTextual/USPSC-VersoPaginaDeRosto-Relatorio.pdf}

% ---
% Resumo
% ---
%% USPSC-Resumo.tex

\setlength{\absparsep}{18pt} % ajusta o espaçamento dos parágrafos do resumo		

\begin{resumo}

	\begin{flushleft} 

			\setlength{\absparsep}{0pt} % ajusta o espaçamento da referência	
			\SingleSpacing 
			\imprimirautorabr~~\textbf{\imprimirtituloresumo}.	\imprimirdata. \pageref{LastPage} p. 
			%Substitua p. por f. quando utilizar oneside em \documentclass
			%\pageref{LastPage} f.
			\imprimirtipotrabalho~-~\imprimirinstituicao, \imprimirlocal, \imprimirdata. 
 	\end{flushleft}

\OnehalfSpacing 			

O Exame Nacional do Ensino Médio (ENEM) consolidou-se como a principal porta de entrada para o ensino superior no Brasil, transcendendo seu papel avaliativo para tornar-se um mecanismo central de mobilidade social. No entanto, o desempenho no exame é historicamente marcado por profundas disparidades associadas à origem social dos candidatos. Este trabalho tem como objetivo investigar e quantificar a influência dos fatores socioeconômicos nas notas do ENEM, utilizando técnicas avançadas de Ciência de Dados e \textit{Machine Learning}. Adotando a metodologia CRISP-DM, foram processados e integrados microdados das edições de 2020 a 2023. Foram treinados e avaliados modelos de regressão baseados em árvores de decisão (\textit{Random Forest}, \textit{XGBoost} e \textit{LightGBM}), culminando na construção de um modelo de \textit{ensemble} que apresentou desempenho superior, com erro percentual (MAPE) próximo a 10\% para áreas como Linguagem e Código. A aplicação de técnicas de interpretabilidade, incluindo \textit{Permutation Importance} e curvas de sensibilidade, revelou que a Renda Familiar, a Escolaridade da Mãe e a Quantidade de Computadores são os preditores mais determinantes para o desempenho acadêmico, superando variáveis demográficas isoladas. Os resultados corroboram estatisticamente a teoria do Capital Cultural de Pierre Bourdieu e destacam a exclusão digital como uma barreira contemporânea crítica para o acesso ao ensino superior. Conclui-se que o perfil socioeconômico é um preditor robusto do sucesso escolar no Brasil, evidenciando que o ENEM, embora padronizado, reflete e reproduz as desigualdades estruturais da sociedade. 

 \textbf{Palavras-chave}: \textbf{Palavras-chave}: ENEM. Ciência de Dados. Machine Learning. Desigualdade Educacional. Capital Cultural.
\end{resumo}
% ---
%página em branco
\includepdf{USPSC-TA-PreTextual/USPSC-PaginaEmBranco.pdf}
% ----------------------------------------------------------
% ELEMENTOS TEXTUAIS
% ----------------------------------------------------------
\textual
% Os capítulos são inseridos como arquivos externos
%% Enibir mudança de páginas entre capítulos
\begingroup
\renewcommand{\cleardoublepage}{}
\renewcommand{\clearpage}{}
\setlength\afterchapskip{\lineskip} 
% ---
% Capítulo 1 - Introdução
% ---

\chapter[Introdução]{Introdução}
\label{introducao}


O Exame Nacional do Ensino Médio (ENEM) consolidou-se, na última década, como a principal avaliação educacional do Ensino Médio no Brasil, transcendendo seu papel inicial de termômetro da qualidade da educação básica para se tornar a porta de entrada para o ensino superior em instituições públicas e privadas, através de programas como o Sistema de Seleção Unificada (SISU), o Programa Universidade Para Todos (ProUni) e o Fundo de Financiamento Estudantil (Fies). Sua relevância reside na capacidade de fornecer um panorama detalhado do desempenho dos estudantes, bem como de aspectos socioeconômicos e contextuais que permeiam o ambiente escolar e familiar dos participantes.

Apesar dos esforços contínuos para aprimorar a qualidade da educação no Brasil, persistem desafios significativos, evidenciados pelas variações no desempenho dos estudantes em avaliações de larga escala como o ENEM. A literatura acadêmica aponta para a influência de múltiplos fatores nesse desempenho, que vão desde as condições socioeconômicas das famílias até as características estruturais e pedagógicas das escolas, além das peculiaridades regionais \cite{ref_01}.  A análise estatística de microdados do ENEM entre 2021 e 2023, por exemplo, revela desigualdades estruturais marcantes entre estudantes de escolas públicas e privadas \cite{ref_05}.  A persistência dessas disparidades indica que as desigualdades educacionais no Brasil não são meramente aleatórias, mas profundamente associadas às desigualdades sociais \cite{ref_04}.

A análise aprofundada dos microdados do ENEM, portanto, constitui uma oportunidade ímpar para desvendar a complexa interação entre os fatores socioeconômicos, as características do ambiente escolar e as peculiaridades regionais que moldam o desempenho dos estudantes. Isso permite ir além da simples constatação das disparidades, oferecendo um panorama mais claro de como um instrumento concebido para democratizar o acesso ao ensino superior pode, na prática, atuar como um espelho das desigualdades sociais estruturais e, em certos contextos, até mesmo contribuir para a sua perpetuação, um fenômeno consistentemente observado em análises de dados históricos \cite{ref_05}.  A compreensão desses mecanismos é vital para a formulação de políticas públicas que não apenas mitiguem as lacunas, mas que atuem nas causas-raiz das iniquidades educacionais.

Nesse contexto, este Trabalho de Conclusão de Curso propõe investigar e quantificar a influência dos principais fatores socioeconômicos, características da escola e particularidades regionais no desempenho dos estudantes no ENEM. A pergunta central que guia esta pesquisa é: ``Quais são os principais fatores socioeconômicos, características da escola e particularidades regionais que influenciam o desempenho dos estudantes no ENEM e qual a magnitude da influência de cada um desses conjuntos de fatores nas notas dos participantes?''. O objetivo geral é utilizar os microdados do exame para fornecer \textit{insights} robustos sobre a qualidade da educação básica no Brasil, contribuindo para a identificação de áreas que necessitam de maior atenção e investimento. A quantificação da influência dos fatores, por meio de modelos preditivos e análise de importância de variáveis \cite{ref_05}, é um diferencial crucial. Não se trata apenas de identificar a existência de correlações, mas de medir o grau de impacto, o que é fundamental para a formulação de políticas públicas eficazes e direcionadas.

Para tanto, buscam-se os seguintes objetivos específicos: i) Coletar, pré-processar e realizar uma análise exploratória dos microdados do ENEM \cite{ref_08} e do Censo Escolar \cite{ref_07}, selecionando as variáveis relevantes; ii) Identificar padrões, tendências e correlações entre as variáveis selecionadas e o desempenho dos estudantes; iii) Aplicar técnicas de Ciência de Dados para construir modelos preditivos e determinar a importância relativa de cada grupo de fatores; e iv) Discutir os resultados obtidos, correlacionando-os com a literatura existente e extraindo dados práticos.

A relevância desta pesquisa reside na sua capacidade de oferecer uma análise quantitativa detalhada das correlações entre múltiplos fatores e o desempenho educacional, utilizando uma vasta base de dados. Os dados gerados podem servir como subsídio para educadores, formuladores de políticas públicas e pesquisadores, auxiliando na compreensão das raízes das desigualdades educacionais e na elaboração de estratégias direcionadas para a melhoria do ensino médio no país. A pesquisa não se limita a um exercício acadêmico; ela tem um potencial transformador social ao fornecer dados concretos para subsidiar políticas públicas mais justas e fortalecer a rede pública de ensino \cite{ref_05}. 
% ---
\vspace{1.0cm}

% ---
% Capítulo 2
% ---
\include{USPSC-TA-Textual/USPSC-RELAT-Cap2-Desenvolvimento}
% ---
\vspace{1.0cm}

% ---
% Capítulo 3 - Considerações Finais
% ---
\include{USPSC-TA-Textual/USPSC-RELAT-Cap3-ConsideracoesFinais}
% ---
\vspace{1.0cm}
% Controle do espaçamento entre um parágrafo e outro:
\setlength{\parskip}{1.0cm} 
% ----------------------------------------------------------
% ELEMENTOS PÓS-TEXTUAIS
% ----------------------------------------------------------
\postextual

% ----------------------------------------------------------

% -----------------------------------------------------------
% Referências bibliográficas
% ----------------------------------------------------------
\bibliography{USPSC-bib/USPSC-modelo-references}
% ----------------------------------------------------------
\endgroup
\end{document}
